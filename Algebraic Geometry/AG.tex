\def\module{Algebraic Geometry}
\def\lecturer{Prof Mark Gross}
\def\term{Michaelmas 2020}
\def\cover{}
\def\syllabus{}
\def\thm{section}

\documentclass{article}

% Packages

\usepackage{amssymb}
\usepackage{amsthm}
\usepackage[UKenglish]{babel}
\usepackage{bbm}
\usepackage{commath}
\usepackage{enumitem}
\usepackage{etoolbox}
\usepackage{fancyhdr}
\usepackage[margin=1in]{geometry}
\usepackage{graphicx}
\usepackage[hidelinks]{hyperref}
\usepackage[utf8]{inputenc}
\usepackage{listings}
\usepackage{mathdots}
\usepackage{mathtools}
\usepackage{stmaryrd}
\usepackage{tikz-cd}
\usepackage{csquotes}

% Formatting

\addto\captionsUKenglish{\renewcommand{\abstractname}{Syllabus}}
\DeclareFontFamily{U}{wncyr}{}
\DeclareFontShape{U}{wncyr}{m}{n}{<->wncyr10}{}
\DeclareSymbolFont{cyr}{U}{wncyr}{m}{n}
\DeclareMathSymbol{\Sha}{\mathord}{cyr}{"58}
\delimitershortfall5pt
\ifx\thm\undefined\newtheorem{n}{}\else\newtheorem{n}{}[\thm]\fi
\newcommand\newoperator[1]{\ifcsdef{#1}{\cslet{#1}{\relax}}{}\csdef{#1}{\operatorname{#1}}}
\newcommand\vC{\check{\C}}
\newcommand\vH{\check{\H}}
\setlength{\parindent}{0cm}

% Environments

\theoremstyle{plain}
\newtheorem{algorithm}[n]{Algorithm}
\newtheorem*{algorithm*}{Algorithm}
\newtheorem{algorithm**}{Algorithm}
\newtheorem{conjecture}[n]{Conjecture}
\newtheorem*{conjecture*}{Conjecture}
\newtheorem{conjecture**}{Conjecture}
\newtheorem{corollary}[n]{Corollary}
\newtheorem*{corollary*}{Corollary}
\newtheorem{corollary**}{Corollary}
\newtheorem{lemma}[n]{Lemma}
\newtheorem*{lemma*}{Lemma}
\newtheorem{lemma**}{Lemma}
\newtheorem{proposition}[n]{Proposition}
\newtheorem*{proposition*}{Proposition}
\newtheorem{proposition**}{Proposition}
\newtheorem{theorem}[n]{Theorem}
\newtheorem*{theorem*}{Theorem}
\newtheorem{theorem**}{Theorem}

\theoremstyle{definition}
\newtheorem{aim}[n]{Aim}
\newtheorem*{aim*}{Aim}
\newtheorem{aim**}{Aim}
\newtheorem{axiom}[n]{Axiom}
\newtheorem*{axiom*}{Axiom}
\newtheorem{axiom**}{Axiom}
\newtheorem{condition}[n]{Condition}
\newtheorem*{condition*}{Condition}
\newtheorem{condition**}{Condition}
\newtheorem{definition}[n]{Definition}
\newtheorem*{definition*}{Definition}
\newtheorem{definition**}{Definition}
\newtheorem{example}[n]{Example}
\newtheorem*{example*}{Example}
\newtheorem{example**}{Example}
\newtheorem{exercise}[n]{Exercise}
\newtheorem*{exercise*}{Exercise}
\newtheorem{exercise**}{Exercise}
\newtheorem{fact}[n]{Fact}
\newtheorem*{fact*}{Fact}
\newtheorem{fact**}{Fact}
\newtheorem{goal}[n]{Goal}
\newtheorem*{goal*}{Goal}
\newtheorem{goal**}{Goal}
\newtheorem{law}[n]{Law}
\newtheorem*{law*}{Law}
\newtheorem{law**}{Law}
\newtheorem{plan}[n]{Plan}
\newtheorem*{plan*}{Plan}
\newtheorem{plan**}{Plan}
\newtheorem{problem}[n]{Problem}
\newtheorem*{problem*}{Problem}
\newtheorem{problem**}{Problem}
\newtheorem{question}[n]{Question}
\newtheorem*{question*}{Question}
\newtheorem{question**}{Question}
\newtheorem{warning}[n]{Warning}
\newtheorem*{warning*}{Warning}
\newtheorem{warning**}{Warning}
\newtheorem{acknowledgements}[n]{Acknowledgements}
\newtheorem*{acknowledgements*}{Acknowledgements}
\newtheorem{acknowledgements**}{Acknowledgements}
\newtheorem{annotations}[n]{Annotations}
\newtheorem*{annotations*}{Annotations}
\newtheorem{annotations**}{Annotations}
\newtheorem{assumption}[n]{Assumption}
\newtheorem*{assumption*}{Assumption}
\newtheorem{assumption**}{Assumption}
\newtheorem{conclusion}[n]{Conclusion}
\newtheorem*{conclusion*}{Conclusion}
\newtheorem{conclusion**}{Conclusion}
\newtheorem{claim}[n]{Claim}
\newtheorem*{claim*}{Claim}
\newtheorem{claim**}{Claim}
\newtheorem{notation}[n]{Notation}
\newtheorem*{notation*}{Notation}
\newtheorem{notation**}{Notation}
\newtheorem{note}[n]{Note}
\newtheorem*{note*}{Note}
\newtheorem{note**}{Note}
\newtheorem{remark}[n]{Remark}
\newtheorem*{remark*}{Remark}
\newtheorem{remark**}{Remark}

% Lectures

\newcommand{\lecture}[3]{ % Lecture
  \marginpar{
    Lecture #1 \\
    #2 \\
    #3
  }
}

% Blackboard

\renewcommand{\AA}{\mathbb{A}} % Blackboard A
\newcommand{\BB}{\mathbb{B}}   % Blackboard B
\newcommand{\CC}{\mathbb{C}}   % Blackboard C
\newcommand{\DD}{\mathbb{D}}   % Blackboard D
\newcommand{\EE}{\mathbb{E}}   % Blackboard E
\newcommand{\FF}{\mathbb{F}}   % Blackboard F
\newcommand{\GG}{\mathbb{G}}   % Blackboard G
\newcommand{\HH}{\mathbb{H}}   % Blackboard H
\newcommand{\II}{\mathbb{I}}   % Blackboard I
\newcommand{\JJ}{\mathbb{J}}   % Blackboard J
\newcommand{\KK}{\mathbb{K}}   % Blackboard K
\newcommand{\LL}{\mathbb{L}}   % Blackboard L
\newcommand{\MM}{\mathbb{M}}   % Blackboard M
\newcommand{\NN}{\mathbb{N}}   % Blackboard N
\newcommand{\OO}{\mathbb{O}}   % Blackboard O
\newcommand{\PP}{\mathbb{P}}   % Blackboard P
\newcommand{\QQ}{\mathbb{Q}}   % Blackboard Q
\newcommand{\RR}{\mathbb{R}}   % Blackboard R
\renewcommand{\SS}{\mathbb{S}} % Blackboard S
\newcommand{\TT}{\mathbb{T}}   % Blackboard T
\newcommand{\UU}{\mathbb{U}}   % Blackboard U
\newcommand{\VV}{\mathbb{V}}   % Blackboard V
\newcommand{\WW}{\mathbb{W}}   % Blackboard W
\newcommand{\XX}{\mathbb{X}}   % Blackboard X
\newcommand{\YY}{\mathbb{Y}}   % Blackboard Y
\newcommand{\ZZ}{\mathbb{Z}}   % Blackboard Z

% Brackets

\renewcommand{\eval}[1]{\left. #1 \right|}                     % Evaluation
\newcommand{\br}{\del}                                         % Brackets
\newcommand{\abr}[1]{\left\langle #1 \right\rangle}            % Angle brackets
\newcommand{\fbr}[1]{\left\lfloor #1 \right\rfloor}            % Floor brackets
\newcommand{\st}{\ \middle| \ }                                % Such that
\newcommand{\intd}[4]{\int_{#1}^{#2} \, #3 \, \dif #4}         % Single integral
\newcommand{\iintd}[4]{\iint_{#1} \, #2 \, \dif #3 \, \dif #4} % Double integral

% Calligraphic

\newcommand{\AAA}{\mathcal{A}} % Calligraphic A
\newcommand{\BBB}{\mathcal{B}} % Calligraphic B
\newcommand{\CCC}{\mathcal{C}} % Calligraphic C
\newcommand{\DDD}{\mathcal{D}} % Calligraphic D
\newcommand{\EEE}{\mathcal{E}} % Calligraphic E
\newcommand{\FFF}{\mathcal{F}} % Calligraphic F
\newcommand{\GGG}{\mathcal{G}} % Calligraphic G
\newcommand{\HHH}{\mathcal{H}} % Calligraphic H
\newcommand{\III}{\mathcal{I}} % Calligraphic I
\newcommand{\JJJ}{\mathcal{J}} % Calligraphic J
\newcommand{\KKK}{\mathcal{K}} % Calligraphic K
\newcommand{\LLL}{\mathcal{L}} % Calligraphic L
\newcommand{\MMM}{\mathcal{M}} % Calligraphic M
\newcommand{\NNN}{\mathcal{N}} % Calligraphic N
\newcommand{\OOO}{\mathcal{O}} % Calligraphic O
\newcommand{\PPP}{\mathcal{P}} % Calligraphic P
\newcommand{\QQQ}{\mathcal{Q}} % Calligraphic Q
\newcommand{\RRR}{\mathcal{R}} % Calligraphic R
\newcommand{\SSS}{\mathcal{S}} % Calligraphic S
\newcommand{\TTT}{\mathcal{T}} % Calligraphic T
\newcommand{\UUU}{\mathcal{U}} % Calligraphic U
\newcommand{\VVV}{\mathcal{V}} % Calligraphic V
\newcommand{\WWW}{\mathcal{W}} % Calligraphic W
\newcommand{\XXX}{\mathcal{X}} % Calligraphic X
\newcommand{\YYY}{\mathcal{Y}} % Calligraphic Y
\newcommand{\ZZZ}{\mathcal{Z}} % Calligraphic Z

% Fraktur

\newcommand{\aaa}{\mathfrak{a}}   % Fraktur a
\newcommand{\bbb}{\mathfrak{b}}   % Fraktur b
\newcommand{\ccc}{\mathfrak{c}}   % Fraktur c
\newcommand{\ddd}{\mathfrak{d}}   % Fraktur d
\newcommand{\eee}{\mathfrak{e}}   % Fraktur e
\newcommand{\fff}{\mathfrak{f}}   % Fraktur f
\renewcommand{\ggg}{\mathfrak{g}} % Fraktur g
\newcommand{\hhh}{\mathfrak{h}}   % Fraktur h
\newcommand{\iii}{\mathfrak{i}}   % Fraktur i
\newcommand{\jjj}{\mathfrak{j}}   % Fraktur j
\newcommand{\kkk}{\mathfrak{k}}   % Fraktur k
\renewcommand{\lll}{\mathfrak{l}} % Fraktur l
\newcommand{\mmm}{\mathfrak{m}}   % Fraktur m
\newcommand{\nnn}{\mathfrak{n}}   % Fraktur n
\newcommand{\ooo}{\mathfrak{o}}   % Fraktur o
\newcommand{\ppp}{\mathfrak{p}}   % Fraktur p
\newcommand{\qqq}{\mathfrak{q}}   % Fraktur q
\newcommand{\rrr}{\mathfrak{r}}   % Fraktur r
\newcommand{\sss}{\mathfrak{s}}   % Fraktur s
\newcommand{\ttt}{\mathfrak{t}}   % Fraktur t
\newcommand{\uuu}{\mathfrak{u}}   % Fraktur u
\newcommand{\vvv}{\mathfrak{v}}   % Fraktur v
\newcommand{\www}{\mathfrak{w}}   % Fraktur w
\newcommand{\xxx}{\mathfrak{x}}   % Fraktur x
\newcommand{\yyy}{\mathfrak{y}}   % Fraktur y
\newcommand{\zzz}{\mathfrak{z}}   % Fraktur z

% Maps

\newcommand{\bijection}[7][]{    % Bijection
  \ifx &#1&
    \begin{array}{rcl}
      \displaystyle #2 & \longleftrightarrow & \displaystyle #3 \\
      \displaystyle #4 & \longmapsto         & \displaystyle #5 \\
      \displaystyle #6 & \longmapsfrom       & \displaystyle #7
    \end{array}
  \else
    \begin{array}{ccrcl}
      \displaystyle #1 & : & \displaystyle #2 & \longrightarrow & \displaystyle #3 \\
                       &   & \displaystyle #4 & \longmapsto     & \displaystyle #5 \\
                       &   & \displaystyle #6 & \longmapsfrom   & \displaystyle #7
    \end{array}
  \fi
}
\newcommand{\correspondence}[2]{ % Correspondence
  \cbr{
    \begin{array}{c}
      \displaystyle #1
    \end{array}
  }
  \qquad
  \leftrightsquigarrow
  \qquad
  \cbr{
    \begin{array}{c}
      \displaystyle #2
    \end{array}
  }
}
\newcommand{\function}[5][]{     % Function
  \ifx &#1&
    \begin{array}{rcl}
      \displaystyle #2 & \longrightarrow & \displaystyle #3 \\
      \displaystyle #4 & \longmapsto     & \displaystyle #5
    \end{array}
  \else
    \begin{array}{ccrcl}
      \displaystyle #1 & : & \displaystyle #2 & \longrightarrow & \displaystyle #3 \\
                       &   & \displaystyle #4 & \longmapsto     & \displaystyle #5
    \end{array}
  \fi
}
\newcommand{\functions}[7][]{    % Functions
  \ifx &#1&
    \begin{array}{rcl}
      \displaystyle #2 & \longrightarrow & \displaystyle #3 \\
      \displaystyle #4 & \longmapsto     & \displaystyle #5 \\
      \displaystyle #6 & \longmapsto     & \displaystyle #7
    \end{array}
  \else
    \begin{array}{ccrcl}
      \displaystyle #1 & : & \displaystyle #2 & \longrightarrow & \displaystyle #3 \\
                       &   & \displaystyle #4 & \longmapsto     & \displaystyle #5 \\
                       &   & \displaystyle #6 & \longmapsto     & \displaystyle #7
    \end{array}
  \fi
}

% Matrices

\newcommand{\onebytwo}[2]{      % One by two matrix
  \begin{pmatrix}
    #1 & #2
  \end{pmatrix}
}
\newcommand{\onebythree}[3]{    % One by three matrix
  \begin{pmatrix}
    #1 & #2 & #3
  \end{pmatrix}
}
\newcommand{\twobyone}[2]{      % Two by one matrix
  \begin{pmatrix}
    #1 \\
    #2
  \end{pmatrix}
}
\newcommand{\twobytwo}[4]{      % Two by two matrix
  \begin{pmatrix}
    #1 & #2 \\
    #3 & #4
  \end{pmatrix}
}
\newcommand{\threebyone}[3]{    % Three by one matrix
  \begin{pmatrix}
    #1 \\
    #2 \\
    #3
  \end{pmatrix}
}
\newcommand{\threebythree}[9]{  % Three by three matrix
  \begin{pmatrix}
    #1 & #2 & #3 \\
    #4 & #5 & #6 \\
    #7 & #8 & #9
  \end{pmatrix}
}

% Operators

\newoperator{ab}    % Abelian
\newoperator{Art}   % Artin
\newoperator{Aut}   % Automorphism
\newoperator{BS}    % Baumslag-Solitar
\newoperator{Ca}    % Cartier
\newoperator{cd}    % Cohomological dimension
\newoperator{cell}  % Cell
\newoperator{ch}    % Characteristic
\newoperator{cl}    % Cup length
\newoperator{Cl}    % Class
\newoperator{codim} % Codimension
\newoperator{Coind} % Coinduction
\newoperator{coker} % Cokernel
\newoperator{Cone}  % Cone
\newoperator{Core}  % Core
\newoperator{Crit}  % Critical
\newoperator{ct}    % Compact
\newoperator{cts}   % Continuous
\newoperator{disc}  % Discriminant
\newoperator{div}   % Divisor
\newoperator{Div}   % Divisor group
\newoperator{End}   % Endomorphism
\newoperator{Ext}   % Ext
\newoperator{Fix}   % Fix
\newoperator{Frac}  % Fraction
\newoperator{Fr}    % Frobenius
\newoperator{Gal}   % Galois
\newoperator{girth} % Girth
\newoperator{GL}    % General linear
\newoperator{Gr}    % Grassmannian
\newoperator{Ht}    % Height
\newoperator{Hom}   % Homomorphism
\newoperator{id}    % Identity
\newoperator{im}    % Image
\newoperator{Im}    % Imaginary
\newoperator{Ind}   % Induction
\newoperator{Inf}   % Inflation
\newoperator{Int}   % Integral
\newoperator{Isom}  % Isometry
\newoperator{ker}   % Kernel
\newoperator{Mat}   % Matrix
\newoperator{meas}  % Measure
\newoperator{Mor}   % Morphism
\newoperator{Morse} % Morse
\newoperator{MV}    % Mayer-Vietoris
\newoperator{ns}    % Nonsingular
\newoperator{Obj}   % Object
\newoperator{ord}   % Order
\newoperator{Pic}   % Picard
\newoperator{Proj}  % Projective
\newoperator{PSL}   % Projective special linear
\newoperator{Re}    % Real
\newoperator{Reg}   % Regulator
\newoperator{res}   % Restriction
\newoperator{Res}   % Residue
\newoperator{rk}    % Rank
\newoperator{sep}   % Separable
\newoperator{sign}  % Sign
\newoperator{SL}    % Special linear
\newoperator{Spec}  % Spectrum
\newoperator{Sum}   % Sum
\newoperator{supp}  % Support
\newoperator{Stab}  % Stabiliser
\newoperator{STr}   % Supertrace
\newoperator{Sym}   % Symmetric
\newoperator{taut}  % Tautological
\newoperator{Tg}    % Transgression
\newoperator{tors}  % Torsion
\newoperator{Tr}    % Trace
\newoperator{triv}  % Trivial
\newoperator{ur}    % Unramified

% Roman

\newcommand{\A}{\mathrm{A}}   % Roman A
\newcommand{\B}{\mathrm{B}}   % Roman B
\newcommand{\C}{\mathrm{C}}   % Roman C
\newcommand{\D}{\mathrm{D}}   % Roman D
\newcommand{\E}{\mathrm{E}}   % Roman E
\newcommand{\F}{\mathrm{F}}   % Roman F
\newcommand{\G}{\mathrm{G}}   % Roman G
\renewcommand{\H}{\mathrm{H}} % Roman H
\newcommand{\I}{\mathrm{I}}   % Roman I
\newcommand{\J}{\mathrm{J}}   % Roman J
\newcommand{\K}{\mathrm{K}}   % Roman K
\renewcommand{\L}{\mathrm{L}} % Roman L
\newcommand{\M}{\mathrm{M}}   % Roman M
\newcommand{\N}{\mathrm{N}}   % Roman N
\renewcommand{\O}{\mathrm{O}} % Roman O
\renewcommand{\P}{\mathrm{P}} % Roman P
\newcommand{\Q}{\mathrm{Q}}   % Roman Q
\newcommand{\R}{\mathrm{R}}   % Roman R
\renewcommand{\S}{\mathrm{S}} % Roman S
\newcommand{\T}{\mathrm{T}}   % Roman T
\newcommand{\U}{\mathrm{U}}   % Roman U
\newcommand{\V}{\mathrm{V}}   % Roman V
\newcommand{\W}{\mathrm{W}}   % Roman W
\newcommand{\X}{\mathrm{X}}   % Roman X
\newcommand{\Y}{\mathrm{Y}}   % Roman Y
\newcommand{\Z}{\mathrm{Z}}   % Roman Z

\renewcommand{\a}{\mathrm{a}} % Roman a
\renewcommand{\b}{\mathrm{b}} % Roman b
\renewcommand{\c}{\mathrm{c}} % Roman c
\renewcommand{\d}{\mathrm{d}} % Roman d
\newcommand{\e}{\mathrm{e}}   % Roman e
\newcommand{\f}{\mathrm{f}}   % Roman f
\newcommand{\g}{\mathrm{g}}   % Roman g
\newcommand{\h}{\mathrm{h}}   % Roman h
\renewcommand{\i}{\mathrm{i}} % Roman i
\renewcommand{\j}{\mathrm{j}} % Roman j
\renewcommand{\k}{\mathrm{k}} % Roman k
\renewcommand{\l}{\mathrm{l}} % Roman l
\newcommand{\m}{\mathrm{m}}   % Roman m
\renewcommand{\n}{\mathrm{n}} % Roman n
\renewcommand{\o}{\mathrm{o}} % Roman o
\newcommand{\p}{\mathrm{p}}   % Roman p
\newcommand{\q}{\mathrm{q}}   % Roman q
\renewcommand{\r}{\mathrm{r}} % Roman r
\newcommand{\s}{\mathrm{s}}   % Roman s
\renewcommand{\t}{\mathrm{t}} % Roman t
\renewcommand{\u}{\mathrm{u}} % Roman u
\renewcommand{\v}{\mathrm{v}} % Roman v
\newcommand{\w}{\mathrm{w}}   % Roman w
\newcommand{\x}{\mathrm{x}}   % Roman x
\newcommand{\y}{\mathrm{y}}   % Roman y
\newcommand{\z}{\mathrm{z}}   % Roman z

% Tikz

\tikzset{
  arrow symbol/.style={"#1" description, allow upside down, auto=false, draw=none, sloped},
  subset/.style={arrow symbol={\subset}},
  cong/.style={arrow symbol={\cong}}
}

% Fancy header

\pagestyle{fancy}
\lhead{\module}
\rhead{\nouppercase{\leftmark}}

% Make title

\title{\module}
\author{Lectured by \lecturer \\ Typed by David Kurniadi Angdinata}
\date{\term}

% Macros
\newcommand{\Sch}{\textbf{Sch}}
\newcommand{\Set}{\textbf{Set}}

\begin{document}

\input{../style/cover}

\section{Brief review of classical algebraic geometry and motivation for scheme theory}

\lecture{1}{Friday}{09/10/20}

The following are the main references for the course.
\begin{itemize}
\item R Hartshorne, Algebraic geometry, 1977
\item U Goertz and T Wedhorn, Algebraic geometry I, 2010
\item R Vakil, The rising sea: foundations of algebraic geometry, 2017
\end{itemize}

\subsection{Classical algebraic geometry}

Throughout this discussion, we take the base field $ k $ to be algebraically closed. An \textbf{affine variety} $ V \subseteq \AA^n\br{k} $, where, once one has chosen coordinates, $ \AA^n\br{k} = k^n $, is given by the vanishing of polynomials $ f_1, \dots, f_r \in k\sbr{X_1, \dots, X_n} $. If $ I = \abr{f_1, \dots, f_r} \triangleleft k\sbr{X_1, \dots, X_n} $ is any ideal, we set
$$ \VV\br{I} = \cbr{z \in \AA^n \st \forall f \in I, \ f\br{z} = 0}. $$
First set $ \PP^n\br{k} = \br{k^{n + 1} \setminus \cbr{0}} / k^* $ with \textbf{homogeneous coordinates} $ \br{x_0 : \dots : x_n} $. A \textbf{projective variety} $ V \subseteq \PP^n $ is given by the vanishing of homogeneous polynomials $ F_1, \dots, F_r \in k\sbr{X_0, \dots, X_n} $. If $ I $ is the ideal generated by the homogeneous ideals $ F_i $, that is if $ F \in I $ then so are all its homogeneous parts, we set
$$ \VV\br{I} = \cbr{z \in \PP^n \st \forall F \in I \ \text{homogeneous}, \ F\br{z} = 0}. $$
If $ V = \VV\br{I} \subseteq \AA^n $, set
$$ \II\br{V} = \cbr{f \in k\sbr{X_1, \dots, X_n} \st \forall x \in V, \ f\br{x} = 0}. $$
Observe that $ \VV\br{\II\br{V}} = V $, by tautology, and $ \II\br{\VV\br{I}} \supseteq \sqrt{I} $, which is obvious. Recall that the \textbf{radical} $ \sqrt{I} $ of the ideal $ I $ is defined by $ f \in \sqrt{I} $ if and only if there exists $ m > 0 $ such that $ f^m \in I $. \textbf{Hilbert's Nullstellensatz} states that, noting $ k = \overline{k} $, $ \II\br{\VV\br{I}} = \sqrt{I} $. The \textbf{coordinate ring} is
$$ k\sbr{V} = k\sbr{X_1, \dots, X_n} / \II\br{V}. $$
This may be regarded as the ring of polynomial functions on $ V $, and it is a finitely generated reduced $ k $-algebra. Recall that a \textbf{$ k $-algebra} is a commutative ring containing $ k $ as a subring. It is \textbf{finitely generated} if it is the quotient of a polynomial ring over $ k $, and \textbf{reduced} if $ a^m = 0 $ implies that $ a = 0 $.

\subsection{Why schemes?}

A better question is what is wrong with varieties?
\begin{itemize}
\item With varieties, always work over algebraically closed fields. For example, let $ I = \abr{X^2 + Y^2 + 1} \subseteq \RR\sbr{X, Y} $. Then $ \VV\br{I} = \emptyset $, but $ I $ is a prime ideal, hence radical, so $ \II\br{\VV\br{I}} = \RR\sbr{X, Y} \ne I $.
\item Number theory? Diophantine equations. If $ I \subseteq \ZZ\sbr{X_1, \dots, X_n} $ is an ideal, have $ \VV\br{I} \subseteq \ZZ^n $. For example, $ X^n + Y^n = Z^n $.
\item Why should we only consider radical, or prime, ideals? For example, a natural situation is
$$ X_1 = \VV\br{X - Y^2} \subseteq \AA^2, \qquad X_2 = \VV\br{X} \subseteq \AA^2. $$
Then $ X_1 \cap X_2 = \VV\br{X - Y^2, X} $. Note $ I = \abr{X - Y^2, X} = \abr{X, Y^2} $ is not a radical ideal, because $ Y \notin I $ and $ Y^2 \in I $ so $ Y \not\in \sqrt{I} $. Recall the coordinate ring of $ X_i $ is $ k\sbr{X_i} = k\sbr{X, Y} / I_i $. Then $ k\sbr{X_1 \cap X_2} = k\sbr{X, Y} / \abr{X, Y^2} \cong k\sbr{Y} / \abr{Y^2} $. So thinking of the coordinate ring of $ X_1 \cap X_2 $ as functions on $ X_1 \cap X_2 $, we have a function $ Y $ whose square is zero, but is not itself zero.
\end{itemize}

\pagebreak

\subsection{Categorical philosophy}

What is a point? In the category of sets, objects are sets, and if $ A $ and $ B $ are sets, then morphisms are $ \Hom\br{A, B} $, the set of maps $ f : A \to B $. Let $ * $ be a one-element set. Then the elements of any set $ X $ are in one-to-one correspondence with $ \Hom\br{*, X} $. In the category of affine varieties, objects are affine varieties and morphisms are $ \Hom\br{X, Y} = \Hom_{\text{$ k $-alg}}\br{k\sbr{Y}, k\sbr{X}} $. In this category, a point is a single point with coordinate ring $ k $. Giving a morphism
$$ \cbr{\text{point}} \to X = \VV\br{I} \subseteq \AA^n, \qquad I \subseteq k\sbr{X_1, \dots, X_n}, $$
for $ I $ a radical ideal, is the same as giving a homomorphism
$$ \function[\phi]{k\sbr{X} = k\sbr{X_1, \dots, X_n} / I}{k}{X_i}{a_i}. $$
Note that $ \phi $ vanishes in $ I $ if and only if $ f\br{a_1, \dots, a_n} = 0 $ for all $ f \in I $, which is if and only if $ \br{a_1, \dots, a_n} \in \VV\br{I} = X $. Note $ \phi $ is surjective, and hence $ \ker \phi $ is a maximal ideal. With $ k $ algebraically closed, the maximal ideals at $ k\sbr{X} $ are all of the form $ \abr{X_1 - a_1, \dots, X_n - a_n} $ for $ \br{a_1, \dots, a_n} \in X $, a consequence of Hilbert's Nullstellensatz. That is, there exists one-to-one correspondences
$$ \cbr{\text{points of} \ X} \quad \leftrightsquigarrow \quad \cbr{\text{$ k $-algebra homomorphisms} \ \phi : k\sbr{X} \to k} \quad \leftrightsquigarrow \quad \cbr{\text{maximal ideals of} \ k\sbr{X}}. $$

\subsection{Solutions over non-algebraically closed fields}

What if $ k $ is not algebraically closed? We may want to consider solutions not just in $ k^n = \AA^n $ but $ \br{k'}^n $ for $ k' $ any field extension of $ k $. That is, we may consider $ k $-algebra homomorphisms
$$ \function[\phi]{k\sbr{X} = k\sbr{X_1, \dots, X_r} / I}{k'}{X_i}{a_i}. $$
This gives a tuple $ \br{a_1, \dots, a_n} \in \br{k'}^n $ with $ f\br{a_1, \dots, a_n} = 0 $ for all $ f \in I $. Then $ \phi $ need not be surjective, so can only say the image of $ \phi $ is a subring of a field, hence an integral domain. Thus $ \ker \phi $ is a prime ideal, and maximal if and only if $ \im \phi $ is a field.

\begin{example*}
The $ \RR $-algebra homomorphism
$$ \functions[\phi]{\RR\sbr{X, Y} / \abr{X^2 + Y^2 + 1}}{\CC}{X}{0}{Y}{i} $$
is surjective with kernel $ \abr{X, Y^2 + 1} $, since $ \RR\sbr{Y} / \abr{Y^2 + 1} \cong \CC $. This is a maximal ideal but is not of the form $ \abr{X - a, Y - b} $ for $ \br{a, b} \in \RR^2 $. If instead we considered the map
$$ \functions{\RR\sbr{X, Y} / \abr{X^2 + Y^2 + 1}}{\CC}{X}{0}{Y}{-i}, $$
we get the same kernel. That is, $ \br{0, i} $ and $ \br{0, -i} $ are solutions to $ X^2 + Y^2 + 1 $, but they correspond to the same maximal ideal. In fact, this maximal ideal corresponds to a Galois orbit of $ \Gal\br{\CC / \RR} $ of solutions.
\end{example*}

\lecture{2}{Monday}{12/10/20}

There are more exotic points by taking even bigger fields.

\begin{example*}
Let $ k\br{X} $ be the field of fractions of $ k\sbr{X} = \RR\sbr{X, Y} / \abr{X^2 + Y^2 + 1} $. There is an inclusion
$$ \functions{k\sbr{X}}{k\br{X}}{f}{\tfrac{f}{1}}{\br{X, Y}}{\br{X, Y}}. $$
The kernel of this map is zero. This gives a solution to the equation $ X^2 + Y^2 + 1 = 0 $ with coordinates in the field $ k\br{X} $. This solution is $ \br{X, Y} \in \AA^2\br{k\br{X}} $.
\end{example*}

The moral is that once we start looking at solutions to equation over any field, then we get maps $ k\sbr{X} \to k' $ with kernel not necessarily maximal. What about solutions over rings?

\pagebreak

\begin{example*}
Let $ A = \ZZ\sbr{X_1, \dots, X_n} / I $, and let $ R $ be any commutative ring. We define an $ R $-valued point of $ \Spec A $ to be a ring homomorphism
$$ \function{A}{R}{X_i}{r_i}. $$
Then $ f\br{r_1, \dots, r_n} = 0 $ for all $ f \in I $. This gives a lot of flexibility. For example,
\begin{itemize}
\item $ R = \ZZ $ gives diophantine equations,
\item $ R = \FF_p $ gives solutions modulo $ p $, and
\item $ R = \QQ $ gives rational solutions.
\end{itemize}
\end{example*}

Take this to its logical conclusion. Let $ A $ be a ring, where all rings are commutative in this course. Given $ A $, we hope for some geometric object $ \Spec A $, the \textbf{spectrum} of $ A $. For a ring $ R $, the set of \textbf{$ R $-valued points} of $ X $ is
$$ X\br{R} = \Hom_{\text{ring}}\br{A, R}. $$
A morphism $ X = \Spec A \to Y = \Spec B $ should be the same thing as giving a morphism $ \phi : B \to A $. Define the category of \textbf{affine schemes} to be the opposite category to the category of rings. Define a \textbf{scheme} to be something which is locally isomorphic to an affine scheme. By analogy, a \textbf{manifold} is a topological space with an open cover $ \cbr{U_i} $ with each $ U_i $ homeomorphic to an open subset of $ \RR^n $. To make sense of the definition of schemes, we need a lot of language.

\subsection{Spectrum of a ring}

\begin{definition*}
Let $ A $ be a ring. Then
$$ \Spec A = \cbr{\ppp \subseteq A \st \ppp \ \text{a prime ideal}}. $$
For $ I \subseteq A $ an ideal, define
$$ \VV\br{I} = \cbr{\ppp \subseteq A \st \ppp \ \text{prime}, \ \ppp \supseteq I}. $$
\end{definition*}

\begin{proposition}
The sets $ \VV\br{I} $ form the closed sets of a topology on $ \Spec A $, called the \textbf{Zariski topology}.
\end{proposition}

\begin{proof}
\hfill
\begin{itemize}
\item $ \VV\br{A} = \emptyset $.
\item $ \VV\br{0} = \Spec A $.
\item If $ \cbr{I_i}_{i \in J} $ is a collection of ideals, then
$$ \VV\br{\sum_{i \in J} I_i} = \bigcap_{i \in J} \VV\br{I_i}. $$
\item Claim that
$$ \VV\br{I_1 \cap I_2} = \VV\br{I_1} \cup \VV\br{I_2}. $$
\begin{itemize}
\item[$ \supseteq $] Obvious.
\item[$ \subseteq $] If $ \ppp \supseteq I_1 \cap I_2 $ is prime, then $ \ppp \supseteq I_1 $ or $ \ppp \supseteq I_2 $. See Atiyah-Macdonald, Proposition 1.11.ii. \footnote{Exercise: try to prove without looking up}
\end{itemize}
\end{itemize}
\end{proof}

\begin{example*}
Let $ A = k\sbr{X_1, \dots, X_n} $ with $ k $ algebraically closed and $ I \subseteq A $ an ideal. Then the maximal ideals $ \mmm $ of $ A $ containing $ I $ are in one-to-one correspondence with the zero set of $ I $ in $ \AA^n\br{k} $, so
$$ \correspondence{\abr{X_1 - a_1, \dots, X_n - a_n} \supseteq I, \ a_i \in k}{\br{a_1, \dots, a_n} \in \VV\br{I} \subseteq \AA^n\br{k}}. $$
\end{example*}

The new $ \VV\br{I} $ now extends this notion of zero set by including possible other prime ideals.

\begin{example*}
If $ k $ is a field, $ \Spec k = \cbr{0} $, so the topological space cannot see the field.
\end{example*}

We fix this by also thinking about what functions are on these spaces.

\pagebreak

\section{Sheaves}

Fix a topological space $ X $.

\subsection{Sheaves}

\begin{definition*}
A \textbf{presheaf} $ \FFF $ on $ X $ consists of the following data.
\begin{itemize}
\item For every open set $ U \subseteq X $ an abelian group $ \FFF\br{U} $.
\item Whenever given an inclusion $ V \subseteq U \subseteq X $, a \textbf{restriction map} $ \rho_{UV} : \FFF\br{U} \to \FFF\br{V} $, a homomorphism, such that
\begin{itemize}
\item $ \rho_{UU} = \id_{\FFF\br{U}} $, and
\item if $ W \subseteq V \subseteq U $, then $ \rho_{UW} = \rho_{VW} \circ \rho_{UV} $.
\end{itemize}
\end{itemize}
\end{definition*}

\begin{remark*}
Can think of a presheaf as a contravariant functor from the category of open sets of $ X $, the category whose objects are open subsets of $ X $ and whose morphisms are inclusions of open sets, to the category of abelian groups. Can replace the category of abelian groups with any desired category, such as commutative rings.
\end{remark*}

\begin{definition*}
A \textbf{morphism of presheaves} $ f : \FFF \to \GGG $ is a collection of homomorphisms $ f_U : \FFF\br{U} \to \GGG\br{U} $ such that for all $ V \subseteq U $ the diagram
$$
\begin{tikzcd}
\FFF\br{U} \arrow{r}{f_U} \arrow{d}[swap]{\rho_{UV}} & \GGG\br{U} \arrow{d}{\rho_{UV}} \\
\FFF\br{V} \arrow{r}[swap]{f_V} & \GGG\br{V}
\end{tikzcd}
$$
is commutative.
\end{definition*}

\begin{definition*}
A presheaf $ \FFF $ is a \textbf{sheaf} if it satisfies the following additional axioms.
\begin{enumerate}[label=S\arabic*.]
\item If $ U \subseteq X $ is covered by an open cover $ \cbr{U_i} $ and $ s \in \FFF\br{U} $ satisfies $ \eval{s}_{U_i} = \rho_{UU_i}\br{s} = 0 $ for all $ i $, then $ s = 0 $.
\item If $ U $ and $ \cbr{U_i} $ are as in S1 and $ s_i \in \FFF\br{U_i} $ such that $ \eval{s_i}_{U_i \cap U_j} = \eval{s_j}_{U_i \cap U_j} $ for all $ i $ and $ j $, then there exists $ s \in \FFF\br{U} $ with $ \eval{s}_{U_i} = s_i $ for all $ i $.
\end{enumerate}
\end{definition*}

\begin{remark*}
\hfill
\begin{itemize}
\item If $ \FFF $ is a sheaf, then $ \emptyset \subseteq X $ is covered by the empty covering, and hence $ \FFF\br{\emptyset} = 0 $.
\item S1 and S2 together can be described as saying, given $ U $ and $ \cbr{U_i}_{i \in I} $,
$$ 0 \to \FFF\br{U} \xrightarrow{\alpha} \prod_{i \in I} \FFF\br{U_i} \overset{\beta_1}{\underset{\beta_2}{\rightrightarrows}} \prod_{i, j} \FFF\br{U_i \cap U_j} $$
is exact, where
$$ \alpha\br{s} = \br{\eval{s}_{U_i}}_{i \in I}, \qquad \beta_1\br{\br{s_i}_{i \in I}} = \br{\eval{s_i}_{U_i \cap U_j}}_{i, j}, \qquad \beta_2\br{\br{s_i}_{i \in I}} = \br{\eval{s_j}_{U_i \cap U_j}}_{i, j}. $$
Exactness means
\begin{itemize}
\item $ \alpha $ is injective, which is S1,
\item $ \beta_1 \circ \alpha = \beta_2 \circ \alpha $, and
\item for any $ \br{s_i} \in \prod_{i \in I} \FFF\br{U_i} $, with $ \beta_1\br{\br{s_i}} = \beta_2\br{\br{s_i}} $, there exists $ s \in \FFF\br{U} $ with $ \alpha\br{s} = \br{s_i} $, which is S2.
\end{itemize}
\end{itemize}
\end{remark*}

\pagebreak

\subsection{Examples}

\lecture{3}{Wednesday}{14/10/20}

\begin{example*}
\hfill
\begin{itemize}
\item Let $ X $ be any topological space, and let
$$ \FFF\br{U} = \cbr{\text{continuous functions} \ U \to \RR}. $$
This is a sheaf, by
$$ \function[\rho_{UV}]{\FFF\br{U}}{\FFF\br{V}}{f}{\eval{f}_V}. $$
\begin{enumerate}[label=S\arabic*.]
\item A continuous function is zero if it is zero on every open set of a cover.
\item Continuous functions can be glued.
\end{enumerate}
\item Let $ X = \CC $ with the Euclidean topology, and let
$$ \FFF\br{U} = \cbr{f : U \to \CC \st f \ \text{is a bounded analytic function}}. $$
This is a presheaf. It satisfies S1, and does not satisfy S2. For example, consider the cover $ \cbr{U_i}_{i \in \cbr{1, 2, \dots}} $ of $ \CC $ given by $ U_i = \cbr{z \in \CC \st \abs{z} < i} $ and
$$ \function[s_i]{U_i}{\CC}{z}{z}. $$
Note if $ i < j $, then $ U_i \cap U_j = U_i $ and $ \eval{s_i}_{U_i \cap U_j} = \eval{s_j}_{U_i \cap U_j} $. But if we glue we get the function $ z : \CC \to \CC $, which is not bounded. Note $ \FFF\br{\CC} = \CC $.
\item Take any group $ G $ and set $ \FFF\br{U} = G $ for any open set $ U $. This is called the \textbf{constant presheaf}. This is not a sheaf. Let $ U = U_1 \sqcup U_2 $. If we wanted a sheaf,
$$
\begin{tikzcd}
\FFF\br{U_1} = G \arrow{dr} & & \FFF\br{U_2} = G \arrow{dl} \\
& \FFF\br{U_1 \cap U_2} = \FFF\br{\emptyset} = 0 &
\end{tikzcd},
$$
so if S2 is satisfied, would want $ s_1 \in \FFF\br{U_1} $ and $ s_2 \in \FFF\br{U_2} $ to glue. We would then want to have $ \FFF\br{U} = G \times G $. Now give $ G $ the discrete topology, and define instead
$$ \FFF\br{U} = \cbr{f : U \to G \ \text{continuous}}, $$
that is $ f $ is locally constant. That is, if $ x \in U $, there exists a neighbourhood $ x \in V \subseteq U $ with $ \eval{f}_V $ constant. This is called the \textbf{constant sheaf} and if $ U $ is non-empty and connected, then $ \FFF\br{U} = G $.
\item If $ X $ is an algebraic variety, and $ U \subseteq X $ is a Zariski open subset, define
$$ \OOO_X\br{U} = \cbr{f : U \to k \st f \ \text{regular function}}. $$
Roughly $ f $ is \textbf{regular} means that every point of $ U $ has an open neighbourhood on which $ f $ is expressed as a ratio of polynomials $ g / h $ with $ h $ non-vanishing on the neighbourhood. Then $ \OOO_X $ is a sheaf, called the \textbf{structure sheaf} of $ X $.
\end{itemize}
\end{example*}

\subsection{Stalks}

\begin{definition*}
Let $ \FFF $ be a presheaf on $ X $. Let $ p \in X $. Then the \textbf{stalk} of $ \FFF $ at $ p $ is
$$ \FFF_p = \cbr{\br{U, s} \st U \subseteq X \ \text{is an open neighbourhood of} \ p, \ s \in \FFF\br{U}} / \equiv, $$
where $ \br{U, s} \equiv \br{V, s'} $ if there exists $ W \subseteq U \cap V $ also a neighbourhood of $ p $ such that $ \eval{s}_W = \eval{s'}_W $. An equivalence class of a pair $ \br{U, s} $ is called a \textbf{germ}.
\end{definition*}

\begin{remark*}
$ \FFF_p = \varinjlim_{p \in U} \FFF\br{U} $.
\end{remark*}

\pagebreak

Note that a morphism $ f : \FFF \to \GGG $ of presheaves induces a morphism
$$ \function[f_p]{\FFF_p}{\GGG_p}{\br{U, s}}{\br{U, f_U\br{s}}}. $$

\begin{proposition}
Let $ f : \FFF \to \GGG $ be a morphism of sheaves. Then $ f $ is an isomorphism if and only if $ f_p $ is an isomorphism for all $ p \in X $.
\end{proposition}

\begin{proof}
\hfill
\begin{itemize}[leftmargin=0.5in]
\item[$ \implies $] Obvious.
\item[$ \impliedby $] Assume $ f_p $ is an isomorphism for all $ p \in X $. Need to show that $ f_U : \FFF\br{U} \to \GGG\br{U} $ is an isomorphism for all $ U \subseteq X $, as then we can define $ \br{f^{-1}}_U = \br{f_U}^{-1} $. Check that with this definition, $ \br{f^{-1}}_U $ is compatible with restriction maps, hence $ f^{-1} $ is a morphism of sheaves. \footnote{Exercise}
\begin{itemize}
\item $ f_U $ is injective. Suppose $ s \in \FFF\br{U} $, and $ f_U\br{s} = 0 $. Then for all $ p \in U $, $ f_p\br{U, s} = \br{U, f_U\br{s}} = \br{U, 0} = 0 \in \GGG_p $. Since $ f_p $ is injective, $ \br{U, s} = 0 $ in $ \FFF_p $. That is, there exists a open neighbourhood $ V_p $ of $ p $ in $ U $ such that $ \eval{s}_{V_p} = 0 $. Since $ \cbr{V_p}_{p \in U} $ cover $ U $, we see by S1 that $ s = 0 $.
\item $ f_U $ is surjective. Let $ t \in \GGG\br{U} $ and write $ t_p = \br{U, t} \in \GGG_p $. Since $ f_p $ is surjective, there exists $ s_p \in \FFF_p $ with $ f_p\br{s_p} = t_p $. That is, there exists $ V_p \subseteq U $ an open neighbourhood of $ p $, and a germ $ \br{V_p, s_p} $ such that $ \br{V_p, f_{V_p}\br{s_p}} \equiv \br{U, t} $. By shrinking $ V_p $ if necessary, we can assume that $ \eval{t}_{V_p} = f_{V_p}\br{s_p} $. Now on $ V_p \cap V_q $,
$$ f_{V_p \cap V_q}\br{\eval{s_p}_{V_p \cap V_q} - \eval{s_q}_{V_p \cap V_q}} = \eval{t}_{V_p \cap V_q} - \eval{t}_{V_p \cap V_q} = 0, $$
and hence by injectivity of $ f_{V_p \cap V_q} $ already proved, we have $ \eval{s_p}_{V_p \cap V_q} = \eval{s_q}_{V_p \cap V_q} $. By S2 the $ s_p $'s glue to give an element $ s \in \FFF\br{U} $ with $ \eval{s}_{V_p} = s_p $, for all $ p \in U $. Now
$$ \eval{f_U\br{s}}_{V_p} = f_{V_p}\br{\eval{s}_{V_p}} = f_{V_p}\br{s_p} = \eval{t}_{V_p}. $$
By S1, applied to $ f_U\br{s} - t $, we get $ f_U\br{s} = t $. Thus $ f_U $ is surjective.
\end{itemize}
\end{itemize}
\end{proof}

\subsection{Sheafification}

\begin{theorem}[Sheafification]
Given a presheaf $ \FFF $, there exists a sheaf $ \FFF^+ $ and a morphism $ \theta : \FFF \to \FFF^+ $ satisfying the following universal property. For any sheaf $ \GGG $ and morphism $ \phi : \FFF \to \GGG $, there exists a unique morphism $ \phi^+ : \FFF^+ \to \GGG $ such that $ \phi^+ \circ \theta = \phi $, so
$$
\begin{tikzcd}
\FFF \arrow{r}{\theta} \arrow{dr}[swap]{\phi} & \FFF^+ \arrow[dashed]{d}{\phi^+} \\
& \GGG
\end{tikzcd}.
$$
The pair $ \br{\FFF^+, \theta} $ is unique up to unique isomorphism, and is called the \textbf{sheafification} of $ \FFF $.
\end{theorem}

\begin{proof}
See example sheet $ 1 $. The idea is to make $ \FFF^+ $ look like functions. Define
$$ \FFF^+\br{U} = \cbr{s : U \to \bigsqcup_{p \in U} \FFF_p \st \begin{array}{l} \forall p \in U, \ s\br{p} \in \FFF_p, \\ \forall p \in U, \ \exists p \in V \subseteq U, \ \exists t \in \FFF\br{V}, \ \forall q \in V, \ s\br{q} = \br{V, t} \in \FFF_q \end{array}}. $$
Then
$$ \function[\theta_U]{\FFF\br{U}}{\FFF^+\br{U}}{s}{\br{p \mapsto \br{U, s} \in \FFF_p}}. $$
\end{proof}

\begin{exercise*}
A recommendation is to do all exercises in chapter II.$ 1 $ of Hartshorne.
\end{exercise*}

\pagebreak

\subsection{Kernels, cokernels, and images}

\lecture{4}{Friday}{16/10/20}

\begin{definition*}
Let $ f : \FFF \to \GGG $ be a morphism of presheaves on a space $ X $. We define the following.
\begin{itemize}
\item The \textbf{presheaf kernel} of $ f $, $ \ker f $, is the presheaf given by $ \br{\ker f}\br{U} = \ker \br{f_U : \FFF\br{U} \to \GGG\br{U}} $.
\item The \textbf{presheaf cokernel} $ \coker f $ is the presheaf given by $ \br{\coker f}\br{U} = \coker \br{f_U} = \GGG\br{U} / \im f_U $.
\item The \textbf{presheaf image} $ \im f $ is the presheaf given by $ \br{\im f}\br{U} = \im f_U $.
\end{itemize}
\end{definition*}

\begin{exercise*}
Check that these are presheaves, that is restrictions work.
\end{exercise*}

\begin{remark}
If $ f : \FFF \to \GGG $ is a morphism of sheaves, then $ \ker f $ is also a sheaf.
\end{remark}

\begin{proof}
S1 is certainly satisfied. If $ s \in \br{\ker f}\br{U} \subseteq \FFF\br{U} $ satisfies $ \eval{s}_{U_i} = 0 $ for all $ U_i $ in a cover of $ U $ so $ s = 0 $ by S1 for $ \FFF $. Given $ s_i \in \br{\ker f}\br{U_i} $ with $ \cbr{U_i} $ an open cover of $ U $, and with $ \eval{s_i}_{U_i \cap U_j} = \eval{s_j}_{U_i \cap U_j} $, then there exists $ s \in \FFF\br{U} $ with $ \eval{s}_{U_i} = s_i $ by S2 for $ \FFF $. But $ f_U\br{s} = 0 $ since $ \eval{f_U\br{s}}_{U_i} = f_{U_i}\br{\eval{s}_{U_i}} = f_{U_i}\br{s_i} = 0 $ so by S1, $ f_U\br{s} = 0 $.
\end{proof}

\begin{example*}
Let $ X = \PP^1 $, or think of the Riemann sphere. Let $ P, Q \in X $ be distinct points. Let $ \GGG $ be the sheaf of regular functions on $ X $, or think of the sheaf of holomorphic functions. Let $ \FFF $ be the sheaf of regular functions on $ X $ which vanish at $ P $ and $ Q $. Note $ \FFF\br{U} = \GGG\br{U} $ if $ U \cap \cbr{P, Q} = \emptyset $. Let $ U = \PP^1 \setminus \cbr{P} $ and $ V = \PP^1 \setminus \cbr{Q} $. Note $ \FFF\br{\PP^1} = 0 $ and $ \GGG\br{\PP^1} = k $, because regular functions on $ \PP^1 $ are constants. Let $ f : \FFF \to \GGG $ be the obvious inclusion. Then
$$ \br{\coker f}\br{\PP^1} = k, \qquad \br{\coker f}\br{U} = \GGG\br{U} / \FFF\br{U} = k\sbr{X} / \abr{X} = k, $$
$$ \br{\coker f}\br{V} = k, \qquad \br{\coker f}\br{U \cap V} = \GGG\br{U \cap V} / \FFF\br{U \cap V} = 0. $$
If S2 holds, then we would need to have $ \br{\coker f}\br{\PP^1} = k \oplus k $. This is not a bug, but a feature.
\end{example*}

\begin{definition*}
Let $ f : \FFF \to \GGG $ be a morphism of sheaves.
\begin{itemize}
\item The \textbf{sheaf kernel} $ \ker f $ of $ f $ is just the presheaf kernel.
\item The \textbf{sheaf cokernel} is the sheaf associated to the presheaf cokernel of $ f $.
\item The \textbf{sheaf image} is the sheaf associated to the presheaf image of $ f $.
\end{itemize}
\end{definition*}

$ \FFF $ is a \textbf{subsheaf} of $ \GGG $ if we have inclusions $ \FFF\br{U} \subseteq \GGG\br{U} $ for all $ U $ compatible with restrictions.

\begin{exercise*}
The sheaf image $ \im f $ is a subsheaf of $ \GGG $.
\end{exercise*}

We say $ f $ is \textbf{injective} if $ \ker f = 0 $. We say $ f $ is \textbf{surjective} if $ \im f = \GGG $. We say a sequence of morphisms of sheaves
$$ \dots \to \FFF^{i - 1} \xrightarrow{f^i} \FFF^i \xrightarrow{f^{i + 1}} \FFF^{i + 1} \to \dots $$
is \textbf{exact} if $ \ker f^{i + 1} = \im f^i $ for all $ i $. If $ \FFF' \subseteq \FFF $ is a subsheaf, we write $ \FFF / \FFF' $ for the sheaf associated to the presheaf $ U \mapsto \FFF\br{U} / \FFF'\br{U} $. That is, this is the cokernel of the inclusion $ \FFF' \hookrightarrow \FFF $. A warning is if $ f : \FFF \to \GGG $ is surjective, we do not necessarily have $ \FFF\br{U} \to \GGG\br{U} $ surjective for all $ U $.

\begin{lemma}
Let $ f : \FFF \to \GGG $ be a morphism of sheaves. Then for all $ p \in X $,
$$ \br{\ker f}_p = \ker \br{f_p : \FFF_p \to \GGG_p}, \qquad \br{\im f}_p = \im f_p. $$
\end{lemma}

\begin{proof}
Have a map
$$ \function{\br{\ker f}_p}{\ker f_p \subseteq \FFF_p}{\br{U, s}}{\br{U, s}}. $$
If $ s \in \br{\ker f}\br{U} = \ker f_U $ represents a germ $ \br{U, s} \in \br{\ker f}_p $, then $ \br{U, s} \in \FFF_p $, and $ f_p\br{U, s} = \br{U, f_U\br{s}} = \br{U, 0} = 0 \in \GGG_p $. So $ \br{U, s} \in \ker f_p $.
\begin{itemize}
\item Injective. If $ \br{U, s} = 0 $ in $ \FFF_p $, there exists a neighbourhood $ V \subseteq U $ of $ p $ such that $ \eval{s}_V = 0 $. Then $ \br{U, s} \sim \br{V, \eval{s}_V} = \br{V, 0} = 0 $ in $ \br{\ker f}_p $.
\item Surjective. If $ \br{U, s} \in \ker f_p $, then $ \br{U, f_U\br{s}} = 0 $ in $ \GGG_p $. That is, there exists a neighbourhood $ V \subseteq U $ of $ p $ such that $ 0 = \eval{f_U\br{s}}_V = f_V\br{\eval{s}_V} $. Thus $ \eval{s}_V \in \br{\ker f}\br{V} $, and $ \br{V, \eval{s}_V} \in \br{\ker f}_p $, and $ \br{V, \eval{s}_V} $ maps to the same element in $ \ker f_p $ represented by $ \br{U, s} $.
\end{itemize}

\pagebreak

Let $ \im' f $ be the presheaf image. An easy fact is if $ \FFF $ is a presheaf with associated sheaf $ \FFF^+ $, then $ \FFF_p \cong \FFF_p^+ $ for all $ p \in X $. \footnote{Exercise: check} Thus $ \br{\im f}_p = \br{\im' f}_p $, so need to show $ \br{\im' f}_p \cong \im f_p $. Define a map by
$$ \function{\br{\im' f}_p}{\im f_p}{\br{U, s}}{\br{U, s}}. $$
\begin{itemize}
\item Injective. If $ \br{U, s} = 0 $ in $ \GGG_p $ then there exists a neighbourhood $ V \subseteq U $ of $ p $ such that $ \eval{s}_V = 0 $. Then $ \br{U, s} \sim \br{V, 0} $ in $ \br{\im' f}_p $.
\item Surjective. If $ \br{U, s} \in \im f_p $, then there exists $ \br{V, t} \in \FFF_p $ with $ \br{U, s} = f_p\br{V, t} = \br{V, f_V\br{t}} $, so after shrinking $ U $ and $ V $ if necessary, then we can take $ U = V $ and $ f_U\br{t} = s $. Then $ \br{U, s} \in \br{\im' f}_p $.
\end{itemize}
\end{proof}

\begin{proposition}
Let $ f : \FFF \to \GGG $ be a morphism of sheaves. Then
\begin{enumerate}
\item $ f $ is injective if and only if $ f_p : \FFF_p \to \GGG_p $ is injective for all $ p $, and
\item $ f $ is surjective if and only if $ f_p : \FFF_p \to \GGG_p $ is surjective for all $ p $.
\end{enumerate}
\end{proposition}

\begin{proof}
\hfill
\begin{enumerate}
\item $ f_p $ is injective for all $ p $ if and only if $ \ker f_p = 0 $ for all $ p $, if and only if $ \br{\ker f}_p = 0 $ for all $ p $, if and only if $ \ker f = 0 $, \footnote{Exercise: check by S1} which is if and only if $ f $ is injective.
\item $ f_p $ is surjective for all $ p $ if and only if $ \im f_p = \GGG_p $ for all $ p $, if and only if $ \br{\im f}_p = \GGG_p $ for all $ p $, if and only if $ \im f = \GGG $, \footnote{Exercise: check using $ \im f \subseteq \GGG $} which is if and only if $ f $ is surjective.
\end{enumerate}
\end{proof}

\begin{remark*}
Given $ f : \FFF \to \GGG $, in fact $ \GGG / \im f \cong \coker f $. \footnote{Exercise}
\end{remark*}

\subsection{Passing between spaces}

\lecture{5}{Monday}{19/10/20}

Let $ f : X \to Y $ be a continuous map between topological spaces, $ \FFF $ a sheaf on $ X $, and $ \GGG $ a sheaf on $ Y $. Define $ f_*\FFF $ by, for $ U \subseteq Y $
$$ \br{f_*\FFF}\br{U} = \FFF\br{f^{-1}\br{U}}. $$

\begin{exercise*}
Check $ f_*\FFF $ is a sheaf on $ Y $.
\end{exercise*}

Define $ f^{-1}\GGG $ to be the sheaf associated to the presheaf
$$ U \subseteq X \mapsto \cbr{\br{V, s} \st V \supseteq f\br{U}, \ V \ \text{open}, \ s \in \GGG\br{V}} / \sim, $$
where $ \br{V, s} \sim \br{V', s'} $ if there exists $ W \subseteq V \cap V' $ such that $ f\br{U} \subseteq W $, and $ \eval{s}_W = \eval{s'}_W $.

\begin{example*}
If $ f : \cbr{p} \to X $ is an inclusion of a point, then $ f^{-1}\GGG = \GGG_p $. This is a group but defines a sheaf on a one-point space. More generally, if $ \iota : Z \hookrightarrow X $ is an inclusion of a subset with induced topology, we often write
$$ \eval{\FFF}_Z = \iota^{-1}\FFF. $$
If $ Z $ is open in $ X $, then this is easy, since if $ U \subseteq Z $ then $ \eval{\FFF}_Z\br{U} = \FFF\br{U} $.
\end{example*}

\begin{remark*}
If $ s \in \FFF\br{U} $ we say $ s $ is a \textbf{section} of $ \FFF $ over $ U $. We often write
$$ \FFF\br{U} = \Gamma\br{U, \FFF}, $$
thinking of $ \Gamma\br{U, \cdot} $ as a functor from the category of sheaves on a space $ X $ to the category of abelian groups.
\end{remark*}

\pagebreak

\section{Schemes}

Want to construct a sheaf $ \OOO $ on $ \Spec A $, analogous to the sheaf of regular functions on a variety, and $ \OOO $ will be a sheaf of rings. That is, $ \OOO\br{U} $ will be a ring for each open set $ U $ and restriction maps will be ring homomorphisms.

\subsection{Localisation of a ring}

Importantly recall the following. Let $ A $ be a ring, where all rings are commutative with unity, and $ S \subseteq A $ be a multiplicatively closed subset, that is $ 1 \in S $ and if $ s_1, s_2 \in S $ then $ s_1s_2 \in S $. We define a ring
$$ S^{-1}A = \cbr{\br{a, s} \st a \in A, s \in S} / \sim, $$
where $ \br{a, s} \sim \br{a', s'} $ if there exists $ s'' \in S $ such that $ s''\br{as' - a's} = 0 $. Then $ S^{-1}A $ is called the \textbf{localisation of $ A $ at $ S $}. Note that we write $ a / s $ for the equivalence class of $ \br{a, s} $. The usual equivalence relation on fractions is $ a / s = a' / s' $ if and only if $ as' = a's $. We need the extra possibility of killing $ as' - a's $ with $ s'' $ if $ A $ is not an integral domain.

\begin{example*}
\hfill
\begin{itemize}
\item Take $ f \in A $ and $ S = \cbr{1, f, \dots} \subseteq A $. Then we write $ A_f = S^{-1}A $. These will correspond to open subsets.
\item If $ \ppp \subseteq A $ is a prime ideal and $ S = A \setminus \ppp $, then
\begin{itemize}
\item $ 1 \in S $, and
\item $ a, b \in S $ and $ ab \in \ppp $ is a contradiction by definition of prime ideals, so $ ab \in S $.
\end{itemize}
Then $ A_\ppp = S^{-1}A $ is the \textbf{localisation of $ A $ at $ \ppp $}. These will correspond to stalks.
\end{itemize}
\end{example*}

\subsection{Construction of the structure sheaf}

Let
$$ \OOO\br{U} = \cbr{s : U \to \bigsqcup_{\ppp \in U} A_\ppp \st \begin{array}{l} \forall \ppp \in U, \ s\br{\ppp} \in A_\ppp \\ \forall \ppp \in U, \ \exists \ppp \in V \subseteq U \ \text{open}, \ \exists a, f \in A, \ \forall \qqq \in V, \ f \notin \qqq, \ s\br{\qqq} = \dfrac{a}{f} \in A_\qqq \end{array}}. $$

\begin{proposition}
For any $ \ppp \in \Spec A $, $ \OOO_\ppp = A_\ppp $.
\end{proposition}

\begin{proof}
Have a map
$$ \function{\OOO_\ppp}{A_\ppp}{\br{U, s}}{s\br{\ppp}}. $$
\begin{itemize}
\item Surjective. Any element of $ A_\ppp $ can be written as $ a / f $ for some $ a \in A $ and $ f \notin \ppp $. Then $ \DD\br{f} = \Spec A \setminus \VV\br{f} = \cbr{\ppp \in \Spec A \st f \notin \ppp} $, since $ \VV\br{f} = \cbr{\ppp \in \Spec A \st f \in \ppp} $. Now $ a / f $ defines an element of $ \OOO\br{\DD\br{f}} $ given by
$$ \function[s]{\DD\br{f}}{A_\qqq}{\qqq}{\dfrac{a}{f}}, $$
and in particular, $ s\br{\ppp} = a / f \in A_\ppp $.
\item Injective. Let $ \ppp \in U \subseteq \Spec A $ and $ s \in \OOO\br{U} $ with $ s\br{\ppp} = 0 $ in $ A_\ppp $. Want to show $ \br{U, s} = 0 $ in $ \OOO_\ppp $. By shrinking $ U $ if necessary, we can assume that $ s $ is given by $ a, f \in A $ with $ s\br{\qqq} = a / f $ for all $ \qqq \in U $. In particular $ f \notin \qqq $ for all $ \qqq \in U $. Thus $ a / f = 0 / 1 $ in $ A_\ppp $ so there exists $ h \in A \setminus \ppp $ such that $ 0 = h \cdot \br{a \cdot 1 - f \cdot 0} = h \cdot a $ in $ A $. Now let $ V = U \cap \DD\br{h} $. Then $ \br{V, \eval{s}_V} = 0 $, since for $ \qqq \in V $, $ \eval{s}_V\br{\qqq} = s\br{\qqq} = a / f \in A_\qqq $ and $ h \cdot a = 0 $, and $ h \in A \setminus \qqq $ so $ h \cdot a = 0 $ implies $ a / f = 0 / 1 $ in $ A_\qqq $. Thus $ \br{U, s} = 0 $ in $ \OOO_\ppp $.
\end{itemize}
\end{proof}

\pagebreak

\begin{proposition}
For any $ f \in A $, $ \OOO\br{\DD\br{f}} = A_f $.
\end{proposition}

In particular, as $ \Spec A = \DD\br{1} $, the \textbf{global sections} of $ \OOO $ is $ \OOO\br{\Spec A} = A_1 = A $.

\begin{proof}
Let
$$ \function[\psi]{A_f}{\OOO\br{\DD\br{f}}}{\dfrac{a}{f^n}}{\br{\ppp \in \DD\br{f} \mapsto \dfrac{a}{f^n} \in A_\ppp}}, $$
since $ f \notin \ppp $ implies that $ f^n \notin \ppp $ for all $ n \ge 0 $.
\begin{itemize}
\item Injective. If $ \psi\br{a / f^n} = 0 $, then for all $ \ppp \in \DD\br{f} $, $ a / f^n = 0 $ in $ A_\ppp $, that is there exists $ h \in A \setminus \ppp $ such that $ h \cdot a = 0 $ in $ A $. Let $ I = \cbr{g \in A \st g \cdot a = 0} $, the \textbf{annihilator} of $ a $. So $ h \in I $ and $ h \notin \ppp $, so $ I \not\subseteq \ppp $. This is true for all $ \ppp \in \DD\br{f} $, so $ \VV\br{I} \cap \DD\br{f} = \emptyset $. Thus $ f \in \bigcap_{\ppp \in \VV\br{I}} \ppp = \sqrt{I} $, the radical, so $ f^m \in I $ for some $ m > 0 $. Thus $ f^m \cdot a = 0 $, so $ a / f^n = 0 $ in $ A_f $. Thus $ \psi $ is injective.

\lecture{6}{Wednesday}{21/10/20}

\item Surjective. Let $ s \in \OOO\br{\DD\br{f}} $. Cover $ \DD\br{f} $ with open sets $ V_i $ on which $ s $ is represented as $ a_i / g_i $ with $ a_i, g_i \in A $ such that $ g_i \notin \ppp $ whenever $ \ppp \in V_i $. Thus $ V_i \subseteq \DD\br{g_i} $. By question $ 1 $ on example sheet $ 1 $, the sets of the form $ \DD\br{h} $ form a base for the Zariski topology on $ \Spec A $. Thus we can assume $ V_i = \DD\br{h_i} $ for some $ h_i \in A $. Since $ \DD\br{h_i} \subseteq \DD\br{g_i} $, we have $ \VV\br{h_i} \supseteq \VV\br{g_i} $, so $ \sqrt{\abr{h_i}} \subseteq \sqrt{\abr{g_i}} $, so $ h_i^n \in \abr{g_i} $ for some $ n $, say $ h_i^n = c_ig_i $, so $ a_i / g_i = c_ia_i / h_i^n $. Now replace $ h_i $ by $ h_i^n $, since this does not change open sets because in general $ \DD\br{h_i} = \DD\br{h_i^n} $, and replace $ a_i $ by $ c_ia_i $. The situation so far is that we can assume $ \DD\br{f} $ is covered by sets $ \DD\br{h_i} $ such that $ s $ is represented by $ a_i / h_i $ on $ \DD\br{h_i} $. Claim that $ \DD\br{f} $ can be covered by a finite number of the $ \DD\br{h_i} $, that is $ \DD\br{f} $ is quasi-compact. Since
\begin{align*}
\DD\br{f} \subseteq \bigcup_i \DD\br{h_i} \qquad
& \iff \qquad \VV\br{f} \supseteq \bigcap_i \VV\br{h_i} = \VV\br{\sum_i \abr{h_i}}
\qquad \iff \qquad f \in \bigcap_{\ppp \in \VV\br{\sum_i \abr{h_i}}} \ppp \\
& \iff \qquad f \in \sqrt{\sum_i \abr{h_i}}
\qquad \iff \qquad \exists n, \ f^n \in \sum_i \abr{h_i},
\end{align*}
we can write $ f^n = \sum_{i \in I} b_ih_i $ for some finite index set $ I $. Thus reversing this argument, $ \DD\br{f} \subseteq \bigcup_{i \in I} \DD\br{h_i} $. We now pass to this finite subcover $ \cbr{\DD\br{h_i}} $. On $ \DD\br{h_i} \cap \DD\br{h_j} = \DD\br{h_ih_j} $, note $ a_i / h_i $ and $ a_j / h_j $ both represent $ s $, so by injectivity shown in the last lecture, $ a_ih_j / h_ih_j = a_i / h_i = a_j / h_j = a_jh_i / h_ih_j $ in $ A_{h_ih_j} $. Thus for some $ n $, $ \br{h_ih_j}^n\br{h_ja_i - h_ia_j} = 0 $ in $ A $. We can pick an $ n $ sufficiently large to work for all pairs $ i $ and $ j $. Rewriting, $ h_j^{n + 1}\br{h_i^na_i} - h_i^{n + 1}\br{h_ja_j} = 0 $. Replace each $ h_i $ by $ h_i^{n + 1} $ and $ a_i $ by $ h_i^na_i $, since $ a_i / h_i = a_ih_i^n / h_i^{n + 1} $. Thus we can assume that $ s $ is still represented on $ \DD\br{h_i} $ by $ a_i / h_i $ but also for each $ i $ and $ j $ have $ h_ia_j = h_ja_i $. Note $ f^n = \sum_ib_ih_i $ for $ b_i \in A $, since $ \cbr{\DD\br{h_i}} $ cover $ \DD\br{f} $. Let $ a = \sum_i b_ia_i $. Then for any $ j $, $ h_ja = \sum_i b_ia_ih_j = \sum_i b_ia_jh_i = f^na_j $. Thus $ a / f^n = a_j / h_j $ on $ \DD\br{h_j} $. Thus $ \psi\br{a / f^n} = s $, so $ \psi $ is surjective.
\end{itemize}
\end{proof}

We now have a topological space $ \Spec A $ equipped with a sheaf of rings $ \OOO $.

\subsection{Ringed spaces}

\begin{definition*}
A \textbf{ringed space} is a pair $ \br{X, \OOO_X} $ where
\begin{itemize}
\item $ X $ is a topological space, and
\item $ \OOO_X $ is a sheaf of rings on $ X $.
\end{itemize}
A \textbf{morphism of ringed spaces} $ f : \br{X, \OOO_X} \to \br{Y, \OOO_Y} $ is the following data.
\begin{itemize}
\item $ f : X \to Y $ a continuous map.
\item $ f^\# : \OOO_Y \to f_*\OOO_X $ a morphism of sheaves of rings, that is for each $ U \subseteq Y $ open, we have a ring homomorphism $ f_U^\# : \OOO_Y\br{U} \to \br{f_*\OOO_X}\br{U} = \OOO_X\br{f^{-1}\br{U}} $.
\end{itemize}
\end{definition*}

\pagebreak

\begin{example*}
\hfill
\begin{itemize}
\item Let $ X $ be a topological space, and let $ \OOO_X $ be the sheaf of continuous $ \RR $-valued functions. Then if $ \br{Y, \OOO_Y} $ is similarly defined, given $ f : X \to Y $, we get $ f^\# : \OOO_Y \to f_*\OOO_X $ defined by
$$ \function[f_U^\#]{\OOO_Y\br{U}}{\OOO_X\br{f^{-1}\br{U}}}{\phi}{\phi \circ f}. $$
\item Let $ X $ be a variety, and let $ \OOO_X $ be the sheaf of regular functions on $ X $. A morphism of varieties $ f : X \to Y $ is a continuous map inducing
$$ \function[f_U^\#]{\OOO_Y\br{U}}{\OOO_X\br{f^{-1}\br{U}}}{\phi}{\phi \circ f}. $$
\end{itemize}
\end{example*}

A ring is \textbf{local} if it has a unique maximal ideal.

\begin{definition*}
A \textbf{locally ringed space} $ \br{X, \OOO_X} $ is a ringed space such that $ \OOO_{X, p} $ is a local ring for all $ p \in X $. A \textbf{morphism $ f : \br{X, \OOO_X} \to \br{Y, \OOO_Y} $ of locally ringed spaces} is a morphism of ringed spaces such that the induced homomorphism $ f_p^\# : \OOO_{Y, f\br{p}} \to \OOO_{X, p} $ is a \textbf{local homomorphism} for all $ p \in X $.
\begin{itemize}
\item The map is defined by \footnote{Exercise: check well-defined}
$$ \function[f_p^\#]{\OOO_{Y, f\br{p}}}{\OOO_{X, p}}{\br{U, s}}{\br{f^{-1}\br{U}, f_U^\#\br{s}}}. $$
\item A ring homomorphism $ \phi : \br{A, \mmm_A} \to \br{B, \mmm_B} $ is \textbf{local} if $ \phi^{-1}\br{\mmm_B} = \mmm_A $, where $ \mmm_A $ is the maximal ideal of $ A $. Note that $ \phi\br{A \setminus \mmm_A} = \phi\br{A^*} \subseteq B^* = B \setminus \mmm_B $, where $ A^* $ is the set of invertible elements of $ A $. Thus $ \phi^{-1}\br{\mmm_B} \subseteq \mmm_A $ always.
\end{itemize}
\end{definition*}

\begin{example*}
In the case of varieties, $ \OOO_{X, p} $ has a unique maximal ideal
$$ \cbr{\br{U, f} \in \OOO_X\br{U} \st f\br{p} = 0} / \sim. $$
If $ f\br{p} \ne 0 $, $ f $ is nowhere vanishing on some neighbourhood of $ p $, so after shrinking $ U $, we can invert $ f $. The local homomorphism condition just follows from the pull-back $ \phi \circ f $ of a function $ \phi $ vanishing at $ f\br{p} $ vanishes at $ p $.
\end{example*}

\subsection{Affine schemes}

\lecture{7}{Friday}{23/10/20}

The key example $ \br{\Spec A, \OOO} $ is a locally ringed space, which we call an affine scheme.

\begin{theorem}
The category of affine schemes with locally ringed morphisms is equivalent to the opposite category of rings.
\end{theorem}

Need to show that
\begin{enumerate}
\item if $ \phi : A \to B $ is a ring homomorphism, we obtain an induced morphism $ \br{f, f^\#} : \br{\Spec B, \OOO_{\Spec B}} \to \br{\Spec A, \OOO_{\Spec A}} $, and
\item any morphism of affine schemes as locally ringed spaces arises in this way.
\end{enumerate}

\begin{proof}
\hfill
\begin{enumerate}
\item Given a ring homomorphism $ \phi : A \to B $, define
$$ \function[f]{\Spec B}{\Spec A}{\ppp}{\phi^{-1}\br{\ppp}}. $$
Note $ \phi^{-1}\br{\ppp} $ is prime, since if $ ab \in \phi^{-1}\br{\ppp} $, then $ \phi\br{ab} = \phi\br{a}\phi\br{b} \in \ppp $, thus either $ \phi\br{a} \in \ppp $ or $ \phi\br{b} \in \ppp $, and hence either $ a \in \phi^{-1}\br{\ppp} $ or $ b \in \phi^{-1}\br{\ppp} $. Then $ f $ is continuous, since
\begin{align*}
f^{-1}\br{\VV\br{I}}
& = f^{-1}\br{\cbr{\ppp \in \Spec A \st \ppp \supseteq I}}
= \cbr{\qqq \in \Spec B \st f\br{\qqq} \supseteq I} \\
& = \cbr{\qqq \in \Spec B \st \phi^{-1}\br{\qqq} \supseteq I}
= \cbr{\qqq \in \Spec B \st \qqq \supseteq \phi\br{I}} = \VV\br{\phi\br{I}}.
\end{align*}

\pagebreak

We need to construct $ f^\# : \OOO_{\Spec A} \to f_*\OOO_{\Spec B} $. For $ \ppp \in \Spec B $, we obtain a natural homomorphism
$$ \function[\phi_\ppp]{A_{\phi^{-1}\br{\ppp}}}{B_\ppp}{\dfrac{a}{s}}{\dfrac{\phi\br{a}}{\phi\br{s}}}. $$
Note $ \phi_\ppp $ is a local homomorphism, since the maximal ideal $ \ppp B_\ppp $ of $ B_\ppp $ is generated by the image of $ \ppp $ under the map
$$ \function{B}{B_\ppp}{b}{\dfrac{b}{1}}, $$
and the maximal ideal $ \phi^{-1}\br{\ppp}A_{\phi^{-1}\br{\ppp}} $ of $ A_{\phi^{-1}\br{\ppp}} $ is generated by the image of $ \phi^{-1}\br{\ppp} $ under the map
$$ \function{A}{A_{\phi^{-1}\br{\ppp}}}{a}{\dfrac{a}{1}}, $$
so have a commutative diagram
$$
\begin{tikzcd}
\phi^{-1}\br{\ppp} \arrow[subset]{r} & A \arrow{r}{\phi} \arrow{d} & B \arrow{d} & \ppp \arrow[subset]{l} \\
f\br{\ppp}A_{f\br{\ppp}} \arrow[subset]{r} & A_{\phi^{-1}\br{\ppp}} \arrow{r}[swap]{\phi_\ppp} & B_\ppp & \ppp B_\ppp \arrow[subset]{l}
\end{tikzcd},
$$
thus $ \phi_\ppp^{-1}\br{\ppp B_\ppp} =\phi^{-1}\br{\ppp}A_{\phi^{-1}\br{\ppp}} $. Given $ V \subseteq \Spec A $ open, we may define
$$ \function[f_V^\#]{\OOO_{\Spec A}\br{V}}{\OOO_{\Spec B}\br{f^{-1}\br{V}}}{\br{\ppp \in V \mapsto s\br{\ppp} \in A_\ppp}}{\br{\qqq \in f^{-1}\br{V} \mapsto \phi_\qqq\br{s\br{f\br{\qqq}}} \in B_\qqq}}. $$
Note that we need to check the local coherence part of the definition of $ \OOO $. That is, if $ s $ is locally given by $ a / h $, then $ f_V^\#\br{s} $ is locally given by $ \phi\br{a} / \phi\br{h} $. This gives the desired map $ f^\# : \OOO_{\Spec A} \to f_*\OOO_{\Spec B} $, and the induced map on stalks $ f_\ppp^\# : \OOO_{\Spec A, f\br{\ppp}} \to \OOO_{\Spec B, \ppp} $ agrees with $ \phi_\ppp : A_{\phi^{-1}\br{\ppp}} \to B_\ppp $, by construction. Hence $ \br{f, f^\#} $ is a morphism of locally ringed spaces.
\item Now suppose given a morphism $ \br{f, f^\#} : \Spec B \to \Spec A $ of locally ringed spaces. Take
$$ \phi = f_{\Spec A}^\# : \Gamma\br{\Spec A, \OOO_{\Spec A}} = A \to \Gamma\br{\Spec B, \OOO_{\Spec B}} = B. $$
We need to show $ \phi $ gives rise to $ \br{f, f^\#} $. We have $ f_\ppp^\# : \OOO_{\Spec A, f\br{\ppp}} = A_{f\br{\ppp}} \to \OOO_{\Spec B, \ppp} = B_\ppp $ a local homomorphism. This is compatible with the corresponding map on global sections, that is
$$
\begin{tikzcd}
\Gamma\br{\Spec A, \OOO_{\Spec A}} \arrow{r}{f_{\Spec A}^\#} \arrow{d} & \Gamma\br{\Spec B, \OOO_{\Spec B}} \arrow{d} \\
\OOO_{\Spec A, f\br{\ppp}} \arrow{r}[swap]{f_\ppp^\#} & \OOO_{\Spec B, \ppp}
\end{tikzcd}
$$
is commutative. That is, we have a commutative diagram
$$
\begin{tikzcd}
f\br{\ppp} \arrow[subset]{r} & A \arrow{r}{\phi} \arrow{d} & B \arrow{d} & \ppp \arrow[subset]{l} \\
f\br{\ppp}A_{f\br{\ppp}} \arrow[subset]{r} & A_{f\br{\ppp}} \arrow{r}[swap]{f_\ppp^\#} & B_\ppp & \ppp B_\ppp \arrow[subset]{l}
\end{tikzcd}.
$$
Then $ \br{f_\ppp^\#}^{-1}\br{\ppp B_\ppp} = f\br{\ppp}A_{f\br{\ppp}} $ since $ f_\ppp^\# $ is a local homomorphism, and by commutativity of the diagram, $ f\br{\ppp} = \phi^{-1}\br{\ppp} $. Thus $ f $ is induced by $ \phi $, and $ f_\ppp^\# = \phi_\ppp $. So $ f^\# $ is as constructed previously.
\end{enumerate}
\end{proof}

\begin{remark*}
Demanding $ \br{f, f^\#} $ was a morphism of locally ringed spaces was crucial to make the proof work.
\end{remark*}

\pagebreak

\begin{definition*}
An \textbf{affine scheme} is a locally ringed space isomorphic, in the category of locally ringed spaces, to $ \br{\Spec A, \OOO_{\Spec A}} $ for some ring $ A $. A \textbf{scheme} is a locally ringed space $ \br{X, \OOO_X} $ with an open cover $ \cbr{\br{U_i, \eval{\OOO_X}_{U_i}}} $ with each $ \br{U_i, \eval{\OOO_X}_{U_i}} $ an affine scheme, where $ \eval{\OOO_X}_{U_i}\br{V} = \OOO_X\br{V} $ for $ V \subseteq U_i $ open. A \textbf{morphism of schemes} is a morphism of locally ringed spaces.
\end{definition*}

\begin{example*}
Let $ k $ be a field. Then $ \Spec k = \br{\cbr{0}, k} $.
\begin{itemize}
\item What does giving a morphism $ f : \Spec k \to X $ to a scheme mean? First, this selects a point $ x \in X $, the image of $ f $. Second, we get a local ring homomorphism $ f_x^\# : \OOO_{X, x} \to \OOO_{\Spec k, 0} = k $, that is $ \br{f_x^\#}^{-1}\br{0} = \mmm_x \subseteq \OOO_{X, x} $, the maximal ideal of $ \OOO_{X, x} $. Thus we get a factorisation $ f_x^\# : \OOO_{X, x} \to \OOO_{X, x} / \mmm_x \to k $, where $ \OOO_{X, x} / \mmm_x $ is a field, written as $ \kappa\br{x} $, called the \textbf{residue field} of $ X $ at $ x $. Thus $ f $ induces an inclusion $ \kappa\br{x} \hookrightarrow k $. Conversely, given such an inclusion $ \iota : \kappa\br{x} \hookrightarrow k $ of fields, we get a scheme morphism by defining $ f\br{0} = x $, and
$$ \function[f^\#]{\OOO_X}{f_*k}{s}{\iota\br{s\br{x}}}, \qquad s\br{x} \in \OOO_{X, x}. $$
The moral is that giving a morphism $ f : \Spec k \to X $ is equivalent to giving a point $ x \in X $ and an inclusion $ \iota : \kappa\br{x} \to k $. Note that if $ X = \Spec A $, giving $ \Spec k \to \Spec A $ is equivalent to giving a homomorphism $ A \to k $, which we viewed at the beginning of the course as a $ k $-valued point on $ \Spec A $.

\lecture{8}{Monday}{26/10/20}

\item What does giving $ X \to \Spec k $ mean? No information in the continuous map, but need also a map $ f^\# : k \to f_*\OOO_X $, that is a map $ k \to \Gamma\br{\Spec k, f_*\OOO_X} = \Gamma\br{X, \OOO_X} $. That is, $ \Gamma\br{X, \OOO_X} $ carries a $ k $-algebra structure. Note this induces $ k $-algebra structures on $ \OOO_X\br{U} $ for all $ U $ via the composition $ k \to \OOO_X\br{X} \to \OOO_X\br{U} $ and similarly all stalks $ \OOO_{X, p} $ are also $ k $-algebras. We say $ X $ is a \textbf{scheme defined over $ k $}. For example, in affine varieties, consider $ A = k\sbr{X_1, \dots, X_n} / I $ with $ I = \sqrt{I} $. Then $ \Spec A $ is our replacement for $ \VV\br{I} \subseteq \AA_k^n $, viewing $ \Spec A $ as a scheme over $ k $. If $ k \subseteq k' $ is a field extension, a \textbf{$ k' $-valued point} of $ X / k $ is a commutative diagram
$$
\begin{tikzcd}
\Spec k' \arrow{rr} \arrow{dr} & & X \arrow{dl} \\
& \Spec k &
\end{tikzcd}.
$$
We write $ X\br{k'} $ for the set of such morphisms.
\end{itemize}
\end{example*}

\begin{remark*}
It is rare in algebraic geometry to work with schemes alone, but rather always working over a base scheme.
\end{remark*}

Fix a base scheme $ S $. Define $ \Sch / S $ to be the category whose objects are morphisms $ T \to S $ and morphisms are commutative diagrams
$$
\begin{tikzcd}
T \arrow{rr} \arrow{dr} & & T' \arrow{dl} \\
& S &
\end{tikzcd}.
$$
We will frequently work with $ \Sch / k = \Sch / \Spec k $. Given $ T \to S $ and $ X \to S $ objects in $ \Sch / S $, a \textbf{$ T $-valued point} of $ X \to S $ is a morphism $ T \to X $ over $ S $, so
$$
\begin{tikzcd}
T \arrow{rr} \arrow{dr} & & X \arrow{dl} \\
& S &
\end{tikzcd},
$$
and we write $ X\br{T} $ for the set of $ T $-valued points. The \textbf{Yoneda philosophy} is that $ X\br{T} $ for all $ T $ determines $ X $.

\pagebreak

\begin{example*}
Fix a field $ k $, and let $ D = \Spec k\sbr{t} / \abr{t^2} = \br{\cbr{\abr{t}}, k\sbr{t} / \abr{t^2}} $. Then $ t $ does not make sense as $ k $-valued function anymore, as $ t^2 = 0 $. Let $ X $ be any scheme over $ k $. What is $ X\br{D} $? Given $ f : D \to X $ a morphism of schemes over $ k $, we get a point $ x \in X $ as the image of $ f $ and a local homomorphism
$$ \function[f_x^\#]{\OOO_{X, x}}{k\sbr{t} / \abr{t^2}}{\mmm_x}{\abr{t}}. $$
Note that $ \mmm_x^2 $ maps to zero, hence we get a $ k $-linear map $ \mmm_x / \mmm_x^2 \to \abr{t} \cong k $ as a $ k $-vector space. We also have a composed surjective $ k $-algebra homomorphism $ \OOO_{X, x} \to k\sbr{t} / \abr{t} \cong k $ with kernel $ \mmm_x $, and hence we have $ \kappa\br{x} = \OOO_{X, x} / \mmm_x \cong k $. So we get
\begin{itemize}
\item a $ k $-valued point $ x $ with residue field $ k $, and
\item a $ k $-vector space map $ \mmm_x / \mmm_x^2 \to k $, that is an element of $ \br{\mmm_x / \mmm_x^2}^* $, the dual vector space.
\end{itemize}
Then $ \br{\mmm_x / \mmm_x^2}^* $ is called the \textbf{Zariski tangent space} to $ X $ at $ x $. Think of $ D $ as a point plus an arrow.
\end{example*}

\begin{example*}
\textbf{Glued schemes} are a special case of a question on example sheet $ 1 $. Suppose given two schemes $ X_1 $ and $ X_2 $ and open subsets $ U_i \subseteq X_i $. Recall $ U_i $ is also a locally ringed space $ \br{U_i, \eval{\OOO_{X_i}}_{U_i}} $, and in fact $ U_i $ is then a scheme. Given an isomorphism $ f : U_1 \xrightarrow{\sim} U_2 $, can glue $ X_1 $ and $ X_2 $ along $ U_1 $ and $ U_2 $ to get a scheme $ X $ with an open cover $ \cbr{X_1, X_2} $, so $ X = X_1 \sqcup X_2 / \sim $ such that $ x_1 \in U_1 \sim x_2 \in U_2 $ if $ f\br{x_1} = x_2 $, and need to define $ \OOO_X $. Now take $ \AA_k^n = \Spec k\sbr{X_1, \dots, X_n} $, so $ \AA_k^1 = \Spec k\sbr{X} $. Take $ X_1 = X_2 = \AA_k^1 $.
\begin{itemize}
\item Glue $ U_1 = \AA^1 \setminus \cbr{0} = \DD\br{X} \subseteq X_1 $ and $ U_2 = \AA^1 \setminus \cbr{0} = \DD\br{X} \subseteq X_2 $ via the identity map. This is the affine line with doubled origin.
\item Could instead glue $ U_1 $ and $ U_2 $ via the map given by $ X \mapsto X^{-1} $, where $ U_1 = \Spec k\sbr{X}_X = U_2 $ and
$$ \function{k\sbr{X}_X}{k\sbr{X}_X}{X}{X^{-1}} $$
induces an isomorphism $ U_1 \to U_2 $. When we glue, we get the projective line over $ k $, $ \PP_k^1 $.
\end{itemize}
\end{example*}

\subsection{Projective schemes}

Let $ S $ be a graded ring, that is
$$ S = \bigoplus_{d \ge 0} S_d, $$
with $ S_d $ an abelian group, and product law satisfies $ S_d \cdot S_{d'} \subseteq S_{d + d'} $.

\begin{example*}
$ S = k\sbr{X_0, \dots, X_n} $, and $ S_d $ is the space of polynomials which are homogeneous of degree $ d $, that is spanned by monomials of degree $ d $.
\end{example*}

We write
$$ S_+ = \bigoplus_{d \ge 1} S_d, $$
which we call the \textbf{irrelevant ideal}.

\begin{definition*}
$ I \subseteq S $ is a \textbf{homogeneous ideal} if $ I $ is generated by its homogeneous elements, that is elements in $ S_d $ for various $ d $.
\end{definition*}

\begin{definition*}
Let
$$ \Proj S = \cbr{\ppp \in \Spec S \st \ppp \ \text{is homogeneous}, \ \ppp \not\supseteq S_+}. $$
For $ I \subseteq S $ a homogeneous ideal, set
$$ \VV\br{I} = \cbr{\ppp \in \Proj S \st \ppp \supseteq I}. $$
\end{definition*}

\begin{exercise*}
Check the $ \VV\br{I} $ form the closed sets of a topology on $ \Proj S $.
\end{exercise*}

\pagebreak

\lecture{9}{Wednesday}{28/10/20}

\begin{notation*}
For $ \ppp \in \Proj S $, let
$$ T = \cbr{f \in S \setminus \ppp \st f \ \text{is homogeneous}}. $$
Then $ T $ is a multiplicatively closed subset of $ S $, and let $ S_{\br{\ppp}} \subseteq T^{-1}S $ be the subring of elements of degree zero, that is written in the form $ s / s' $ with $ s \in S $ homogeneous and $ s' \in T $ with $ \deg s = \deg s' $. For $ f \in S $ homogeneous, we write $ S_{\br{f}} \subseteq S_f $ for the subset of elements of degree zero.
\end{notation*}

Can now define a sheaf $ \OOO $ on $ \Proj S $. For $ U \subseteq \Proj S $ open, set
$$ \OOO\br{U} = \cbr{s : U \to \bigsqcup_{\ppp \in U} S_{\br{\ppp}} \st \begin{array}{l} \forall \ppp \in U, \ s\br{\ppp} \in S_{\br{\ppp}} \\ \forall \ppp \in U, \ \exists \ppp \in V \subseteq U \ \text{open}, \ \exists a, f \in S, \ \forall \qqq \in V, \ f \notin \qqq, \ s\br{\qqq} = \dfrac{a}{f} \in S_{\br{\qqq}} \end{array}}, $$
where $ a $ and $ f $ are homogeneous of the same degree. As before, $ \OOO_\ppp = S_{\br{\ppp}} $. \footnote{Exercise: check} Is the locally ringed space $ \br{\Proj S, \OOO} $ a scheme?

\begin{notation*}
If $ f \in S $ is homogeneous, then we write
$$ \DD_+\br{f} = \cbr{\ppp \in \Proj S \st f \notin \ppp}, $$
which is an open set and $ \DD_+\br{f} = \Proj S \setminus \VV\br{f} $.
\end{notation*}

\begin{proposition}
$ \br{\DD_+\br{f}, \eval{\OOO}_{\DD_+\br{f}}} \cong \Spec S_{\br{f}} $ as locally ringed spaces. Further, the open sets $ \DD_+\br{f} $ for $ f \in S_+ $ cover $ \Proj S $. Hence $ \br{\Proj S, \OOO} $ is a scheme.
\end{proposition}

\begin{proof}
Will be on example sheet $ 2 $.
\end{proof}

\begin{definition*}
If $ A $ is a ring, define
$$ \PP_A^n = \Proj A\sbr{X_0, \dots, X_n}. $$
\end{definition*}

\begin{example*}
If $ k $ is an algebraically closed field, consider $ \PP_k^1 = \Proj k\sbr{X_0, X_1} $. The closed points, that is points $ \ppp $ such that $ \cbr{\ppp} $ is closed, correspond to maximal elements of $ \Proj S $. \footnote{Exercise: check} These maximal elements are ideals of the form $ \abr{aX_0 - bX_1} $. The only maximal homogeneous ideal of $ k\sbr{X_0, X_1} $ is $ \abr{X_0, X_1} = S_+ $, since any maximal ideal is of the form $ \abr{X_0 - a_0, X_1 - a_1} $. The other prime ideals of $ k\sbr{X_0, X_1} $ are principal, that is of the form $ \abr{f} $ with $ f $ irreducible or $ f = 0 $. For $ \abr{f} $ to be homogeneous, $ f $ must be homogeneous. Any such polynomial splits into linear factors, all homogeneous, so in order for $ f $ to be irreducible it must be linear. Note we have a one-to-one correspondence between
$$ \function{\cbr{\abr{aX_0 - bX_1} \st a, b \in K, \ a, b \ \text{not both zero}}}{\br{k^2 \setminus \cbr{\br{0, 0}}} / k^*}{\abr{aX_0 - bX_1}}{\br{b : a}}, $$
where $ k^* $ acts by $ \br{a, b} \mapsto \br{\lambda a, \lambda b} $ for $ \lambda \in k^* $. The conclusion is that the closed points of $ \PP_k^1 $ are in one-to-one correspondence with points of $ \br{k^2 \setminus \cbr{\br{0, 0}}} / k^* $. More generally, the closed points of $ \PP_k^n $ are in one-to-one correspondence with points of $ \br{k^{n + 1} \setminus \cbr{0}} / k^* $. Can see this by making use of the open cover $ \cbr{\DD_+\br{X_i} \st 0 \le i \le n} $, \footnote{Exercise: good exercise} which is an open cover since $ \ppp \notin \DD_+\br{X_i} $ for any $ i $ implies that $ X_i \in \ppp $ for all $ i $, so $ S_+ \subseteq \ppp $ and so $ \ppp \notin \Proj S $.
\end{example*}

\begin{example*}
Let $ S = k\sbr{X_0, \dots, X_n} $, but grade by $ \deg X_i = w_i $, where $ w_0, \dots, w_n $ are positive integers. Define $ \W\PP^n\br{w_0, \dots, w_n} = \Proj S $, the \textbf{weighted projective space}. For example, $ \W\PP^2\br{1, 1, 2} $ has an open cover $ \cbr{\DD_+\br{X_i} \st 0 \le i \le 2} $. Consider $ \DD_+\br{X_2} = \Spec S_{\br{X_2}} $. Note
$$ S_{\br{X_2}} = k\sbr{\dfrac{X_0^2}{X_2}, \dfrac{X_0X_1}{X_2}, \dfrac{X_1^2}{X_2}} \cong k\sbr{U, V, W} / \abr{UW - V^2} \subseteq S_{X_2}, $$
so $ \Spec S_{\br{X_2}} $ is a quadric cone with a singular point. Similarly, $ \DD_+\br{X_0} $ and $ \DD_+\br{X_1} $ are both isomorphic to $ \AA_k^2 $.
\end{example*}

\pagebreak

\begin{example*}
Let $ M = \ZZ^n $ and $ M_\RR = M \otimes_\ZZ \RR = \RR^n $. Let $ \Delta \subseteq M_\RR $ be a compact convex lattice polytope. That is, there exists a finite set $ V \subseteq M $ such that $ \Delta $ is the convex hull of $ V $, that is the smallest convex set containing $ V $. Let
$$ \C\br{\Delta} = \cbr{\br{m, r} \in M_\RR \oplus \RR \st m \in r\Delta, \ r \ge 0} \subseteq M_\RR \oplus \RR. $$
Here $ r\Delta = \cbr{rm \st m \in \Delta} $. This is the \textbf{cone over $ \Delta $}. Let
$$ S = k\sbr{\C\br{\Delta} \cap \br{M \oplus \ZZ}} = \bigoplus_{P \in \C\br{\Delta} \cap \br{M \oplus \ZZ}} kZ^P, $$
with multiplication given by $ Z^PZ^{P'} = Z^{P + P'} $, since $ \C\br{\Delta} \cap \br{M \oplus \ZZ} $ is a monoid, that is it is closed under addition and contains zero. This makes $ S $ into a ring, and it is graded by $ \deg Z^{\br{m, r}} = r $. Define $ \PP_\Delta = \Proj S $. This is called a \textbf{projective toric variety}.
\begin{itemize}
\item Let $ \Delta $ be the convex hull of $ \cbr{0, e_1, \dots, e_n} $ with $ e_1, \dots, e_n $ the standard basis of $ M = \ZZ^n $. Check that $ S = k\sbr{X_0, \dots, X_n} $ with standard grading $ X_0 = Z^{\br{0, 1}} $ and $ X_i = Z^{\br{e_i, 1}} $. \footnote{Exercise} So $ \PP_\Delta = \PP_k^n $.
\item Let $ n = 2 $, and let $ \Delta $ be the convex hull of $ \cbr{\br{0, 0}, \br{1, 0}, \br{0, 1}, \br{1, 1}} $. In $ S $, the degree $ d $ monomials are $ \cbr{Z^{\br{a, b, d}} \st 0 \le a \le d, \ 0 \le b \le d} $. Any of these can be written as a product of monomials of degree one, that is the monomials $ X = Z^{\br{0, 0, 1}} $, $ Y = Z^{\br{1, 0, 1}} $, $ W = Z^{\br{0, 1, 1}} $, and $ T = Z^{\br{1, 1, 1}} $. Thus $ S = k\sbr{X, Y, W, T} / \abr{XT - YW} $. So $ \Proj S $ can be thought of as a quadric surface in $ \PP_k^3 $.
\end{itemize}
\end{example*}

\subsection{Open and closed subschemes}

\lecture{10}{Friday}{30/10/20}

\begin{definition*}
An \textbf{open subscheme} of a scheme $ X $ is a scheme $ \br{U, \eval{\OOO_X}_U} $ for $ U \subseteq X $ an open subset. Note that this is a scheme because from question $ 1 $ and question $ 11 $ on the first example sheet, open affine subsets of $ X $ form a basis for the topology on $ X $. An \textbf{open immersion} is a morphism $ f : X \to Y $ which induces a isomorphism of $ X $ with an open subscheme of $ Y $. A \textbf{closed immersion} $ f : X \to Y $ is a morphism which is a homeomorphism onto a closed subset of $ Y $, and the induced morphism $ f^\# : \OOO_Y \to f_*\OOO_X $ is surjective. A \textbf{closed subscheme} of $ Y $ is an equivalence class of closed immersions, where
$$
\begin{tikzcd}
X \arrow[dashed]{rr}{i} \arrow{dr} & & X' \arrow{dl} \\
& Y &
\end{tikzcd}
$$
are equivalent if there exists an isomorphism $ i $ making the diagram commute.
\end{definition*}

\begin{example*}
\hfill
\begin{itemize}
\item Let $ Y = \Spec A $, let $ I \subseteq A $ be an ideal, and let $ X = \Spec A / I $. Note the map of schemes induced by the quotient map $ A \to A / I $ identifies $ \Spec A / I $ with $ \VV\br{I} \subseteq \Spec A $. Thus $ f : X \to Y $, induced by $ A \to A / I $, satisfies the first condition of being a closed immersion. Note that $ \OOO_Y \to f_*\OOO_X $ is surjective on stalks. For $ \ppp \in \VV\br{I} $, $ \OOO_{Y, \ppp} = A_\ppp $ and $ \br{f_*\OOO_X}_\ppp = \OOO_{X, \ppp} $ since all open sets in $ X $ are of the form $ U \cap X $ for $ U $ an open set of $ Y $ and $ \OOO_{X, \ppp} = \br{A / I}_{\ppp / I} $. Certainly $ A_\ppp \to \br{A / I}_{\ppp / I} $ is surjective.
\item Let $ \Spec k\sbr{X, Y} / \abr{X} \to \Spec k\sbr{X, Y} = \AA^2 $. This gives a closed subscheme structure to the set $ \VV\br{X} $. Note $ \VV\br{X^2, XY} = \VV\br{X} $. This gives a closed immersion $ \Spec k\sbr{X, Y} / \abr{X^2, XY} \to \AA^2 $. This gives a different closed subscheme structure on $ \VV\br{X} $. Note these two subschemes are isomorphic away from the origin, which we can see by looking at $ \DD\br{Y} \subseteq \Spec k\sbr{X, Y} / \abr{X} $, where
$$ \DD\br{Y} \cong \Spec \br{k\sbr{X, Y} / \abr{X}}_Y = \Spec k\sbr{Y}_Y. $$
Looking at $ \DD\br{Y} \subseteq \Spec k\sbr{X, Y} / \abr{X^2, XY} $,
$$ \DD\br{Y} \cong \Spec \br{k\sbr{X, Y} / \abr{X^2, XY}}_Y \cong \Spec \br{k\sbr{X, Y}_Y / \abr{X}} \cong \Spec k\sbr{Y}_Y. $$
\end{itemize}
\end{example*}

\pagebreak

\subsection{Fibre products}

Let $ \CCC $ be a category and
$$
\begin{tikzcd}
& Y \arrow{d}{g} \\
X \arrow{r}[swap]{f} & Z
\end{tikzcd}
$$
be a diagram in $ \CCC $. Then the \textbf{fibre product}, if it exists, is an object $ W $ equipped with morphisms $ p : W \to X $ and $ q : W \to Y $ such that $ f \circ p = g \circ q $ satisfying the following universal property. For any $ W' $ equipped with maps $ p' : W' \to X $ and $ q' : W' \to Y $ such that $ f \circ p' = g \circ q' $, there exists a unique morphism $ h : W' \to W $ making the diagram
$$
\begin{tikzcd}
W' \arrow[dashed]{dr}{\exists !h} \arrow[bend left=30]{drr}{q'} \arrow[bend right=30]{ddr}[swap]{p'} & & \\
& W \arrow{r}{q} \arrow{d}[swap]{p} & Y \arrow{d}{g} \\
& X \arrow{r}[swap]{f} & Z
\end{tikzcd}
$$
commute, that is $ p \circ h = p' $ and $ q \circ h = q' $. Note that if the fibre product exists, it is unique up to unique isomorphism.

\begin{example*}
Let $ \CCC $ be the category of sets. Then
$$ X \times_Z Y = \cbr{\br{x, y} \in X \times Y \st f\br{x} = g\br{y}}. $$
\end{example*}

It will be helpful to think about the fibre product, and more generally other universal properties, via the Yoneda lemma.

\begin{definition*}
Let $ \CCC $ be a category. Write $ \h_X $ for the contravariant functor
$$ \functions[\h_X]{\CCC}{\Set}{Y}{\Hom\br{Y, X}}{f : Y \to Z}{\br{\phi \in \Hom\br{Z, X} \mapsto \phi \circ f \in \Hom\br{Y, X}}}. $$
\end{definition*}

Recall that a \textbf{natural transformation} between contravariant functors $ F, G : \CCC \to \DDD $, written as $ T : \CCC \to \DDD $, consists of the data $ T\br{X} : F\br{X} \to G\br{X} $ for all $ X \in \Ob \CCC $ such that for all $ f : X \to Y $ in $ \CCC $
$$
\begin{tikzcd}
F\br{X} \arrow{d}[swap]{T\br{X}} & F\br{Y} \arrow{l}[swap]{F\br{f}} \arrow{d}{T\br{Y}} \\
G\br{X} & G\br{Y} \arrow{l}{G\br{f}}
\end{tikzcd}
$$
is commutative.

\begin{lemma}[Yoneda's lemma]
The set of natural transformations between $ \h_X : \CCC \to \Set $ and $ G : \CCC \to \Set $ is $ G\br{X} $.
\end{lemma}

\begin{proof}
Given $ \eta \in G\br{X} $, we need to define a map
$$ \function{\h_X\br{Y} = \Hom\br{Y, X}}{G\br{Y}}{f}{G\br{f}\br{\eta}}, $$
for all objects $ Y \in \CCC $. Check that this defines a natural transformation $ \h_X \to G $. \footnote{Exercise} Conversely, given $ T : \h_X \to G $ a natural transformation, take $ \eta = T\br{X}\br{\id_X} $. Check that these two maps are inverse to each other. \footnote{Exercise}
\end{proof}

\begin{corollary}
The set of natural transformations $ \h_X \to \h_Y $ is $ \h_Y\br{X} = \Hom\br{X, Y} $.
\end{corollary}

\pagebreak

\begin{definition*}
A contravariant functor $ F : \CCC \to \Set $ is said to be \textbf{representable} if $ F \cong \h_X $ for some $ X \in \Ob \CCC $.
\end{definition*}

Lots of questions in algebraic geometry are about representability of functors. Redefining, the fibre product in a category $ \CCC $ is an object which represents the functor
$$ T \mapsto \Hom\br{T, X} \times_{\Hom\br{T, Z}} \Hom\br{T, Y}, $$
since an element of the set $ \Hom\br{T, X} \times_{\Hom\br{T, Z}} \Hom\br{T, Y} $ is a commutative diagram
$$
\begin{tikzcd}
T \arrow[dashed]{dr} \arrow[bend left=30]{drr}{q} \arrow[bend right=30]{ddr}[swap]{p} & & \\
& W \arrow[dashed]{r} \arrow[dashed]{d} & Y \arrow{d}{g} \\
& X \arrow{r}[swap]{f} & Z
\end{tikzcd}.
$$
The advantage of using Yoneda is that we can check identities using fibre products using identities of the products of sets.

\begin{example*}
In $ \Set $,
$$ \bijection{\br{A \times_B B} \times_C D}{A \times_B D}{\br{\br{a, c}, d}}{\br{a, d}}{\br{\br{a, f\br{d}}, d}}{\br{a, d}}, \qquad f : D \to C. $$
Then we have two functors
$$
\begin{tikzcd}
T \arrow{r} \arrow{dr} & \h_A\br{T} \times_{\h_B\br{T}} \h_C\br{T} \times_{\h_C\br{T}} \h_D\br{T} \arrow{d}{\sim} \\
& \h_A\br{T} \times_{\h_B\br{T}} \h_D\br{T}
\end{tikzcd},
$$
and natural transformations showing those functors are isomorphic, and hence represent isomorphic objects.
\end{example*}

\end{document}