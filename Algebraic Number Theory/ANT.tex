\def\module{Algebraic Number Theory}
\def\lecturer{Professor Anthony Scholl}
\def\term{Lent 2020}
\def\cover{}
\def\syllabus{}
\def\thm{section}

\documentclass{article}

% Packages

\usepackage{amssymb}
\usepackage{amsthm}
\usepackage[UKenglish]{babel}
\usepackage{commath}
\usepackage{enumitem}
\usepackage{etoolbox}
\usepackage{fancyhdr}
\usepackage[margin=1in]{geometry}
\usepackage{graphicx}
\usepackage[hidelinks]{hyperref}
\usepackage[utf8]{inputenc}
\usepackage{listings}
\usepackage{mathdots}
\usepackage{mathtools}
\usepackage{stmaryrd}
\usepackage{tikz-cd}
\usepackage{csquotes}

% Formatting

\addto\captionsUKenglish{\renewcommand{\abstractname}{Syllabus}}
\delimitershortfall5pt
\ifx\thm\undefined\newtheorem{n}{}\else\newtheorem{n}{}[\thm]\fi
\newcommand\newoperator[1]{\ifcsdef{#1}{\cslet{#1}{\relax}}{}\csdef{#1}{\operatorname{#1}}}
\newcommand\vC{\check{\C}}
\newcommand\vH{\check{\H}}
\setlength{\parindent}{0cm}

% Environments

\theoremstyle{plain}
\newtheorem{algorithm}[n]{Algorithm}
\newtheorem*{algorithm*}{Algorithm}
\newtheorem{algorithm**}{Algorithm}
\newtheorem{conjecture}[n]{Conjecture}
\newtheorem*{conjecture*}{Conjecture}
\newtheorem{conjecture**}{Conjecture}
\newtheorem{corollary}[n]{Corollary}
\newtheorem*{corollary*}{Corollary}
\newtheorem{corollary**}{Corollary}
\newtheorem{lemma}[n]{Lemma}
\newtheorem*{lemma*}{Lemma}
\newtheorem{lemma**}{Lemma}
\newtheorem{proposition}[n]{Proposition}
\newtheorem*{proposition*}{Proposition}
\newtheorem{proposition**}{Proposition}
\newtheorem{theorem}[n]{Theorem}
\newtheorem*{theorem*}{Theorem}
\newtheorem{theorem**}{Theorem}

\theoremstyle{definition}
\newtheorem{aim}[n]{Aim}
\newtheorem*{aim*}{Aim}
\newtheorem{aim**}{Aim}
\newtheorem{axiom}[n]{Axiom}
\newtheorem*{axiom*}{Axiom}
\newtheorem{axiom**}{Axiom}
\newtheorem{condition}[n]{Condition}
\newtheorem*{condition*}{Condition}
\newtheorem{condition**}{Condition}
\newtheorem{definition}[n]{Definition}
\newtheorem*{definition*}{Definition}
\newtheorem{definition**}{Definition}
\newtheorem{example}[n]{Example}
\newtheorem*{example*}{Example}
\newtheorem{example**}{Example}
\newtheorem{exercise}[n]{Exercise}
\newtheorem*{exercise*}{Exercise}
\newtheorem{exercise**}{Exercise}
\newtheorem{fact}[n]{Fact}
\newtheorem*{fact*}{Fact}
\newtheorem{fact**}{Fact}
\newtheorem{goal}[n]{Goal}
\newtheorem*{goal*}{Goal}
\newtheorem{goal**}{Goal}
\newtheorem{law}[n]{Law}
\newtheorem*{law*}{Law}
\newtheorem{law**}{Law}
\newtheorem{plan}[n]{Plan}
\newtheorem*{plan*}{Plan}
\newtheorem{plan**}{Plan}
\newtheorem{problem}[n]{Problem}
\newtheorem*{problem*}{Problem}
\newtheorem{problem**}{Problem}
\newtheorem{question}[n]{Question}
\newtheorem*{question*}{Question}
\newtheorem{question**}{Question}
\newtheorem{warning}[n]{Warning}
\newtheorem*{warning*}{Warning}
\newtheorem{warning**}{Warning}
\newtheorem{acknowledgements}[n]{Acknowledgements}
\newtheorem*{acknowledgements*}{Acknowledgements}
\newtheorem{acknowledgements**}{Acknowledgements}
\newtheorem{annotations}[n]{Annotations}
\newtheorem*{annotations*}{Annotations}
\newtheorem{annotations**}{Annotations}
\newtheorem{assumption}[n]{Assumption}
\newtheorem*{assumption*}{Assumption}
\newtheorem{assumption**}{Assumption}
\newtheorem{conclusion}[n]{Conclusion}
\newtheorem*{conclusion*}{Conclusion}
\newtheorem{conclusion**}{Conclusion}
\newtheorem{claim}[n]{Claim}
\newtheorem*{claim*}{Claim}
\newtheorem{claim**}{Claim}
\newtheorem{notation}[n]{Notation}
\newtheorem*{notation*}{Notation}
\newtheorem{notation**}{Notation}
\newtheorem{note}[n]{Note}
\newtheorem*{note*}{Note}
\newtheorem{note**}{Note}
\newtheorem{remark}[n]{Remark}
\newtheorem*{remark*}{Remark}
\newtheorem{remark**}{Remark}

% Lectures

\newcommand{\lecture}[3]{ % Lecture
  \marginpar{
    Lecture #1 \\
    #2 \\
    #3
  }
}

% Blackboard

\renewcommand{\AA}{\mathbb{A}} % Blackboard A
\newcommand{\BB}{\mathbb{B}}   % Blackboard B
\newcommand{\CC}{\mathbb{C}}   % Blackboard C
\newcommand{\DD}{\mathbb{D}}   % Blackboard D
\newcommand{\EE}{\mathbb{E}}   % Blackboard E
\newcommand{\FF}{\mathbb{F}}   % Blackboard F
\newcommand{\GG}{\mathbb{G}}   % Blackboard G
\newcommand{\HH}{\mathbb{H}}   % Blackboard H
\newcommand{\II}{\mathbb{I}}   % Blackboard I
\newcommand{\JJ}{\mathbb{J}}   % Blackboard J
\newcommand{\KK}{\mathbb{K}}   % Blackboard K
\newcommand{\LL}{\mathbb{L}}   % Blackboard L
\newcommand{\MM}{\mathbb{M}}   % Blackboard M
\newcommand{\NN}{\mathbb{N}}   % Blackboard N
\newcommand{\OO}{\mathbb{O}}   % Blackboard O
\newcommand{\PP}{\mathbb{P}}   % Blackboard P
\newcommand{\QQ}{\mathbb{Q}}   % Blackboard Q
\newcommand{\RR}{\mathbb{R}}   % Blackboard R
\renewcommand{\SS}{\mathbb{S}} % Blackboard S
\newcommand{\TT}{\mathbb{T}}   % Blackboard T
\newcommand{\UU}{\mathbb{U}}   % Blackboard U
\newcommand{\VV}{\mathbb{V}}   % Blackboard V
\newcommand{\WW}{\mathbb{W}}   % Blackboard W
\newcommand{\XX}{\mathbb{X}}   % Blackboard X
\newcommand{\YY}{\mathbb{Y}}   % Blackboard Y
\newcommand{\ZZ}{\mathbb{Z}}   % Blackboard Z

% Brackets

\renewcommand{\eval}[1]{\left. #1 \right|}                     % Evaluation
\newcommand{\br}{\del}                                         % Brackets
\newcommand{\abr}[1]{\left\langle #1 \right\rangle}            % Angle brackets
\newcommand{\fbr}[1]{\left\lfloor #1 \right\rfloor}            % Floor brackets
\newcommand{\st}{\ \middle| \ }                                % Such that
\newcommand{\intd}[4]{\int_{#1}^{#2} \, #3 \, \dif #4}         % Single integral
\newcommand{\iintd}[4]{\iint_{#1} \, #2 \, \dif #3 \, \dif #4} % Double integral

% Calligraphic

\newcommand{\AAA}{\mathcal{A}} % Calligraphic A
\newcommand{\BBB}{\mathcal{B}} % Calligraphic B
\newcommand{\CCC}{\mathcal{C}} % Calligraphic C
\newcommand{\DDD}{\mathcal{D}} % Calligraphic D
\newcommand{\EEE}{\mathcal{E}} % Calligraphic E
\newcommand{\FFF}{\mathcal{F}} % Calligraphic F
\newcommand{\GGG}{\mathcal{G}} % Calligraphic G
\newcommand{\HHH}{\mathcal{H}} % Calligraphic H
\newcommand{\III}{\mathcal{I}} % Calligraphic I
\newcommand{\JJJ}{\mathcal{J}} % Calligraphic J
\newcommand{\KKK}{\mathcal{K}} % Calligraphic K
\newcommand{\LLL}{\mathcal{L}} % Calligraphic L
\newcommand{\MMM}{\mathcal{M}} % Calligraphic M
\newcommand{\NNN}{\mathcal{N}} % Calligraphic N
\newcommand{\OOO}{\mathcal{O}} % Calligraphic O
\newcommand{\PPP}{\mathcal{P}} % Calligraphic P
\newcommand{\QQQ}{\mathcal{Q}} % Calligraphic Q
\newcommand{\RRR}{\mathcal{R}} % Calligraphic R
\newcommand{\SSS}{\mathcal{S}} % Calligraphic S
\newcommand{\TTT}{\mathcal{T}} % Calligraphic T
\newcommand{\UUU}{\mathcal{U}} % Calligraphic U
\newcommand{\VVV}{\mathcal{V}} % Calligraphic V
\newcommand{\WWW}{\mathcal{W}} % Calligraphic W
\newcommand{\XXX}{\mathcal{X}} % Calligraphic X
\newcommand{\YYY}{\mathcal{Y}} % Calligraphic Y
\newcommand{\ZZZ}{\mathcal{Z}} % Calligraphic Z

% Fraktur

\newcommand{\aaa}{\mathfrak{a}}   % Fraktur a
\newcommand{\bbb}{\mathfrak{b}}   % Fraktur b
\newcommand{\ccc}{\mathfrak{c}}   % Fraktur c
\newcommand{\ddd}{\mathfrak{d}}   % Fraktur d
\newcommand{\eee}{\mathfrak{e}}   % Fraktur e
\newcommand{\fff}{\mathfrak{f}}   % Fraktur f
\renewcommand{\ggg}{\mathfrak{g}} % Fraktur g
\newcommand{\hhh}{\mathfrak{h}}   % Fraktur h
\newcommand{\iii}{\mathfrak{i}}   % Fraktur i
\newcommand{\jjj}{\mathfrak{j}}   % Fraktur j
\newcommand{\kkk}{\mathfrak{k}}   % Fraktur k
\renewcommand{\lll}{\mathfrak{l}} % Fraktur l
\newcommand{\mmm}{\mathfrak{m}}   % Fraktur m
\newcommand{\nnn}{\mathfrak{n}}   % Fraktur n
\newcommand{\ooo}{\mathfrak{o}}   % Fraktur o
\newcommand{\ppp}{\mathfrak{p}}   % Fraktur p
\newcommand{\qqq}{\mathfrak{q}}   % Fraktur q
\newcommand{\rrr}{\mathfrak{r}}   % Fraktur r
\newcommand{\sss}{\mathfrak{s}}   % Fraktur s
\newcommand{\ttt}{\mathfrak{t}}   % Fraktur t
\newcommand{\uuu}{\mathfrak{u}}   % Fraktur u
\newcommand{\vvv}{\mathfrak{v}}   % Fraktur v
\newcommand{\www}{\mathfrak{w}}   % Fraktur w
\newcommand{\xxx}{\mathfrak{x}}   % Fraktur x
\newcommand{\yyy}{\mathfrak{y}}   % Fraktur y
\newcommand{\zzz}{\mathfrak{z}}   % Fraktur z

% Maps

\newcommand{\bijection}[7][]{    % Bijection
  \ifx &#1&
    \begin{array}{rcl}
      #2 & \longleftrightarrow & #3 \\
      #4 & \longmapsto         & #5 \\
      #6 & \longmapsfrom       & #7
    \end{array}
  \else
    \begin{array}{ccrcl}
      #1 & : & #2 & \longrightarrow & #3 \\
         &   & #4 & \longmapsto     & #5 \\
         &   & #6 & \longmapsfrom   & #7
    \end{array}
  \fi
}
\newcommand{\correspondence}[2]{ % Correspondence
  \cbr{
    \begin{array}{c}
      #1
    \end{array}
  }
  \qquad
  \leftrightsquigarrow
  \qquad
  \cbr{
    \begin{array}{c}
      #2
    \end{array}
  }
}
\newcommand{\function}[5][]{     % Function
  \ifx &#1&
    \begin{array}{rcl}
      #2 & \longrightarrow & #3 \\
      #4 & \longmapsto     & #5
    \end{array}
  \else
    \begin{array}{ccrcl}
      #1 & : & #2 & \longrightarrow & #3 \\
         &   & #4 & \longmapsto     & #5
    \end{array}
  \fi
}
\newcommand{\functions}[7][]{    % Functions
  \ifx &#1&
    \begin{array}{rcl}
      #2 & \longrightarrow & #3 \\
      #4 & \longmapsto     & #5 \\
      #6 & \longmapsto     & #7
    \end{array}
  \else
    \begin{array}{ccrcl}
      #1 & : & #2 & \longrightarrow & #3 \\
         &   & #4 & \longmapsto     & #5 \\
         &   & #6 & \longmapsto     & #7
    \end{array}
  \fi
}

% Matrices

\newcommand{\onebytwo}[2]{      % One by two matrix
  \begin{pmatrix}
    #1 & #2
  \end{pmatrix}
}
\newcommand{\onebythree}[3]{    % One by three matrix
  \begin{pmatrix}
    #1 & #2 & #3
  \end{pmatrix}
}
\newcommand{\twobyone}[2]{      % Two by one matrix
  \begin{pmatrix}
    #1 \\
    #2
  \end{pmatrix}
}
\newcommand{\twobytwo}[4]{      % Two by two matrix
  \begin{pmatrix}
    #1 & #2 \\
    #3 & #4
  \end{pmatrix}
}
\newcommand{\threebyone}[3]{    % Three by one matrix
  \begin{pmatrix}
    #1 \\
    #2 \\
    #3
  \end{pmatrix}
}
\newcommand{\threebythree}[9]{  % Three by three matrix
  \begin{pmatrix}
    #1 & #2 & #3 \\
    #4 & #5 & #6 \\
    #7 & #8 & #9
  \end{pmatrix}
}

% Operators

\newoperator{Aut}   % Automorphism
\newoperator{cell}  % Cell
\newoperator{ch}    % Characteristic
\newoperator{cl}    % Cup length
\newoperator{codim} % Codimension
\newoperator{coker} % Cokernel
\newoperator{Cone}  % Cone
\newoperator{Crit}  % Critical
\newoperator{div}   % Divisor
\newoperator{Div}   % Divisor group
\newoperator{End}   % Endomorphism
\newoperator{Ext}   % Ext
\newoperator{Frac}  % Fraction
\newoperator{Frob}  % Frobenius
\newoperator{Gal}   % Galois
\newoperator{GL}    % General linear
\newoperator{Gr}    % Grassmannian
\newoperator{Ht}    % Height
\newoperator{Hom}   % Homomorphism
\newoperator{id}    % Identity
\newoperator{im}    % Image
\newoperator{Int}   % Integral
\newoperator{ker}   % Kernel
\newoperator{Mat}   % Matrix
\newoperator{Morse} % Morse
\newoperator{MV}    % Mayer-Vietoris
\newoperator{nr}    % Unramified
\newoperator{ns}    % Nonsingular
\newoperator{Ob}    % Object
\newoperator{ord}   % Order
\newoperator{Pic}   % Picard
\newoperator{Proj}  % Projective
\newoperator{res}   % Restriction
\newoperator{rk}    % Rank
\newoperator{Spec}  % Spectrum
\newoperator{supp}  % Support
\newoperator{taut}  % Tautological
\newoperator{tors}  % Torsion
\newoperator{Tr}    % Trace
\newoperator{triv}  % Trivial

% Roman

\newcommand{\A}{\mathrm{A}}   % Roman A
\newcommand{\B}{\mathrm{B}}   % Roman B
\newcommand{\C}{\mathrm{C}}   % Roman C
\newcommand{\D}{\mathrm{D}}   % Roman D
\newcommand{\E}{\mathrm{E}}   % Roman E
\newcommand{\F}{\mathrm{F}}   % Roman F
\newcommand{\G}{\mathrm{G}}   % Roman G
\renewcommand{\H}{\mathrm{H}} % Roman H
\newcommand{\I}{\mathrm{I}}   % Roman I
\newcommand{\J}{\mathrm{J}}   % Roman J
\newcommand{\K}{\mathrm{K}}   % Roman K
\renewcommand{\L}{\mathrm{L}} % Roman L
\newcommand{\M}{\mathrm{M}}   % Roman M
\newcommand{\N}{\mathrm{N}}   % Roman N
\renewcommand{\O}{\mathrm{O}} % Roman O
\renewcommand{\P}{\mathrm{P}} % Roman P
\newcommand{\Q}{\mathrm{Q}}   % Roman Q
\newcommand{\R}{\mathrm{R}}   % Roman R
\renewcommand{\S}{\mathrm{S}} % Roman S
\newcommand{\T}{\mathrm{T}}   % Roman T
\newcommand{\U}{\mathrm{U}}   % Roman U
\newcommand{\V}{\mathrm{V}}   % Roman V
\newcommand{\W}{\mathrm{W}}   % Roman W
\newcommand{\X}{\mathrm{X}}   % Roman X
\newcommand{\Y}{\mathrm{Y}}   % Roman Y
\newcommand{\Z}{\mathrm{Z}}   % Roman Z

\renewcommand{\a}{\mathrm{a}} % Roman a
\renewcommand{\b}{\mathrm{b}} % Roman b
\renewcommand{\c}{\mathrm{c}} % Roman c
\renewcommand{\d}{\mathrm{d}} % Roman d
\newcommand{\e}{\mathrm{e}}   % Roman e
\newcommand{\f}{\mathrm{f}}   % Roman f
\newcommand{\g}{\mathrm{g}}   % Roman g
\newcommand{\h}{\mathrm{h}}   % Roman h
\renewcommand{\i}{\mathrm{i}} % Roman i
\renewcommand{\j}{\mathrm{j}} % Roman j
\renewcommand{\k}{\mathrm{k}} % Roman k
\renewcommand{\l}{\mathrm{l}} % Roman l
\newcommand{\m}{\mathrm{m}}   % Roman m
\renewcommand{\n}{\mathrm{n}} % Roman n
\renewcommand{\o}{\mathrm{o}} % Roman o
\newcommand{\p}{\mathrm{p}}   % Roman p
\newcommand{\q}{\mathrm{q}}   % Roman q
\renewcommand{\r}{\mathrm{r}} % Roman r
\newcommand{\s}{\mathrm{s}}   % Roman s
\renewcommand{\t}{\mathrm{t}} % Roman t
\renewcommand{\u}{\mathrm{u}} % Roman u
\renewcommand{\v}{\mathrm{v}} % Roman v
\newcommand{\w}{\mathrm{w}}   % Roman w
\newcommand{\x}{\mathrm{x}}   % Roman x
\newcommand{\y}{\mathrm{y}}   % Roman y
\newcommand{\z}{\mathrm{z}}   % Roman z

% Tikz

\tikzset{
  arrow symbol/.style={"#1" description, allow upside down, auto=false, draw=none, sloped},
  subset/.style={arrow symbol={\subset}},
  cong/.style={arrow symbol={\cong}}
}

% Fancy header

\pagestyle{fancy}
\lhead{\module}
\rhead{\nouppercase{\leftmark}}

% Make title

\title{\module}
\author{Lectured by \lecturer \\ Typed by David Kurniadi Angdinata}
\date{\term}

% Macros
\newcommand{\mods}{\mod\!\!^*\ }

\begin{document}

% Title page
\maketitle
\cover
\vfill
\begin{abstract}
\noindent\syllabus
\end{abstract}

\pagebreak

% Contents page
\tableofcontents

\pagebreak

% Document page
\setcounter{section}{-1}

\setcounter{section}{0}

\section{Absolute values and places}

\subsection{Absolute values}

\lecture{1}{Thursday}{21/01/21}

Let $ K $ be a field. Recall that an \textbf{absolute value (AV)} on $ K $ is a function $ \abs{\cdot} : K \to \RR_{\ge 0} $ such that for all $ x, y \in K $,
\begin{enumerate}
\item $ \abs{x} = 0 $ if and only if $ x = 0 $,
\item $ \abs{xy} = \abs{x} \cdot \abs{y} $, and
\item $ \abs{x + y} \le \abs{x} + \abs{y} $.
\end{enumerate}
Also assume
\begin{itemize}
\item[$ 4 $.] there exists $ x \in K $ such that $ \abs{x} \ne 0, 1 $.
\end{itemize}
This excludes the trivial AV
$$ \abs{x} =
\begin{cases}
0 & x = 0 \\
1 & x \ne 0
\end{cases}.
$$
An AV is a \textbf{non-archimedean} if
\begin{itemize}
\item[$ 3^{\text{NA}} $.] $ \abs{x + y} \le \max\br{\abs{x}, \abs{y}} $,
\end{itemize}
and \textbf{archimedean} otherwise. An AV determines a metric $ \d\br{x, y} = \abs{x - y} $ which makes $ K $ a \textbf{topological field}, so $ + $, $ \times $, and $ \br{\cdot}^{-1} $ are continuous.

\begin{remark*}
It is convenient to weaken $ 3 $ to
\begin{itemize}
\item[$ 3' $.] there exists $ \alpha > 0 $ such that for all $ x $ and $ y $, $ \abs{x + y}^\alpha \le \abs{x}^\alpha + \abs{y}^\alpha $.
\end{itemize}
For non-archimedean AV, makes no difference. Does mean that if $ \abs{\cdot} $ is an AV, then so is $ \abs{\cdot}^\alpha $ for any $ \alpha > 0 $. The point is that we want the function $ z \mapsto z\overline{z} $ on $ \CC $ to be an AV. Explain why later.
\end{remark*}

Let us suppose $ \abs{\cdot} $ is a non-archimedean AV. Then
$$ R = \cbr{x \in K \st \abs{x} \le 1} $$
is a subring of $ K $. It is a \textbf{local ring} with maximal ideal
$$ \mmm_R = \cbr{\abs{x} < 1}. $$
It is a \textbf{valuation ring} of $ K $, so if $ x \in K \setminus R $ then $ x^{-1} \in R $.

\begin{lemma}
\label{lem:1.1}
$ R $ is a maximal subring of $ K $.
\end{lemma}

\begin{proof}
Let $ x \in K \setminus R $. Then $ \abs{x} > 1 $. Then if $ y \in R $, there exists $ n \ge 0 $ such that $ \abs{yx^{-n}} = \abs{y} / \abs{x}^n \le 1 $, that is $ y \in x^nR $ for $ n \gg 0 $. So $ R\sbr{x} = K $, hence $ R $ is maximal.
\end{proof}

\begin{remark*}
There is a general notion of valuation, not necessarily $ \RR $-valued, seen in algebraic geometry. The valuations we are considering here are rank one valuations, and they have this maximality property.
\end{remark*}

AVs $ \abs{\cdot} $ and $ \abs{\cdot}' $ are \textbf{equivalent} if there exists $ \alpha > 0 $ such that $ \abs{\cdot}' = \abs{\cdot}^\alpha $.

\begin{proposition}
\label{prop:1.2}
The following are equivalent.
\begin{itemize}
\item $ \abs{\cdot} $ and $ \abs{\cdot}' $ are equivalent.
\item for all $ x, y \in K $, $ \abs{x} \le \abs{y} $ if and only if $ \abs{x}' \le \abs{y}' $.
\item for all $ x, y \in K $, $ \abs{x} < \abs{y} $ if and only if $ \abs{x}' < \abs{y}' $.
\end{itemize}
\end{proposition}

\begin{proof}
See local fields.
\end{proof}

A corollary is if $ \abs{\cdot} $ and $ \abs{\cdot}' $ are non-archimedean AVs with valuation rings $ R $ and $ R' $, then $ \abs{\cdot} $ and $ \abs{\cdot}' $ are equivalent if and only if $ R = R' $, if and only if $ R \subset R' $, by \ref{lem:1.1}.

\pagebreak

Equivalent AVs define equivalent metrics on $ K $, hence the completion of $ K $ with respect to $ \abs{\cdot} $ depends only on the equivalence class of $ \abs{\cdot} $. Inequivalent AVs determine independent topologies, in the following sense.

\begin{proposition}[Weak approximation]
Let $ \abs{\cdot}_i $ for $ 1 \le i \le n $ be pairwise inequivalent AVs on $ K $, let $ a_1, \dots, a_n \in K $, and let $ \delta > 0 $. Then there exists $ x \in K $ such that for all $ i $, $ \abs{x - a_i}_i < \delta $.
\end{proposition}

\begin{proof}
Suppose $ z_j \in K $ such that $ \abs{z_j}_j > 1 $ and $ \abs{z_j}_i < 1 $ for all $ i \ne j $. Then $ \abs{z_j^N / \br{z_j^N + 1}}_i \to 0 $ as $ N \to \infty $ if $ i \ne j $ but $ \abs{z_j^N / \br{z_j^N + 1} - 1}_j = \abs{1 / \br{z_j^N + 1}}_j \to 0 $. So
$$ x = \sum_j a_j\dfrac{z_j^N}{z_j^N + 1} $$
works if $ N $ is sufficiently large. So it is enough to find $ z_j $, and by symmetry take $ j = 1 $. Induction on $ n $.
\begin{itemize}[leftmargin=0.5in]
\item[$ n = 1 $.] Trivial.
\item[$ n > 1 $.] Suppose have $ y $ with $ \abs{y}_1 > 1 $ and $ \abs{y}_2, \dots, \abs{y}_{n - 1} < 1 $. If $ \abs{y}_n < 1 $, finished. Otherwise, pick $ w \in K $ with $ \abs{w}_1 > 1 > \abs{w}_n $, such as by \ref{prop:1.2}. If $ \abs{y}_n = 1 $, then $ z = y^Nw $ works, for $ N $ sufficiently large. If $ \abs{y}_n > 1 $, then $ z = y^Nw / \br{y^N + 1} $ works, for $ N $ sufficiently large.
\end{itemize}
\end{proof}

\begin{remark*}
If $ K = \QQ $ and $ \abs{\cdot}_1, \dots, \abs{\cdot}_n $ are $ p_i $-adic AVs for distinct primes $ p_i $, and $ a_i \in \ZZ $, then weak approximation says that for all $ n_i \ge 1 $, there exists $ x \in \QQ $, which is a $ p_i $-adic integer for all $ i \in \cbr{1, \dots, n} $ and $ x \equiv a_i \mod p_i^{n_i} $. This of course follows from CRT, which guarantees there exists $ x \in \ZZ $ satisfying this.
\end{remark*}

\subsection{Places}

\begin{definition*}
A \textbf{place} of $ K $ is an equivalence class of AVs on $ K $.
\end{definition*}

\begin{example*}
If $ K = \QQ $, by Ostrowski's theorem, every AV on $ \QQ $ is equivalent to one of
\begin{itemize}
\item a $ p $-adic AV $ \abs{\cdot}_p $ for $ p $ prime, or
\item a Euclidean AV $ \abs{\cdot}_\infty $.
\end{itemize}
So places of $ \QQ $ are in bijection with $ \cbr{\text{primes}} \cup \cbr{\infty} $. We will usually simply denote the places of $ \QQ $ by $ \cbr{2, 3, \dots, \infty} = \cbr{p \le \infty} $.
\end{example*}

\begin{notation*}
Let
\begin{itemize}
\item $ \V_K $ be the places of $ K $,
\item $ \V_{K, \infty} $ be the places given by archimedean AVs, the \textbf{infinite places}, and
\item $ \V_{K, \f} $ be the places given by non-archimedean AVs, the \textbf{finite places}.
\end{itemize}
Often use letters $ v $ and $ w $, decorated suitably, to denote places. If $ v \in \V_K $, then $ K_v $ will denote the completion. If $ v : K^\times \to \RR $ is a valuation, will also use $ v $ to denote the corresponding place, that is the class of AVs $ x \mapsto r^{-v\br{x}} $ for $ r > 1 $.
\end{notation*}

Can restate weak approximation in terms of places.

\begin{proposition}
Let $ v_1, \dots, v_n $ be distinct places of $ K $. Then the image of the diagonal inclusion
$$ K \hookrightarrow \prod_{1 \le i \le n} K_{v_i} $$
is dense, for the product topology.
\end{proposition}

\pagebreak

\subsection{Extensions of places}

Let $ L / K $ be finite separable, and let $ v $ and $ w $ be places of $ K $ and $ L $ respectively. Say $ w $ \textbf{lies over}, or \textbf{divides}, $ v $, denoted $ w \mid v $, if $ v = \eval{w}_K $ is the restriction of $ w $ to $ K $. Then there exists a unique continuous $ K_v \hookrightarrow L_w $ extending $ K \hookrightarrow L $.

\begin{proposition}
There is a unique isomorphism of topological rings mapping
$$ \function{L \otimes_K K_v}{\prod_{w \in \V_L, \ w \mid v} L_w}{x \otimes y}{\br{xy}_w}. $$
\end{proposition}

In the local fields course, proved this for finite places of number fields.

\begin{proof}
Let $ L = K\br{a} $, and let $ f \in K\sbr{T} $ be the minimal polynomial, which is separable. Factor $ f = \prod_i g_i $ for $ g_i \in K_v\sbr{T} $ irreducible and distinct. Let $ L_i = K_v\sbr{T} / \abr{g_i} $. Then $ L \otimes_K K_v = K_v\sbr{T} / \abr{f} \xrightarrow{\sim} \prod_i L_i $ by CRT. Let $ w \mid v $, inducing $ \iota_w : L \hookrightarrow L_w $. Let $ g_w \in K_v\sbr{T} $ be the minimal polynomial of $ \iota_w\br{a} $ over $ K_v $. Then $ g_w \mid f $ so $ g_w \in \cbr{g_i} $ and $ L_w = K_v\br{\iota_w\br{a}} $ is some $ L_i $. Conversely, $ K_v $ is complete and $ L_i / K_v $ is finite, so there exists a unique extension of $ v $ to $ L_i $, so there is a bijection $ \cbr{g_i} \leftrightarrow \cbr{w \mid v} $, and thus
$$ L \otimes_K K_v \cong \prod_w L_w. $$
Use that both sides are finite-dimensional normed $ K_v $-spaces. For the left hand side, choose a basis of $ L / K $ for $ L \otimes_K K_v \cong K_v^{\sbr{L : K}} $ with norm $ \norm{\br{x_i}} = \sup_i \abs{x_i}_v $, where $ \abs{\cdot}_v $ is an AV in class of $ v $ satisfying triangle inequality. For the right hand side, $ \norm{\br{y_w}} = \sup_w \abs{y_w}_w $, where $ \abs{\cdot}_w $ is the AV in class of $ w $ extending $ \abs{\cdot}_v $. A fact is that any two norms on a finite-dimensional vector space over a field complete with respect to an AV are equivalent. For local fields, exactly the same proof as for $ \RR $, and in general not much harder. See Cassels and Fr\"ohlich chapter II, section $ 8 $.
\end{proof}

\begin{corollary}
\label{cor:1.6}
\hfill
\begin{itemize}
\item $ \cbr{w \mid v} $ is finite, non-empty, and
$$ \sum_{w \mid v} \sbr{L_w : K_v} = \sbr{L : K}. $$
\item For all $ x \in L $,
$$ \N_{L / K}\br{x} = \prod_{w \mid v} \N_{L_w / K_v}\br{x}, \qquad \Tr_{L / K}\br{x} = \sum_{w \mid v} \Tr_{L_w / K_v}\br{x}. $$
\end{itemize}
\end{corollary}

\lecture{2}{Saturday}{23/01/21}

Let $ L / K $ be a finite Galois extension with $ G = \Gal\br{L / K} $. Then $ G $ acts on places $ w $ of $ L $ lying over a given place $ v $ of $ K $. If $ \abs{\cdot} $ is an AV on $ L $, then for all $ g \in G $, the map $ x \mapsto \abs{g^{-1}\br{x}} $ is an AV on $ L $, agreeing with $ \abs{\cdot} $ on $ K $. So this defines a left action of $ G $ on $ \cbr{w \mid v} $ by $ g\br{w} = w \circ g^{-1} $. If $ w = \v_\ppp $ for a prime $ \ppp $ in a Dedekind domain, then $ g\br{w} = \v_{g\br{\ppp}} $.

\begin{definition*}
Define the \textbf{decomposition group} $ \D_w $ or $ G_w $ to be the stabiliser of $ w $ in $ G $.
\end{definition*}

If $ g \in G_w $, then it is continuous for the topology induced by $ w $ on $ L $, so extends to an automorphism of $ L_w $, the completion. Then $ G_w \hookrightarrow \Aut\br{L_w / K_v} $, by continuity, so $ \#G_w \le \sbr{L_w : K_v} $, and
$$ \#G = \br{G : G_w}\#G_w \le \br{G : G_w}\sbr{L_w : K_v} = \sum_{g \in G / G_w} \sbr{L_{g\br{w}} : K_v} \le \sum_{w' \mid v} \sbr{L_{w'} : K_v} = \sbr{L : K} = \#G, $$
by \ref{cor:1.6}. So have equality, hence $ \sbr{L_w : K_v} = \#G_w $, and so $ L_w / K_v $ is Galois with group $ \Gal\br{L_w / K_v} \xrightarrow{\sim} G_w \subset G $, and $ G $ acts transitively on places over $ v $.

\begin{notation*}
Suppose $ v $ is discrete valuation of $ L $, so a finite place, and the valuation ring is a DVR. Then so is any $ w \mid v $, and define $ \f\br{w \mid v} = \f_{L_w / K_v} $ to be the degree of residue class extension and $ \e\br{w \mid v} $ to be the ramification degree, and
$$ \sbr{L_w : K_v} = \e\br{w \mid v}\f\br{w \mid v}. $$
\end{notation*}

\pagebreak

\section{Number fields}

\begin{remark*}
A lot of theory applies to other global fields, that is \textbf{function fields} $ K / \FF_p\br{t} $ that are finite extensions. These are less interesting, at least to number theorists, since there are no infinite places.
\end{remark*}

\subsection{Dedekind domains}

Let $ K $ be a \textbf{number field}, a finite extension of $ \QQ $, with \textbf{ring of integers} $ \OOO_K $, the integral closure of $ \ZZ $ in $ K $. A basic property is that $ \OOO_K $ is a Dedekind domain, that is
\begin{enumerate}
\item Noetherian, in fact, by finiteness of integral closure, $ \OOO_K $ is a finitely generated $ \ZZ $-module,
\item integrally closed in $ K $, by definition, and
\item every non-zero prime ideal is maximal, so Krull dimension at most one.
\end{enumerate}
The following are basic results about Dedekind domains.

\begin{theorem}
\label{thm:2.1}
\hfill
\begin{enumerate}
\item A local domain is Dedekind if and only if it is a DVR.
\item For a domain $ R $, the following are equivalent.
\begin{enumerate}
\item $ R $ is Dedekind.
\item $ R $ is Noetherian and for all non-zero prime $ \ppp \subset R $, $ R_\ppp $ is a DVR.
\item Every fractional ideal of $ R $ is invertible.
\end{enumerate}
\item A Dedekind domain with only finitely many prime ideals, so \textbf{semi-local}, is a PID.
\end{enumerate}
\end{theorem}

A \textbf{fractional ideal} of $ R $ is a non-zero $ R $-submodule $ I \subset K $ such that for some $ 0 \ne x \in R $, $ xI \subset R $ is an ideal, and $ I $ is \textbf{invertible} if there exists a fractional ideal $ I^{-1} $ such that $ II^{-1} = R $.

\begin{proof}
\hfill
\begin{enumerate}
\item A DVR is a local PID. Proved in local fields. The forward direction is the hardest part.
\item Let $ K = \Frac R $.
\begin{itemize}[leftmargin=0.5in]
\item[$ \br{a} \implies \br{b} $.] Enough to check \footnote{Exercise} that properties $ 1 $ to $ 3 $ are preserved under localisation, then use part $ 1 $.
\item[$ \br{b} \implies \br{c} $.] To prove $ \br{c} $, may assume $ I \subset R $ is an ideal. Let
$$ I^{-1} = \cbr{x \in K \st xI \subset R}. $$
If $ 0 \ne y \in I $, then $ R \subset I^{-1} \subset y^{-1}R $, so $ I^{-1} $ is a fractional ideal and $ I^{-1}I \subset R $. Let $ \ppp \subset R $ be prime, so $ R_\ppp $ is a DVR. It suffices to prove $ I^{-1}I \not\subset \ppp $. Let $ I = \abr{a_1, \dots, a_n} $ for $ a_i \in R $. Without loss of generality, $ \v_\ppp\br{a_1} \le \v_\ppp\br{a_i} $ for all $ i $. Then $ IR_\ppp = a_1R_\ppp $, so for all $ i $, $ a_i / a_1 = x_i / y_i \in R_\ppp $ for $ x_i \in R $ and $ y_i \in R \setminus \ppp $. Then $ y = \prod_i y_i \notin \ppp $ as $ \ppp $ is prime, and $ ya_i / a_1 \in R $ for all $ i $, so $ y / a_1 \in I^{-1} $. Thus $ y \in II^{-1} \setminus \ppp $.
\item[$ \br{c} \implies \br{a} $.] Check the following.
\begin{itemize}
\item $ R $ is Noetherian. Let $ I \subset R $ be an ideal. Then $ II^{-1} = R $, so $ 1 = \sum_{i = 1}^n a_ib_i $ for $ a_i \in I $ and $ b_i \in I^{-1} $. Let $ I' = \abr{a_1, \dots, a_n} \subset I $. Then $ I'I^{-1} = R = II^{-1} $, so $ I' = I $. So $ I $ is finitely generated.
\item $ R $ is integrally closed. Let $ x \in K $, integral over $ R $. Then $ I = R\sbr{x} = \sum_{0 \le i < d} Rx^i \subset K $, where $ d $ is the degree of the polynomial of integral independence, is a fractional ideal. Obviously $ I^2 = I $, so $ I = I^2I^{-1} = II^{-1} = R $, that is $ x \in R $.
\item Every non-zero prime is maximal. Let $ \cbr{0} \ne \qqq \subset \ppp \subsetneq R $ for $ \ppp $ and $ \qqq $ prime. Then $ R \subsetneq \ppp^{-1} \subset \qqq^{-1} $, so $ \qqq \subsetneq \ppp^{-1}\qqq \subset R $, and $ \ppp\br{\ppp^{-1}\qqq} = \qqq $, so as $ \qqq $ is prime and $ \ppp^{-1}\qqq \not\subset \qqq $, so $ \ppp \subset \qqq $, that is $ \ppp = \qqq $.
\end{itemize}
\end{itemize}

\pagebreak

\item Let $ R $ be semi-local Dedekind with non-zero primes $ \ppp_1, \dots, \ppp_n $. Choose $ x \in R $ with $ x \in \ppp_1 \setminus \ppp_1^2 $ and $ x \notin \ppp_2, \dots, \ppp_n $. Then $ \ppp_1 = \abr{x} $, and every ideal is a product of powers of $ \cbr{\ppp_i} $, by below, so $ R $ is a PID.
\end{enumerate}
\end{proof}

\begin{theorem}
Let $ R $ be Dedekind. Then
\begin{enumerate}
\item the group of fractional ideals is freely generated by the non-zero prime ideals, and
$$ I = \prod_\ppp \ppp^{\v_\ppp\br{I}}, \qquad \v_\ppp\br{I} = \inf \cbr{\v_\ppp\br{x} \st x \in I}, $$
\item if $ \br{R : I} < \infty $ for all $ I \ne \cbr{0} $, then for all $ I $ and $ J $,
$$ \br{R : IJ} = \br{R : I}\br{R : J}. $$
\end{enumerate}
\end{theorem}

\begin{proof}
\hfill
\begin{enumerate}
\item If $ I \ne R $, then $ I \subset \ppp $ for some prime ideal $ \ppp $. Then $ I = \ppp I' $ where $ I' = I\ppp^{-1} \supsetneq I $ then by Noetherian induction, using the ascending chain condition on ideals, $ I $ is a product of powers of prime ideals, $ I = \prod_\ppp \ppp^{a_\ppp} $. Then get the same for fractional ideals $ J = x^{-1}I $. Consider the homomorphisms
$$ \function{\cbr{\text{fractional ideals of} \ R}}{\cbr{\text{fractional ideals of} \ R_\ppp}}{I}{IR_\ppp}, \qquad \function{\cbr{\text{fractional ideals of} \ R_\ppp}}{\ZZ}{\abr{\pi^n}}{n}. $$
The composition is $ I \mapsto \v_\ppp\br{I} $, and if $ \qqq \ne \ppp $ then $ \v_\ppp\br{\qqq} = 0 $. So
$$ \function[\br{\v_\ppp}_\ppp]{\cbr{\text{fractional ideals of} \ R}}{\bigoplus_\ppp \ZZ}{\prod_\ppp \ppp^{a_\ppp}}{\br{a_\ppp}_\ppp}. $$
So $ a_\ppp $ are unique and $ \br{\v_\ppp}_\ppp $ is an isomorphism.

\lecture{3}{Tuesday}{26/01/21}

\item By unique factorisation of ideals in $ 1 $,
$$ \prod_\ppp \ppp^{a_\ppp} \cap \prod_\ppp \ppp^{b_\ppp} = \prod_\ppp \ppp^{\max\br{a_\ppp, b_\ppp}}, $$
so if $ I + J = R $, then $ IJ = I \cap J $, so by CRT, $ R / IJ \cong R / I \times R / J $ so the result holds if $ I + J = R $. So reduced to showing that $ \br{R : \ppp^{n + 1}} = \br{R : \ppp}\br{R : \ppp^n} $. Now $ R / \ppp^n \cong R_\ppp / \ppp^nR_\ppp $, so without loss of generality, $ R $ is local, so a DVR, $ \ppp = \abr{\pi} $, and
$$ \cdot \pi : R / \abr{\pi^n} \xrightarrow{\sim} \abr{\pi} / \abr{\pi^{n + 1}}, $$
hence $ \br{R : \ppp^{n + 1}} = \br{R : \ppp}\br{\ppp : \ppp^{n + 1}} = \br{R : \ppp}\br{R : \ppp^n} $.
\end{enumerate}
\end{proof}

The quotient group
$$ \Cl R = \cbr{\text{fractional ideals of} \ R} / \cbr{\text{principal fractional ideals} \ aR \ \text{for} \ a \in K^\times} $$
is the \textbf{class group} of $ R $, or the \textbf{Picard group} $ \Pic R $. If $ K $ is a number field, write $ \Cl\br{K} = \Cl \OOO_K $, the \textbf{ideal class group} of $ K $.

\begin{fact*}
For a number field $ K $, $ \Cl\br{K} $ is finite.
\end{fact*}

\pagebreak

\subsection{Places of number fields}

Recall that $ \V_\QQ = \cbr{p \mid p \ \text{prime}} \cup \cbr{\infty} $. Let $ K $ be a number field. Let $ \ppp \subset \OOO_K $ be non-zero prime. Then $ \ppp $ determines a discrete valuation $ \v_\ppp $ of $ K $ and so a non-archimedean AV $ \abs{x}_\ppp = r^{-\v_\ppp\br{x}} $ for $ r > 1 $.

\begin{theorem}
This gives a bijection
$$ \cbr{\text{non-zero primes of} \ \OOO_K} \xrightarrow{\sim} \V_{K, \f}. $$
\end{theorem}

\begin{proof}
Let $ \ppp \ne \qqq $. Then there exists $ x \in \ppp \setminus \qqq $, and then $ \abs{x}_\ppp < 1 = \abs{x}_\qqq $, so $ \abs{\cdot}_\ppp $ and $ \abs{\cdot}_\qqq $ are inequivalent, so the map is injective. Let $ \abs{\cdot} $ be a non-archimedean AV on $ K $, with valuation ring $ R = \cbr{x \in K \st \abs{x} \le 1} $. As $ \abs{\cdot} $ is non-archimedean, $ \ZZ \subset R $, hence $ R \supset \OOO_K $, as $ R $ is integrally closed, and so $ R \supset \OOO_{K, \ppp} $ for some prime $ \ppp = \mmm_R \cap \OOO_K $. Thus $ R = \OOO_{K, \ppp} $, since by \ref{lem:1.1} $ \OOO_{K, \ppp} $ is a maximal subring of $ K $, so $ \abs{\cdot} $ and $ \abs{\cdot}_\ppp $ are equivalent.
\end{proof}

\begin{notation*}
If $ v \in \V_{K, \f} $, then
\begin{itemize}
\item $ \ppp_v $ is the corresponding prime ideal of $ \OOO_K $,
\item $ K_v $ is a complete discretely valued field, the completion of $ K $,
\item $ \OOO_v = \OOO_{K_v} \subset K_v $ is the valuation ring, not to be confused with $ \OOO_{K, \ppp_v} $,
\item $ \pi_v \in \OOO_v $ is any generator of the maximal ideal, the \textbf{uniformiser}, often assuming $ \pi_v \in K $,
\item $ v : K^\times \twoheadrightarrow \ZZ $ is the \textbf{normalised discrete valuation} such that $ v\br{\pi_v} = 1 $,
\item $ \kappa_v = \OOO_K / \ppp_v \cong \OOO_v / \abr{\pi_v} $ is finite of order $ \q_v = p^{\f_v} $ for a prime $ p $ such that $ v \mid p $, and
\item $ \abs{x}_v = \q_v^{-v\br{x}} $ is the \textbf{normalised AV}, so $ \abs{\pi_v}_v = 1 / \q_v $.
\end{itemize}
\end{notation*}

Recall that if $ L / K $ is a finite separable field extension and $ v $ is a place of $ K $, then $ L \otimes_K K_v \cong \prod_{w \mid v} L_w $. There is a unique infinite place $ \infty $ of $ \QQ $ and $ \QQ_\infty = \RR $. So
$$ K \otimes_\QQ \RR \xrightarrow{\sim} \prod_{v \in \V_{K, \infty}} K_v. $$
Each $ K_v $ is a finite extension of $ \RR $, so either $ K_v = \RR $, and $ v $ is \textbf{real}, or $ K_v \cong \CC $, and $ v $ is \textbf{complex}. In the second case, as $ K \subset K_v $ is dense, $ K \not\subset \RR $. On the other hand, by Galois theory, $ \Sigma_K = \cbr{\text{homomorphisms} \ \sigma : K \hookrightarrow \CC} $ has order $ n = \sbr{K : \QQ} $ and there is an isomorphism
\begin{equation}
\label{eq:1}
\function{K \otimes_\QQ \CC}{\prod_{\sigma \in \Sigma_K} \CC}{x \otimes z}{\br{\sigma\br{x}z}_\sigma}.
\end{equation}
Complex conjugation acts on both sides by $ x \otimes z \mapsto x \otimes \overline{z} $ and $ \br{z_\sigma}_\sigma \mapsto \br{\overline{z_{\overline{\sigma}}}}_\sigma $. Let
$$ \sigma_1, \dots, \sigma_{\r_1} : K \hookrightarrow \RR, \qquad \sigma_{\r_1 + 1} = \overline{\sigma_{\r_1 + \r_2 + 1}}, \dots, \sigma_{\r_1 + \r_2} = \overline{\sigma_{\r_1 + 2\r_2}} : K \hookrightarrow \CC, \qquad \r_1 + 2\r_2 = n. $$
Then taking fixed points under complex conjugation of $ \br{\ref{eq:1}} $,
$$ K \otimes_\QQ \RR \xrightarrow{\sim} \prod_{\sigma \ \text{real}} \RR \times \prod_{\br{\sigma, \overline{\sigma}}, \ \sigma \ne \overline{\sigma}} \cbr{\br{z, \overline{z}} \in \CC \times \CC} \cong \RR^{\r_1} \times \CC^{\r_2}. $$
Therefore the following holds.

\begin{theorem}
There is a bijection
$$ \function{\Sigma_K / \br{\sigma \sim \overline{\sigma}}}{\V_{K, \infty}}{\sigma}{\text{class of AV} \ \abs{\sigma\br{\cdot}} \ \text{in} \ \RR \ \text{or} \ \CC}. $$
\end{theorem}

\pagebreak

\begin{notation*}
Define
$$ K_\infty = K \otimes_\QQ \RR \cong \prod_{v \in \V_{K, \infty}} K_v \cong \RR^{\cbr{\text{real} \ v}} \times \CC^{\cbr{\text{complex} \ v}}, $$
where for $ v $ complex, $ K_v \cong \CC $ is well-defined up to complex conjugation. For normalised AVs,
\begin{itemize}
\item $ v $ real corresponds to $ \sigma : K \hookrightarrow \RR $ and $ \abs{x}_v = \abs{\sigma\br{x}} $ is the Euclidean AV, and
\item $ v $ complex corresponds to $ \sigma \ne \overline{\sigma} : K \hookrightarrow \CC $ and $ \abs{x}_v = \sigma\br{x}\overline{\sigma}\br{x} = \abs{\sigma\br{x}}^2 $ is the square of modulus.
\end{itemize}
\end{notation*}

\subsection{Extensions of places of number fields}

Let $ L / K $ be an extension of number fields, and let $ w \mid v $. If $ v $ is finite, $ L_w / K_v $ is a finite extension of non-archimedean local fields and $ \sbr{L_w : K_v} = \e\br{w \mid v}\f\br{w \mid v} $. If $ v $ is infinite,
$$ L_w / K_v \cong
\begin{cases}
\RR / \RR & \f = \e = 1 \\
\CC / \CC & \f = \e = 1 \\
\CC / \RR & \e = 2, \ \f = 1
\end{cases}.
$$

\lecture{4}{Thursday}{28/01/21}

\begin{proposition}
Let $ x \in L $ and $ v \in \V_K $. Then
$$ \abs{\N_{L / K}\br{x}}_v = \prod_{w \mid v} \abs{x}_w. $$
\end{proposition}

\begin{proof}
$ \N_{L / K}\br{x} = \prod_{w \mid v} \N_{L_w / K_v}\br{x} $ so it is enough to show $ \abs{\N_{L_w / K_v}\br{x}}_v = \abs{x}_w $. If $ v $ is finite, it is enough to take $ x = \pi_w \in L $, and
$$ \abs{\N_{L_w / K_v}\br{\pi_w}}_v = \abs{u\pi_v^{\f\br{w \mid v}}}_v = \q_v^{-\f\br{w \mid v}} = \q_w^{-1} = \abs{\pi_w}_w, \qquad u \in \OOO_{K_v}^\times. $$
If $ v $ is infinite, need only consider $ L_w / K_v \cong \CC / \RR $ and $ \N_{\CC / \RR}\br{z} = z\overline{z} $.
\end{proof}

\begin{theorem}[Product formula]
Let $ x \in K^\times $. Then $ \abs{x}_v = 1 $ for all but finitely many $ v $ and
$$ \prod_{v \in \V_K} \abs{x}_v = 1. $$
\end{theorem}

\begin{proof}
Let $ x = a / b $ for $ a, b \in \OOO_K \setminus \cbr{0} $. Then
$$ \cbr{v \in \V_K \st \abs{x}_v \ne 1} \subset \V_{K, \infty} \cup \cbr{v \in \V_{K, \f} \st v\br{a} > 0 \ \text{or} \ v\br{b} > 0} $$
is a finite set. Now
$$ \prod_{v \in \V_K} \abs{x}_v = \prod_{p \le \infty} \prod_{v \mid p} \abs{x}_v = \prod_{p \le \infty} \abs{\N_{K / \QQ}\br{x}}_p. $$
So it is enough to prove for $ K = \QQ $, and by multiplicativity, reduce to
\begin{itemize}
\item $ x = q $ prime, where
$$ \abs{q}_p =
\begin{cases}
\dfrac{1}{q} & p = q \\
1 & p \ne q, \infty \\
q & p = \infty
\end{cases},
$$
\item $ x = -1 $, where $ \abs{-1}_p = 1 $ for all $ p \le \infty $.
\end{itemize}
\end{proof}

\begin{remark*}
\hfill
\begin{itemize}
\item $ \RR $, with standard measure $ \d x $, transforms under $ a \in \RR^\times $ by $ \d\br{ax} = \abs{a}\d x $.
\item $ \CC $, with standard measure $ \d x\d y $, transforms under $ a \in \CC^\times $ by $ \d\br{ax}\d\br{ay} = \abs{a}^2\d x\d y $, with the normalised AV on $ \CC $.
\end{itemize}
\end{remark*}

\begin{fact*}
On $ K_v $, for any $ v $, there is a translation-invariant measure, the Haar measure, $ \d_v\br{x} $, and for all $ a \in K_v^\times $, $ \d_v\br{ax} = \abs{a}_v\d_v\br{x} $ where $ \abs{\cdot}_v $ is the normalised AV.
\end{fact*}

\pagebreak

\section{Different and discriminant}

\subsection{Discriminant}

Let $ R \subset S $ be rings, commutative with unity, such that $ S $ is a free $ R $-module of finite rank $ n \ge 1 $. Then we have a trace map given by
$$ \function[\Tr_{S / R}]{S}{R}{x}{\Tr\br{y \mapsto xy}}, $$
the trace of the $ R $-linear map $ S \to S \cong R^n $. If $ x_1, \dots, x_n \in S $, define
$$ \disc_{S / R} \br{x_i} = \disc \br{x_i} = \det \br{\Tr_{S / R}\br{x_ix_j}} \in R. $$
If $ y_i = \sum_{j = 1}^n r_{ji}x_j $ for $ r_{ji} \in R $, then $ \Tr_{S / R}\br{y_iy_j} = \sum_{k, l} r_{ki}r_{lj}\Tr_{S / R}\br{x_kx_l} $, so
\begin{equation}
\label{eq:2}
\disc \br{y_i} = \det \br{r_{ij}}^2 \disc \br{x_i}.
\end{equation}

\begin{definition*}
Let $ S = \bigoplus_{i = 1}^n Re_i $. Then the \textbf{discriminant}
$$ \disc\br{S / R} = \disc_{S / R} \br{e_i}R \subset R $$
is an ideal of $ R $, independent of the basis by $ \br{\ref{eq:2}} $.
\end{definition*}

The following are obvious properties.
\begin{itemize}
\item If $ S = S_1 \times S_2 $ for $ S_i $ free over $ R $, then
$$ \disc\br{S / R} = \disc\br{S_1 / R}\disc\br{S_2 / R}. $$
\item If $ f : R \to R' $ is a ring homomorphism, then
$$ \disc\br{S \otimes_R R' / R'} = f\br{\disc\br{S / R}}R'. $$
\item If $ R $ is a field, then $ \disc\br{S / R} = R $ or $ \disc\br{S / R} = \cbr{0} $ and $ \disc\br{S / R} = R $ if and only if the $ R $-bilinear form
$$ \function{S \times S}{R}{\br{x, y}}{\Tr_{S / R}\br{xy}} $$
is non-degenerate, that is there is a duality of the $ R $-vector space $ S $ with itself.
\end{itemize}
By field theory, if $ L / K $ is a finite field extension, then $ \disc\br{L / K} = K $ if and only if the trace form is non-degenerate, if and only if there exists $ x \in L $ with $ \Tr_{L / K}\br{x} \ne 0 $, if and only if $ L / K $ is separable. More generally is the following.

\begin{theorem}
\label{thm:3.1}
Let $ k $ be a field, and let $ A $ be a finite-dimensional $ k $-algebra. Then $ \disc\br{A / k} \ne 0 $, so $ \disc\br{A / k} = k $, if and only if $ A = \prod_i K_i $ for $ K_i / k $ a finite separable field extension.
\end{theorem}

\begin{proof}
Write $ A = \prod_{i = 1}^m A_i $ where $ A_i $ are indecomposable $ k $-algebras, so $ A_i $ is local. So may assume $ A $ is local with maximal ideal $ \mmm $. If $ \mmm = 0 $, that is $ A $ is a field, reduced to the previous statement. If not, then every element of $ \mmm $ is nilpotent, since $ \dim_k A < \infty $. So there exists $ x \in \mmm \setminus \cbr{0} $ nilpotent. So the endomorphism $ y \mapsto xy $ of $ A $ is nilpotent and for all $ r \in A $, so is $ y \mapsto \br{rx}y $, so for all $ r \in A $, $ \Tr_{A / k}\br{rx} = 0 $. So the trace form is degenerate, and the discriminant is zero. See Atiyah-Macdonald chapter on Artinian rings for an explanation of $ A = \prod_i A_i $.
\end{proof}

\lecture{5}{Saturday}{30/01/21}

Let $ R $ be a Dedekind domain, let $ K = \Frac R $, let $ L / K $ be finite separable, and let $ S $ be the integral closure of $ R $ in $ L $. Say $ S / R $ is an \textbf{extension of Dedekind domains}. Then $ S $ is a finitely generated $ R $-module, but need not be free.

\begin{proposition}
$ S $ is \textbf{locally free} $ R $-module of rank $ n = \sbr{L : K} $, that is for all $ \ppp \subset R $, $ S_\ppp \cong R_\ppp^n $.
\end{proposition}

\begin{proof}
$ S \subset L $ so $ S $ is torsion-free, hence so is $ S_\ppp $, and $ R_\ppp $ is a PID, so $ S_\ppp $ is free, clearly of rank $ \dim_K L = n $.
\end{proof}

\pagebreak

\begin{lemma}
If $ x \in S $, then $ \Tr_{L / K}\br{x} \in R $.
\end{lemma}

\begin{proof}
If $ R $ is local, then $ S $ is a free $ R $-module so $ \Tr_{L / K}\br{x} = \Tr_{S \otimes_R K / K}\br{x \otimes 1} = \Tr_{S / R}\br{x} \in R $. So in general, for all $ 0 \ne \ppp \subset R $, $ y = \Tr_{L / K}\br{x} \in R_\ppp $ and
$$ \bigcap_\ppp R_\ppp = \cbr{x \in K \st \forall \ppp, \ \v_\ppp\br{x} \ge 0} = R. $$
\end{proof}

Then there are two equivalent definitions of $ \disc\br{S / R} $.

\begin{definition*}
$ \disc\br{S / R} $ is defined to be the ideal of $ R $ generated by
$$ \cbr{\disc_{L / K} \br{x_1, \dots, x_n} \st x_1, \dots, x_n \in S}. $$
\end{definition*}

If $ S / R $ is free, this gives the previous definition. As $ S \otimes_R K = L $ is separable over $ K $, $ \disc\br{L / K} = K \ne 0 $ and so $ \disc\br{S / R} \ne 0 $. This is how we prove that $ S / R $ is finitely generated.

\begin{proposition}
\label{prop:3.4}
$ \disc\br{S / R}R_\ppp = \disc\br{S_\ppp / R_\ppp} $ for all $ \ppp $.
\end{proposition}

\begin{proof}
Claim there exist $ x_1, \dots, x_n \in S $ which is an $ R_\ppp $-basis for $ S_\ppp $. Certainly there exist $ e_1, \dots, e_n \in S_\ppp $ which is an $ R_\ppp $-basis. Let
$$ \QQQ = \cbr{\text{primes} \ \qqq \subset S \st \exists i, \ \v_\qqq\br{e_i} < 0} $$
be a finite set. By CRT, there exist $ a_i \in S $ such that $ \v_\qqq\br{a_i} + \v_\qqq\br{e_i} \ge 0 $ for all $ \qqq \in \QQQ $ and $ a_i - 1 \in \ppp S $. Then $ x_i = a_ie_i \in S $ and $ x_i \equiv e_i \mod \ppp S $. So $ \br{x_i} $ is an $ R / \ppp $-basis for $ S / \ppp S = S_\ppp / \ppp S_\ppp $, so $ \br{x_i} $ is an $ R_\ppp $-basis for $ S_\ppp $. Thus $ \disc\br{S_\ppp / R_\ppp} = \disc \br{x_i}R_\ppp $, and $ \disc \br{x_i} \in \disc\br{S / R} $. So $ \disc \br{S_\ppp / R_\ppp} \subset \disc\br{S / R}R_\ppp $ and the other inclusion is obvious.
\end{proof}

There is an alternative definition of $ \disc\br{S / R} $. If $ x_1, \dots, x_n \in S $ is a $ K $-basis for $ L $, then $ \disc_{L / K} \br{x_i} \ne 0 $. Let
$$ \PPP = \cbr{\ppp \subset R \st \v_\ppp\br{\disc_{L / K} \br{x_i}} > 0} $$
be a finite set. So for all $ \ppp \notin \PPP $, $ \disc\br{S_\ppp / R_\ppp} = R_\ppp $.

\begin{definition*}
Define
$$ \disc\br{S / R} = \prod_{\ppp \in \PPP} \ppp^{\v_\ppp\br{\disc\br{S_\ppp / R_\ppp}}}, $$
which is equivalent by \ref{prop:3.4} to the previous definition.
\end{definition*}

\begin{theorem}
\label{thm:3.5}
$ \v_\ppp\br{\disc\br{S / R}} = 0 $ if and only if $ \ppp $ is unramified in $ S $ and for all $ \qqq \subset S $ over $ \ppp $, the residue field extension $ \br{S / \qqq} / \br{R / \ppp} $ is separable.
\end{theorem}

\begin{proof}
May assume $ R $ is local, so $ S $ is free over $ R $. Have $ \ppp S = \prod_\qqq \qqq^{e_\qqq} $, so
$$ S \otimes_R \br{R / \ppp} \cong S / \ppp S \cong \prod_\qqq S / \qqq^{e_\qqq}. $$
So $ \v_\ppp\br{\disc\br{S / R}} = 0 $ if and only if $ \disc\br{\br{S / \ppp S} / \br{R / \ppp}} = R / \ppp $, if and only if each $ S / \qqq^{e_\qqq} $ is a finite separable field extension of $ R / \ppp $ by \ref{thm:3.1}, if and only if for all $ \qqq $, $ e_\qqq = 1 $ and $ \br{S / \qqq} / \br{R / \ppp} $ is separable.
\end{proof}

\begin{corollary}
In an extension $ S / R $ of Dedekind domains, only finitely many primes are ramified, just the $ \ppp $ such that $ \v_\ppp\br{\disc\br{S / R}} > 0 $.
\end{corollary}

\begin{proposition}
Let $ \ppp \subset R $. Then
$$ \v_\ppp\br{\disc\br{S / R}} = \sum_{\qqq \supset \ppp} \v_\ppp\br{\disc\br{\widehat{S_\qqq} / \widehat{R_\ppp}}}. $$
\end{proposition}

\begin{proof}
By \ref{prop:3.4} may assume $ R $ is local, so $ S $ is a free $ R $-module, and $ S \otimes_R \widehat{R} \cong \prod_{\qqq \subset S} \widehat{S_\qqq} $ so
$$ \v_\ppp\br{\disc\br{S / R}} = \v_\ppp\br{\disc\br{S \otimes_R \widehat{R} / \widehat{R}}} = \sum_\qqq \v_\ppp\br{\disc\br{\widehat{S_\qqq} / \widehat{R}}}. $$
\end{proof}

\pagebreak

\subsection{Different}

There is a finer invariant of ramification.

\begin{definition*}
The \textbf{inverse different} $ \DDD_{S / R}^{-1} $ of an extension $ S / R $ of Dedekind domains is
$$ \DDD_{S / R}^{-1} = \cbr{x \in L \st \forall y \in S, \ \Tr_{L / K}\br{xy} \in R}. $$
\end{definition*}

This is the dual of $ S $ with respect to the trace form $ \br{x, y} \mapsto \Tr_{L / K}\br{xy} $, which is non-degenerate and clearly an $ S $-submodule of $ L $. If $ \bigoplus_{i = 1}^n Rx_i \subset S $, let $ \br{y_i} $ be the dual basis to $ \br{x_i} $ for the trace form, that is $ \Tr_{L / K}\br{x_iy_j} = \delta_{ij} $. Then $ S \subset \DDD_{S / R}^{-1} \subset \bigoplus_{i = 1}^n Ry_i $, so $ \DDD_{S / R}^{-1} $ is a fractional ideal, since it is finitely generated.

\begin{definition*}
$ \DDD_{S / R} $ is an ideal of $ S $, the \textbf{different}.
\end{definition*}

\begin{proposition}
\label{prop:3.8}
\hfill
\begin{enumerate}
\item If $ \ppp \subset R $, then $ \DDD_{S_\ppp / R_\ppp} = \DDD_{S / R}S_\ppp $.
\item $ \N_{L / K}\br{\DDD_{S / R}} = \disc\br{S / R} $.
\item Let $ \qqq \subset S $ lying over $ \ppp \subset R $. Then $ \v_\qqq\br{\DDD_{S / R}} = \v_\qqq\br{\DDD_{\widehat{S_\qqq} / \widehat{R_\ppp}}} $.
\end{enumerate}
\end{proposition}

\begin{proof}
\hfill
\begin{enumerate}
\item Exercise. \footnote{Exercise: the same idea as \ref{prop:3.4}}
\item By $ 1 $ and \ref{prop:3.4}, can suppose $ R $ is local. Then $ S $ is a PID by \ref{thm:2.1}.$ 3 $. So $ \DDD_{S / R}^{-1} = x^{-1}S $ for some $ 0 \ne x \in S $. Let $ \br{e_i} $ be a basis for $ S $ over $ R $. Then there exists a basis $ \br{e_i'} $ for $ S $ over $ R $ such that $ \Tr_{L / K}\br{e_ix^{-1}e_j'} = \delta_{ij} $. Let $ x^{-1}e_j' = \sum_k b_{kj}e_k $ for $ b_{kj} \in K $. Then
$$ \abr{1} = \abr{\det \br{\Tr_{L / K}\br{e_ix^{-1}e_j'}}} = \abr{\det \br{\Tr_{L / K}\br{e_ie_j}}\det \br{b_{ij}}} = \det \br{b_{ij}}\disc\br{S / R}. $$
But $ \N_{L / K}\br{x^{-1}} $ is $ \det \br{b_{ij}} $ times some unit in $ R $. So $ \abr{1} = \abr{\N_{L / K}\br{x^{-1}}}\disc\br{S / R} $.

\lecture{6}{Tuesday}{02/02/21}

\item Assume $ R $ is local and $ \ppp = \abr{\pi_\ppp} $. Write $ \widehat{K} = \Frac \widehat{R} $ and for $ \qqq = \abr{\pi_\qqq} \subset S $ write $ \widehat{L_\qqq} = \Frac \widehat{S_\qqq} $. So say
$$ L \otimes_K \widehat{K} \supset S \otimes_R \widehat{R} \xrightarrow{\sim} \prod_\qqq \widehat{S_\qqq} \subset \prod_\qqq \widehat{L_\qqq}, $$
and
\begin{equation}
\label{eq:3}
\Tr_{L \otimes_K \widehat{K} / \widehat{K}}\br{x} = \sum_\qqq \Tr_{\widehat{L_\qqq} / \widehat{K}}\br{x}.
\end{equation}
Let $ S = \bigoplus_{i = 1}^n Rx_i $, and $ \prod_\qqq \pi_\qqq^{-a_\qqq}S = \DDD_{S / R}^{-1} = \bigoplus_{i = 1}^n Ry_i $ for some $ a_\qqq \ge 0 $ and $ y_i \in L $, the dual basis to $ x_i $. Then as $ S \otimes_R \widehat{R} = \bigoplus_{i = 1}^n \widehat{R}\br{x_i \otimes 1} $,
\begin{align*}
\DDD_{S \otimes_R \widehat{R} / \widehat{R}}^{-1}
& = \cbr{x \in L \otimes_K \widehat{K} \st \forall y \in S \otimes_R \widehat{R}, \ \Tr_{L \otimes_K \widehat{K} / \widehat{K}}\br{xy} \in \widehat{R}} \\
& = \bigoplus_{i = 1}^n \widehat{R}\br{y_i \otimes 1}
= \DDD_{S / R}^{-1}\br{S \otimes_R \widehat{R}}
= \prod_\qqq \pi_\qqq^{-a_\qqq}\br{S \otimes_R \widehat{R}}
\subset L \otimes_K \widehat{K},
\end{align*}
since $ \Tr_{L / K}\br{x_iy_j} = \delta_{ij} $ and trace commutes with base change. On the other hand, by $ \br{\ref{eq:3}} $ and the definitions
$$ \DDD_{S \otimes_R \widehat{R} / \widehat{R}}^{-1} \cong \prod_\qqq \DDD_{\widehat{S_\qqq} / \widehat{R}}^{-1} \subset \prod_\qqq \widehat{L_\qqq}, $$
so
$$ \DDD_{\widehat{S_\qqq} / \widehat{R}}^{-1} = \prod_{\qqq'} \pi_{\qqq'}^{-a_{\qqq'}}\widehat{S_\qqq} = \pi_\qqq^{-a_\qqq}\widehat{S_\qqq}, $$
as $ \v_\qqq\br{\pi_{\qqq'}} = 0 $ if $ \qqq' \ne \qqq $.
\end{enumerate}
\end{proof}

\pagebreak

Use this to prove the following.

\begin{theorem}
Assume all extensions of residue fields are separable. Let $ \ppp S = \prod_{i = 1}^g \qqq_i^{e_i} \subset S $. Then
\begin{enumerate}
\item $ \qqq_i \mid \DDD_{S / R} $ if and only if $ e_i > 1 $, and
\item $ \qqq_i^{e_i - 1} \mid \DDD_{S / R} $.
\end{enumerate}
\end{theorem}

\begin{proof}
First assume $ R $ is complete local and $ \ppp = \abr{\pi_R} $. Then $ S $ is also local, and complete, with unique prime $ \qqq = \abr{\pi_S} $, so $ g = 1 $.
\begin{enumerate}
\item So $ \DDD_{S / R} = \abr{\pi_S}^d $ for $ d \ge 0 $. By \ref{prop:3.8}.$ 2 $, $ \disc\br{S / R} = \abr{\N_{L / K}\br{\pi_S}^d} = \abr{\pi_R}^{d\f} $. So as $ \v_\ppp\br{\disc\br{S / R}} = 0 $ if and only if $ \ppp $ is unramified by \ref{thm:3.5}, get the first statement.
\item Claim $ \Tr_{L / K}\br{\qqq} \subset \ppp $. Let $ x \in \qqq $. Then multiplication by $ x $ is a nilpotent endomorphism of $ S \otimes_R \br{R / \ppp} \cong S / \qqq^\e $, so $ \Tr_{S \otimes_R \br{R / \ppp} / \br{R / \ppp}}\br{x \otimes 1} = 0 $, that is $ \Tr_{L / K}\br{x} = \Tr_{S / R}\br{x} \in \ppp $. Hence the claim. Therefore $ \Tr_{L / K}\br{\qqq^{1 - \e}} = \Tr_{L / K}\br{\pi_R^{-1}\qqq} \subset R $, so $ \qqq^{1 - \e} \subset \DDD_{S / R}^{-1} $, that is $ \qqq^{\e - 1} \mid \DDD_{S / R} $.
\end{enumerate}
For the general case, apply the above to $ \widehat{S_{\qqq_i}} / \widehat{R_\ppp} $ and use \ref{prop:3.8}.$ 3 $.
\end{proof}

\begin{fact*}
\hfill
\begin{itemize}
\item If $ \ppp \nmid e_i $ then $ \v_{\qqq_i}\br{\DDD_{S / R}} = e_i - 1 $. If $ \ppp \mid e_i $ then $ \v_{\qqq_i}\br{\DDD_{S / R}} \ge e_i $. More precisely, $ \v_{\qqq_i}\br{\DDD_{S / R}} $ is determined by the orders of the higher ramification groups, for a Galois closure of $ L / K $. See for example Serre, Local fields, Chapter $ 4 $, Section $ 1 $, Proposition $ 4 $.
\item If $ S = R\sbr{x} $, and $ x $ has minimal polynomial $ f \in R\sbr{T} $ then $ \DDD_{S / R} = \abr{f'\br{x}} $ where $ f' $ is the derivative. See example sheet $ 1 $. This means that $ \DDD_{S / R} $ is the annihilator of the cyclic $ S $-module $ \Omega_{S / R} $ of K\"ahler differentials, generated by $ \d x $.
\end{itemize}
\end{fact*}

For an extension $ L / K $ of number fields write
$$ \DDD_{L / K} = \DDD_{\OOO_L / \OOO_K} \subset \OOO_L, \qquad \delta_{L / K} = \disc\br{\OOO_L / \OOO_K} \subset \OOO_K. $$

\begin{remark*}
Let $ K / \QQ $, and let $ \br{e_i} $ be a $ \ZZ $-basis for $ \OOO_K $. Then $ \delta_{K / \QQ} \subset \ZZ $ is $ \abr{\disc \br{e_i}} $ and if $ \br{e_i'} $ is another basis such that $ e_i' = \sum_{i, j} a_{ji}e_j $, then $ \disc \br{e_i'} = \br{\det \br{a_{ij}}}^2\disc \br{e_i} = \disc \br{e_i} $, since $ \det \br{a_{ij}} = \pm 1 $. So the integer $ \disc \br{e_i} $ is independent of the basis, not just the ideal it generates. This is called the \textbf{absolute discriminant} $ \d_K \in \ZZ \setminus \cbr{0} $ of $ K $. The sign is significant.
\end{remark*}

\begin{theorem}[Kummer-Dedekind criterion]
Let $ S / R $ be an extension of Dedekind domains, and let $ x \in S $ such that $ L = K\br{x} $. Suppose $ \ppp \subset R $ such that $ S_\ppp = R_\ppp\sbr{x} $. Let $ g \in R\sbr{T} $ be the minimal polynomial of $ x $ and $ g = \prod_i \overline{g_i}^{e_i} \in \br{R / \ppp}\sbr{T} $ the factorisation of reduction of $ g $ into powers of distinct monic irreducibles $ \overline{g_i} $. Let $ g_i \in R\sbr{T} $ be any monic lifting of $ \overline{g_i} $ and $ f_i = \deg g_i = \deg \overline{g_i} $. Then $ \qqq_i = \ppp S + \abr{g_i\br{x}} \subset S $ is prime with
$$ \sbr{S / \qqq_i : R / \ppp} = f_i, \qquad \forall i \ne j, \ \qqq_i \ne \qqq_j, \qquad \ppp S = \prod_i \qqq_i^{e_i}. $$
\end{theorem}

\begin{proof}
Can assume $ R $ is local, so then $ S = R\sbr{x} $. Set $ \ppp = \abr{\pi} $ and $ R / \ppp = \kappa $. Then $ \qqq_i $ is prime with residue degree $ f_i $, since $ S / \qqq_i \cong \kappa\sbr{T} / \abr{\overline{g_i}} $, and $ \overline{g_i} $ is irreducible of degree $ f_i $. Claim that $ \qqq_i \ne \qqq_j $. If $ i \ne j $, there exist $ a, b \in R\sbr{T} $ such that $ \overline{a}\overline{g_i} + \overline{b}\overline{g_j} = 1 \in \kappa\sbr{T} $, so $ 1 = ag_i + bg_j + \pi c $ for some $ c \in R\sbr{T} $, so $ 1 \in \abr{\pi, g_i\br{x}, g_j\br{x}} = \qqq_i + \qqq_j $. Let $ g = \prod_i g_i^{e_i} + \pi h $ for $ h \in R\sbr{T} $. Then
$$ \prod_i \qqq_i^{e_i} = \prod_i \abr{\pi, g_i\br{x}}^{e_i} \subset \prod_i \abr{\pi, g_i\br{x}^{e_i}} \subset \abr{\pi, \prod_i g_i\br{x}^{e_i}} = \abr{\pi, \pi h\br{x}} \subset \ppp S = \abr{\pi}. $$
Now $ \dim_\kappa \br{S / \ppp S} = n = \sbr{L : K} $, and
$$ \dim_\kappa \br{S / \qqq_i^{e_i}} = \sum_{j = 0}^{e_i - 1} \dim_\kappa \br{\qqq_i^j / \qqq_i^{j + 1}} = e_i\dim_\kappa \br{S / \qqq_i} = e_if_i, $$
so $ \prod_i \qqq_i^{e_i} \subset \ppp S $ gives $ \sum_i e_if_i \ge n $. As $ \sum_i e_if_i = \sum_i e_i\deg \overline{g_i} = \deg \overline{g} = n $, have equality.
\end{proof}

\pagebreak

\section{Example: quadratic fields}

\lecture{7}{Thursday}{04/02/21}

Let $ K = \QQ\br{\sqrt{d}} $ for $ d \in \QQ^\times $ not a square. Multiplying $ d $ by a square, can assume $ d \in \ZZ \setminus \cbr{0, 1} $ is squarefree. Then $ \OOO_K \supset \ZZ\sbr{\sqrt{d}} = \ZZ \oplus \ZZ\sqrt{d} $.

\subsection{Discriminant and different}

Since $ \Tr_{K / \QQ}\br{1} = 2 $ and $ \Tr_{K / \QQ}\br{\sqrt{d}} = 0 $, $ \disc \br{1, \sqrt{d}} = 4d $, so either $ \d_K = 4d $, and
$$ \OOO_K = \ZZ\sbr{\sqrt{d}}, $$
or $ \d_K = d $, and $ \br{\OOO_K : \ZZ\sbr{\sqrt{d}}} = 2 $. This holds if and only if there exist $ m, n \in \ZZ $ not both even with $ \tfrac{m + n\sqrt{d}}{2} \in \OOO_K $, if and only if $ \tfrac{1 + \sqrt{d}}{2} \in \OOO_K $ since obviously $ \tfrac{1}{2}, \tfrac{\sqrt{d}}{2} \notin \OOO_K $, if and only if $ d \equiv 1 \mod 4 $ since the minimal polynomial of $ \tfrac{1 + \sqrt{d}}{2} $ is $ \br{T - \tfrac{1}{2}}^2 - \tfrac{d}{4} = T^2 - T - \tfrac{d - 1}{4} $, in which case
$$ \OOO_K = \ZZ \oplus \ZZ\tfrac{1 + \sqrt{d}}{2} = \ZZ\sbr{\tfrac{1 + \sqrt{d}}{2}}. $$
The dual basis of $ \br{1, \sqrt{d}} $ for the trace form is $ \br{\tfrac{1}{2}, \tfrac{1}{2\sqrt{d}}} $, so
$$ \DDD_{K / \QQ} =
\begin{cases}
\abr{2\sqrt{d}} & d \not\equiv 1 \mod 4 \\
\abr{\sqrt{d}} & d \equiv 1 \mod 4
\end{cases}.
$$

\subsection{Decomposition of primes}

By Kummer-Dedekind.
\begin{itemize}
\item If $ p \ne 2 $ or $ d \not\equiv 1 \mod 4 $ then $ p \nmid \br{\OOO_K : \ZZ\sbr{\sqrt{d}}} $. So applying the criterion to $ T^2 - d $, see that
\begin{itemize}
\item $ \abr{p} = \ppp^2 $ is ramified if $ p \mid d $, so $ \ppp = \abr{p, \sqrt{d}} $,
\item $ \abr{p} = \ppp $ is inert if $ \br{\tfrac{d}{p}} = -1 $, and
\item $ \abr{p} = \ppp\ppp' $ is split if $ \br{\tfrac{d}{p}} = 1 $, so if $ d \equiv a^2 \mod p $ then $ \ppp = \abr{p, \sqrt{d} - a} \ne \abr{p, \sqrt{d} + a} = \ppp' $.
\end{itemize}
\item The remaining case is $ p = 2 $ and $ d \equiv 1 \mod 4 $. Factoring $ T^2 - T - \tfrac{d - 1}{4} $ modulo two, get
\begin{itemize}
\item $ \abr{2} $ is inert if $ d \equiv 5 \mod 8 $, and
\item $ \abr{2} = \ppp\ppp' $ is split if $ d \equiv 1 \mod 8 $ and $ \ppp = \abr{2, \tfrac{\sqrt{d} + 1}{2}} \ne \abr{2, \tfrac{\sqrt{d} - 1}{2}} = \ppp' $.
\end{itemize}
\end{itemize}
Go through the calculations if you have not seen them before. \footnote{Exercise}

\pagebreak

\section{Example: cyclotomic fields}

Recall some Galois theory. Let $ n > 1 $, and let $ K $ be a field of characteristic zero or characteristic $ p \nmid n $. Suppose $ L = K\br{\zeta_n} $, where $ \zeta_n \in L $ is a primitive $ n $-th root of unity, that is $ \zeta_n^m \ne 1 $ for all $ 1 \le m < n $. Equivalently, $ \zeta_n $ is a root of the $ n $-th cyclotomic polynomial $ \Phi_n \in \ZZ\sbr{T} $ of degree $ \phi\br{n} $, defined recursively by
$$ T^n - 1 = \prod_{d \mid n} \Phi_d\br{T}. $$
Then $ L / K $ is Galois, with abelian Galois group, and
$$ \function{\Gal\br{L / K}}{\br{\ZZ / n\ZZ}^\times}{g}{\text{unique} \ a \mod n \ \text{such that} \ g\br{\zeta_n} = \zeta_n^a}. $$
is an injective homomorphism.

\subsection{Cyclotomic fields}

\begin{theorem}
\label{thm:5.1}
Let $ L = \QQ\br{\zeta_n} $. Then
\begin{enumerate}
\item $ \Gal\br{L / \QQ} \xrightarrow{\sim} \br{\ZZ / n\ZZ}^\times $,
\item $ p $ ramifies in $ L $ if and only if $ p \mid n $, and
\item $ \OOO_L = \ZZ\sbr{\zeta_n} $.
\end{enumerate}
\end{theorem}

\begin{remark*}
$ 1 $ if and only if $ \Phi_n $ is irreducible over $ \QQ $, if and only if $ \sbr{L : \QQ} = \phi\br{n} $.
\end{remark*}

\begin{proof}
Let $ n = p^rm $ for $ r \ge 1 $ and $ p \nmid m $ prime. Let $ \zeta_m = \zeta_n^{p^r} $ and $ \zeta_{p^r} = \zeta_n^m $. Then there exist $ a, b \in \ZZ $ such that $ p^ra + mb = 1 $, so $ \zeta_n = \zeta_m^a\zeta_{p^r}^b $. Let $ K = \QQ\br{\zeta_m} $. Then $ L = K\br{\zeta_{p^r}} $. Will prove that
\begin{itemize}
\item $ \Phi_{p^r} $ is irreducible over $ K $,
\item if $ v \in \V_{K, \f} $ and $ v \nmid p $ then $ v $ is unramified in $ L / K $,
\item if $ v \mid p $ then $ v $ is totally ramified in $ L / K $, and
\item $ \OOO_L = \OOO_K\sbr{\zeta_{p^r}} $.
\end{itemize}
This proves \ref{thm:5.1} by induction on $ n $. For a place $ w $ of $ L $, write $ x_w \in L_w $ for the image of $ \zeta_{p^r} $ under $ L \hookrightarrow L_w $. Suppose $ v \mid p $. By induction, $ p $ is unramified in $ K / \QQ $, so $ v\br{p} = 1 $. Then
$$ \Phi_{p^r}\br{T + 1} = \dfrac{\br{T + 1}^{p^r} - 1}{\br{T + 1}^{p^{r - 1}} - 1} $$
is an Eisenstein polynomial in $ \OOO_{K_v}\sbr{T} $. Indeed $ \Phi_{p^r}\br{T + 1} \equiv T^{p^{r - 1}\br{p - 1}} \mod p $, and the constant coefficient is $ p $, so has valuation one. Then from local fields,
\begin{itemize}
\item $ \Phi_{p^r} $ is irreducible over $ K_v $, hence over $ K $,
\item $ L / K $ is totally ramified at $ v $, and
\item if $ w $ is the unique place of $ L $ over $ v $, then $ \OOO_{L_w} = \OOO_{K_v}\sbr{\pi_w} $ where $ \pi_w = x_w - 1 $ is the root of $ \Phi_{p^r}\br{T + 1} $ in $ L_w $.
\end{itemize}
Now let $ v \mid q \ne p $. Then $ \Phi_{p^r} $ is separable modulo $ q $. Have
$$ K_v \otimes_K L \cong \prod_{w \mid v} L_w = \prod_{w \mid v} K_v\br{x_w}. $$
Let $ f_w \in \OOO_{K_v}\sbr{T} $ be the minimal polynomial of $ x_w $ over $ K_v $. Then
\begin{itemize}
\item $ \prod_{w \mid v} f_w = \Phi_{p^r} $, so the reduction of $ f_w $ at $ v $ is separable, hence $ L_w / K_v $ is unramified, and
\item by local fields again, $ \OOO_{L_w} = \OOO_{K_v}\sbr{x_w} $.
\end{itemize}

\pagebreak

Thus for all $ v \in \V_{K, \f} $,
$$ \OOO_{K_v} \otimes_{\OOO_K} \OOO_K\sbr{\zeta_{p^r}} \cong \OOO_{K_v}\sbr{T} / \abr{\Phi_{p^r}} \cong \prod_{w \mid v} \OOO_{K_v}\sbr{T} / \abr{f_w} = \prod_{w \mid v} \OOO_{L_w} \cong \OOO_{K_v} \otimes_{\OOO_K} \OOO_L, $$
by CRT, so must have $ \OOO_K\sbr{\zeta_{p^r}} = \OOO_L $.
\end{proof}

\subsection{Quadratic reciprocity}

Recall Frobenius elements. Let $ L / K $ be a Galois extension of number fields, let $ w \mid v $ be finite places, and let $ G = \Gal\br{L / W} \supset G_w \cong \Gal\br{L_w / K_v} $ be the decomposition group of $ w $. Then
$$ 1 \to \I_w \to G_w \to \Gal\br{\ell_w / \kappa_v} \to 1, $$
where $ \I_w $ is the inertia subgroup. Suppose $ w $ is unramified in $ L / K $, if and only if $ v $ is unramified in $ L / K $. Then $ \I_w = \cbr{1} $. Define the \textbf{Frobenius} at $ w $ to be the unique element $ \sigma_w \in G_w $ mapping to the generator $ x \mapsto x^{\q_v} $ of $ \Gal\br{\ell_w / \kappa_v} $. So $ \ord \sigma_w = \f\br{w \mid v} = \sbr{\ell_w : \kappa_v} = \sbr{\ell_{w'} : \kappa_v} $ for any $ w' \mid v $, as $ G $ acts transitively on $ \cbr{w'} $. In particular, $ \sigma_w = 1 $ if and only if $ v $ splits completely in $ L / K $, that is there exist $ \sbr{L : K} $ places of $ L $ over $ v $. Suppose $ G $ is abelian. Then $ G_w $ and $ \sigma_w $ are independent of $ w $, so depends only on $ v $.

\begin{notation*}
$ \sigma_v = \sigma_{L / K, v} = \sigma_w $ is the \textbf{arithmetic Frobenius} at $ v $. There are other notations, such as $ \phi_{L / K, v} $ or $ \br{v, L / K} $, the \textbf{norm residue symbol}.
\end{notation*}

\begin{remark*}
Let $ L / F / K $ where $ L / K $ is abelian. Then $ \eval{\sigma_{L / K}}_F = \sigma_{F / K} $ by definition.
\end{remark*}

Let $ L = \QQ\br{\zeta_n} $, let $ K = \QQ $, and let $ n > 2 $. Have an isomorphism
$$ \function[\lambda]{\br{\ZZ / n\ZZ}^\times}{\Gal\br{L / \QQ}}{a \mod n}{\br{\zeta_n \mapsto \zeta_n^a}}. $$
Claim that
$$ \sigma_p = \sigma_{L / \QQ, p} = \lambda\br{p \mod n} = \br{\zeta_n \mapsto \zeta_n^p} \in \Gal\br{L / \QQ}, $$
if $ p \nmid n $. Indeed, $ \sigma_p $ is characterised by for all $ v \mid p $, $ \sigma_p $ induces $ x \mapsto x^p $ on the residue field $ \ZZ\sbr{\zeta_n} / \ppp_v $, whereas $ \lambda\br{p} $ induces $ x \mapsto x^p $ over $ \ZZ\sbr{\zeta_n} / \abr{p} $.

\lecture{8}{Saturday}{06/02/21}

\begin{remark*}
\hfill
\begin{itemize}
\item These elements $ \sigma_p $ generate $ \Gal\br{L / \QQ} $, since every integer prime to $ n $ is a product of $ p \nmid n $, so gives, with some thought, another proof that $ \Gal\br{L / \QQ} \cong \br{\ZZ / n\ZZ}^\times $.
\item If $ \sigma : L \hookrightarrow \CC $ is any embedding, then $ \overline{\sigma\br{\zeta_n}} = \sigma\br{\zeta_n^{-1}} $. So $ \lambda\br{-1 \mod n} $ is complex conjugation, for any $ \sigma : L \hookrightarrow \CC $.
\end{itemize}
\end{remark*}

Specialise to the case $ n = q > 2 $ is prime. Then $ \Gal\br{L / \QQ} = \br{\ZZ / q\ZZ}^\times $ is cyclic of order $ q - 1 $, so has a unique index two subgroup $ H \cong \br{\br{\ZZ / q\ZZ}^\times}^2 $. Let $ K = L^H $ be a quadratic extension of $ \QQ $. Every $ p \ne q $ is unramified in $ L $, hence also in $ K $. So $ K = \QQ\br{\sqrt{\pm q}} $, and as $ \abr{2} $ is unramified in $ K $, must have
$$ K = \QQ\br{\sqrt{q^*}}, \qquad q^* =
\begin{cases}
q & q \equiv 1 \mod 4 \\
-q & q \equiv 3 \mod 4
\end{cases},
\qquad \d_K = q^*. $$
Now let $ p \ne q $ be an odd prime. Then
$$ \sigma_{K / \QQ, p} = 1 \qquad \iff \qquad \sigma_{L / \QQ, p} = \lambda\br{p} \in H \qquad \iff \qquad \br{\tfrac{p}{q}} = 1. $$
But
$$ \sigma_{K / \QQ, p} = 1 \qquad \iff \qquad p \ \text{splits completely in} \ K \qquad \iff \qquad \br{\tfrac{q^*}{p}} = 1. $$
That is, $ \br{\tfrac{p}{q}} = \br{\tfrac{q^*}{p}} $. Combine with $ \br{\tfrac{-1}{q}} = \br{-1}^{\br{p - 1} / 2} $ to get the quadratic reciprocity law. In algebraic number theory, quadratic reciprocity says that splitting of $ p $ in $ K / \QQ $ depends only on the congruence class of $ p $ modulo something. Class field theory tells us that a similar thing holds for any abelian extension of number fields, since there is a law describing the decomposition of primes in an abelian extension which is just a congruence condition.

\pagebreak

\section{Ideles and adeles}

To study congruences modulo $ p^n $ for $ n \ge 1 $ Hensel introduced $ \ZZ_p $ and $ \QQ_p $ such that $ \QQ \hookrightarrow \ZZ_p $. For congruences to arbitrary moduli, or to study local-global problems in general, it would be nice to simultaneously embed $ \QQ \hookrightarrow \QQ_p $ for all $ p \le \infty $, which are locally compact. The first guess is $ \QQ \hookrightarrow \prod_{p \le \infty} \QQ_p $, but this product is not nice, for example not locally compact. Better is to notice that if $ x \in \QQ $, then the image of $ x $ lies in $ \ZZ_p $ for all but finitely many $ p $. So Chevalley introduced a small product with better properties, for any number field $ K $, the ring of adeles or valuation vectors $ \AA_K $ of $ K $ and the group of ideles $ \JJJ_K = \AA_K^\times $ of $ K $. These are topological rings and groups respectively. They are highly disconnected, that is have plenty of open subgroups. Open subgroups are closed, so if $ H \subset G $ is an open subgroup, then $ G / H $ is discrete, that is $ G = \bigsqcup_x xH $ is a topological disjoint union.

\subsection{Adeles}

Let $ K $ be a number field, let $ \V_K = \V_{K, \infty} \sqcup \V_{K, \f} $, and let $ K_v $ be its completions. If $ v \in \V_{K, \f} $, have $ \OOO_v = \OOO_{K_v} = \cbr{x \st \abs{x}_v \le 1} \subset K_v $.

\begin{definition*}
The \textbf{adele ring} of $ K $ is
$$ \AA_K = \cbr{\br{x_v} \in \prod_{v \in \V_K} K_v \st \text{for all but finitely many} \ v, \ x_v \in \OOO_v} = \bigcup_{\text{finite} \ S \subset \V_{K, \f}} \U_{K, S} \subset \prod_{v \in \V_K} K_v, $$
where
$$ \U_{K, S} = \prod_{v \in \V_{K, \infty}} K_v \times \prod_{v \in S} K_v \times \prod_{v \in \V_{K, \f} \setminus S} \OOO_v. $$
\end{definition*}

\begin{notation*}
Let
$$ K_\infty = \prod_{v \in \V_{K, \infty}} K_v = K \otimes_\QQ \RR \cong \RR^{\r_1} \times \CC^{\r_2}. $$
\end{notation*}

Then $ \AA_K $ is a ring. The topology on $ \AA_K $ is generated by all open $ V \subset \U_{K, S} $ as $ S $ varies, and where $ \U_{K, S} $ has the product topology, so
$$ V = \prod_{v \in S} X_v \times \prod_{v \notin S} \OOO_{K_v}, $$
where $ S $ is finite, containing $ \V_{K, \infty} $, and $ X_v $ is open in $ K_v $. This means in particular that every $ \U_{K, S} \subset \AA_K $ is open, so
$$ \U_{K, \emptyset} = K_\infty \times \prod_{v \in \V_{K, \f}} \OOO_v = K_\infty \times \widehat{\OOO_K}, $$
where $ \widehat{\OOO_K} $ is the profinite completion, is open and has the product topology. This completely determines the topology on $ \AA_K $. See example sheet $ 1 $ exercise $ 1 $(ii).

\begin{example*}
Let $ K = \QQ $. Then
$$ \AA_\QQ = \RR \times \cbr{\br{x_p}_p \in \prod_{p < \infty} \QQ_p \st \text{for all but finitely many} \ p, \ x_p \in \ZZ_p}. $$
So, letting $ m \in \ZZ_{> 0} $ be the product of the denominators $ p^i $ of $ x_p $ see that $ m\br{x_p}_p \in \prod_{p < \infty} \ZZ_p = \widehat{\ZZ} $, that is $ \br{x_p}_p \in \br{1 / m}\widehat{\ZZ} \subset \prod_p \QQ_p $. Let \footnote{Exercise: easy}
$$ \widehat{\QQ} = \bigcup_{m \ge 1} \dfrac{1}{m}\widehat{\ZZ} \cong \widehat{\ZZ} \otimes_\ZZ \QQ. $$
Then $ \AA_\QQ = \RR \times \widehat{\QQ} $.
\end{example*}

\pagebreak

\begin{proposition}
$ \AA_K $ is Hausdorff and locally compact, so every point has a compact neighbourhood.
\end{proposition}

\begin{proof}
$ \U_{K, \emptyset} $ is Hausdorff, and is locally compact, since $ K_\infty $ is locally compact and $ \widehat{\OOO_K} $ is compact, and it is an open neighbourhood of zero. So by translation, $ \AA_K $ is Hausdorff and locally compact.
\end{proof}

There is a diagonal embedding $ K \hookrightarrow \AA_K $.

\begin{proposition}
$ K $ is discrete in $ \AA_K $.
\end{proposition}

\begin{proof}
Find a neighbourhood of zero containing only $ 0 \in K $. Let
$$ U = \cbr{x = \br{x_v} \in \AA_K \st \begin{array}{l} \forall v \in \V_{K, \f}, \ \abs{x_v}_v \le 1 \\ \forall v \in \V_{K, \infty}, \ \abs{x_v}_v < 1 \end{array}}. $$
Then $ U \subset \AA_K $ is open. If $ x \in K \cap U $, then $ \abs{x_v}_v \le 1 $ for all $ v \nmid \infty $ implies that $ x \in \OOO_K $, and $ \abs{x_v}_v < 1 $ for all $ v \mid \infty $ implies that $ \abs{\N_{K / \QQ}\br{x}} < 1 $, that is $ x = 0 $. So zero is isolated in $ K $. Thus $ K $ is discrete.
\end{proof}

\lecture{9}{Tuesday}{09/02/21}

Let $ L / K $ be an extension of number fields. For all $ v \in \V_K $, $ K_v \hookrightarrow \prod_{w \mid v} L_w $ induces an inclusion of rings $ \AA_K \hookrightarrow \AA_L $ visibly continuous.

\begin{proposition}
\label{prop:6.3}
Let $ \br{a_1, \dots, a_n} $ be a $ K $-basis for $ L $. Consider
$$
\begin{array}{ccccc}
\AA_K^n & \xrightarrow{f} & \AA_K \otimes_K L & \xrightarrow{g} & \AA_L \\
\displaystyle\br{x^{\br{i}}}_{1 \le i \le n} & \longmapsto & \displaystyle\sum_i x^{\br{i}} \otimes a_i & \longmapsto & \displaystyle\sum_i a_ix^{\br{i}}
\end{array},
$$
viewing $ x^{\br{i}} \in \AA_K \hookrightarrow \AA_L $ as above. Then $ g $ is a ring isomorphism, $ f $ is an $ \AA_K $-module isomorphism, and $ g \circ f $ is a homeomorphism. This then defines a unique topology on $ \AA_K \otimes_K L $ such that $ g $ is an isomorphism of topological rings.
\end{proposition}

\begin{proof}
Since $ L = \bigoplus_i Ka_i \cong K^n $, $ f $ is an $ \AA_K $-module isomorphism. By definition, $ g $ is a ring homomorphism. So it suffices to prove $ g \circ f $ is bijective, and that it maps $ X^n = \br{K_\infty \times \widehat{\OOO_K}}^n $ homeomorphically to an open subgroup of $ \AA_L $. Note that multiplication by any $ x \in K^\times $ is a self-homeomorphism of $ \AA_K $ with itself, since the inverse is multiplication by $ x^{-1} $. Similarly for $ \AA_L $. So may replace $ \br{a_i} $ by non-zero $ K $-multiples, so without loss of generality, $ a_i \in \OOO_L $. Let
$$ S = \cbr{v \in \V_{K, \f} \st v\br{\br{\OOO_L : \sum_i a_i\OOO_K}} > 0} $$
be a finite subset of $ \V_{K, \f} $. Then for all $ v \in \V_{K, \f} \setminus S $,
$$ \br{a_i} : \OOO_{K_v}^n \xrightarrow{\sim} \OOO_{K_v} \otimes_{\OOO_K} \OOO_L \cong \prod_{w \mid v} \OOO_{L_w}, $$
and for all $ v \in S $, $ \sum_i a_i\OOO_{K_v} = M_v $ is an open $ \OOO_{K_v} $-submodule of $ \prod_{w \mid v} \OOO_{L_w} $. Then
$$ g \circ f : \br{K_\infty \times \widehat{\OOO_K}}^n \xrightarrow{\sim} L_\infty \times \prod_{v \notin S} \prod_{w \mid v} \OOO_{L_w} \times \prod_{v \in S} M_v $$
is a homeomorphism onto an open subgroup in $ \AA_L $. Moreover, for any finite $ S' \supset S \cup \V_{K, \infty} $,
$$ g \circ f : \U_{K, S'} = \br{\prod_{v \in S'} K_v \times \prod_{v \notin S'} \OOO_{K_v}}^n \xrightarrow{\sim} \prod_{w \mid v \in S'} L_w \times \prod_{w \mid v \notin S'} \OOO_{L_w}. $$
So $ g \circ f $ is bijective.
\end{proof}

In particular, $ \AA_K = \AA_\QQ \otimes_\QQ K $.

\pagebreak

\begin{corollary}
$ \AA_L $ is a free $ \AA_K $-module of rank $ \sbr{L : K} $, and the diagram
$$
\begin{tikzcd}
\displaystyle\prod_{w \mid v} L_w \arrow[hookrightarrow]{r} \arrow{d}{\sum_w \Tr_{L_w / K_v}} & \AA_L \arrow{d}{\Tr_{\AA_L / \AA_K}} & \AA_K \otimes_K L \arrow{l}[swap]{\sim} \arrow{d}{\id \otimes \Tr_{L / K}} & L \arrow[hookrightarrow]{l} \arrow{d}{\Tr_{L / K}} \\
K_v \arrow[hookrightarrow]{r} & \AA_K & \AA_K \otimes_K K \arrow{l}{\sim} & K \arrow[hookrightarrow]{l}
\end{tikzcd}
$$
commutes, where the left hand inclusions are
$$ \br{x_w}_{w \mid v} \mapsto \br{y_w}, \qquad y_w =
\begin{cases}
x_w & w \mid v \\
0 & \text{otherwise}
\end{cases}.
$$
\end{corollary}

\begin{proof}
Exercise. \footnote{Exercise}
\end{proof}

\begin{theorem}
$ \AA_K / K $ is compact Hausdorff.
\end{theorem}

\begin{proof}
Since $ K $ is closed in $ \AA_K $ and $ \AA_K $ is Hausdorff, $ \AA_K / K $ is Hausdorff. By \ref{prop:6.3}, $ \AA_K / K \cong \br{\AA_\QQ / \QQ}^{\sbr{K : \QQ}} $ as topological groups, so may assume $ K = \QQ $. Let $ X = \sbr{0, 1} \times \widehat{\ZZ} \subset \AA_\QQ $. Then $ X $ is compact. So it is enough to show that $ X + \QQ = \AA_\QQ $, as then $ X \twoheadrightarrow \AA_\QQ / \QQ $. Let $ x = \br{x_p}_{p \le \infty} \in \AA_\QQ $. Let
$$ S = \cbr{p < \infty \st x_p \notin \ZZ_p} $$
be a finite set. There exists $ r_p \in \ZZ\sbr{1 / p} $ such that $ x_p - r_p \in \ZZ_p $ for all $ p \in S $. Let $ r = \sum_{p \in S} r_p \in \QQ $. For all $ p < \infty $, $ x_p - r \in \ZZ_p $, that is $ x - r \in \RR \times \widehat{\ZZ} $, and then for suitable $ m \in \ZZ $, $ x - \br{r + m} \in \sbr{0, 1} \times \widehat{\ZZ} $.
\end{proof}

From \ref{prop:6.3} also get $ \AA_K = K_\infty \times \widehat{K} $ where
$$ \widehat{K} = \widehat{\OOO_K} \otimes_\ZZ \QQ = \widehat{\OOO_K} \otimes_{\OOO_K} K, $$
where $ \widehat{\OOO_K} \cong \prod_\ppp \widehat{\OOO_{K, \ppp}} = \prod_{v \nmid \infty} \OOO_{K_v} $ is the profinite completion of $ \OOO_K $.

\subsection{Ideles}

\begin{definition*}
The \textbf{idele group} of $ K $ is the group of units of $ \AA_K $,
$$ \JJJ_K = \AA_K^\times = \cbr{\br{x_v} \in \prod_{v \in \V_K} K_v^\times \st \text{for all but finitely many finite} \ v, \ x_v \in \OOO_v^\times} = \bigcup_{\text{finite} \ S \subset \V_{K, \f}} \JJJ_{K, S}, $$
where
$$ \JJJ_{K, S} = K_\infty^\times \times \prod_{v \in S} K_v^\times \times \prod_{v \in \V_{K, \f} \setminus S} \OOO_v^\times. $$
\end{definition*}

The topology on $ \JJJ_K $ is generated by open subsets of $ \JJJ_{K, S} $, as $ S $ varies, and $ \JJJ_{K, S} $ is given the product topology. In particular, $ K_\infty^\times \times \prod_{v \nmid \infty} \OOO_v^\times $ is an open subgroup, and has the product topology.

\begin{remark*}
$ \JJJ_K \hookrightarrow \AA_K $ is continuous, by the definitions, but is not a homeomorphism onto its image, since $ x \mapsto x^{-1} $ on $ \AA_K^\times $ is not continuous for the $ \AA_K $-topology, by example sheet $ 1 $ exercise $ 8 $, but
$$ \function{\JJJ_K}{\AA_K \times \AA_K}{x}{\br{x, x^{-1}}} $$
is a homeomorphism of $ \JJJ_K $ onto the closed subset $ \cbr{xy = 1} \subset \AA_K^2 $. In geometry, $ \GL_n K \subset \AA^{n^2} $ and
$$ \function{\GL_n K}{\AA^{n^2 + 1}}{\br{a_{ij}}}{\br{a_{ij}, \det \br{a_{ij}}^{-1}}} $$
has closed image.
\end{remark*}

Then $ K^\times \hookrightarrow \JJJ_K $ since if $ x \in K^\times $ then $ \abs{x}_v = 1 $ for all but finitely many $ v $. The image is discrete, since $ \JJJ_K \hookrightarrow \AA_K $ is continuous and $ K \subset \AA_K $ is discrete.

\pagebreak

\lecture{10}{Thursday}{11/02/21}

\begin{definition*}
The \textbf{idele class group} of $ K $ is
$$ \CCC_K = \JJJ_K / K^\times. $$
\end{definition*}

This is a Hausdorff and locally compact topological group. There are two important homomorphisms.

\begin{definition*}
Let $ x = \br{x_v} \in \JJJ_K $. Then for all $ v $, $ \abs{x_v}_v \ne 0 $, and for all but finitely many $ v $, $ \abs{x_v}_v = 1 $. So can define the \textbf{idele norm} homomorphism
$$ \function[\abs{\cdot}_\AA]{\JJJ_K}{\RR_{> 0}}{\br{x_v}}{\prod_{v \in \V_K} \abs{x_v}_v}, $$
\end{definition*}

This is continuous, since the restriction to $ \JJJ_{K, S} $ is $ \prod_v \abs{\cdot}_v : \JJJ_{K, S} \to \prod_{v \in S \cup \V_{K, \infty}} K_v^\times \to \RR_{> 0} $. Clearly $ \abs{\cdot}_\AA $ is surjective, since $ K_\infty^\times \subset \JJJ_K $. A key fact is that for all $ x \in K^\times $, $ \abs{x}_\AA = 1 $ by the product formula, so $ \abs{\cdot}_\AA : \JJJ_K \to \CCC_K \to \RR_{> 0} $.

\begin{definition*}
Let
$$ \JJJ_K^1 = \cbr{x \in \JJJ_K \st \abs{x}_\AA = 1}, \qquad \CCC_K^1 = \JJJ_K^1 / K^\times. $$
\end{definition*}

\begin{proposition}
$$ \JJJ_K \cong \JJJ_K^1 \times \RR_{> 0}, \qquad \CCC_K \cong \CCC_K^1 \times \RR_{> 0}. $$
\end{proposition}

\begin{proof}
Have $ \abs{\cdot}_\AA : \JJJ_K \twoheadrightarrow \RR_{> 0} $. Consider
$$ \function[\i]{\RR_{> 0}}{K_\infty^\times \subset \JJJ_K}{x}{\br{x^{\tfrac{1}{n}}}_{v \mid \infty}}. $$
Because $ \abs{x}_v $ is the Euclidean AV if $ v $ is real and the square of modulus if $ v $ is complex, this homomorphism is a right inverse to $ \abs{\cdot}_\AA $. So defines a splitting $ \JJJ_K \cong \JJJ_K^1 \times \RR_{> 0} $. As $ \i\br{\RR_{> 0}} \cap K^\times = \cbr{1} $, also have $ \CCC_K \cong \CCC_K^1 \times \RR_{> 0} $.
\end{proof}

Recall $ \ppp_v $ is the prime ideal corresponding to a finite place $ v $. Write $ v $ also for the corresponding normalised discrete valuation.

\begin{definition*}
Let
$$ \I\br{K} = \cbr{\text{group of fractional ideals of} \ K} \cong \cbr{\text{free abelian group generated by} \ \V_{K, \f}}. $$
The \textbf{content map} is
$$ \function[\c]{\JJJ_K}{\I\br{K}}{\br{x_v}}{\prod_{v \in \V_{K, \f}} \ppp_v^{v\br{x_v}}}. $$
\end{definition*}

This is a continuous homomorphism, for the discrete topology on $ \I\br{K} $, since $ \ker \c = \JJJ_{K, \emptyset} = K_\infty^\times \times \prod_{v \nmid \infty} \OOO_v^\times $ is open. If $ x \in K^\times $ then $ \c\br{x} $ is the principal fractional ideal $ \abr{x} $. So $ \c $ descends to a homomorphism
$$ \c : \CCC_K = \JJJ_K / K^\times \to \Cl\br{K} = \I\br{K} / \P\br{K}, $$
where $ \P\br{K} $ is the group of principal fractional ideals. The image of the inclusion $ K^\times \hookrightarrow \JJJ_K $ is called the \textbf{subgroup of principal ideles}. Then $ \c $ is clearly surjective, since $ v : K_v^\times \twoheadrightarrow \ZZ $. So $ \CCC_K \twoheadrightarrow \Cl\br{K} $. As $ \c \circ \i : \RR_{> 0} \to \Cl\br{K} $ is zero, have a continuous surjection $ \CCC_K^1 \twoheadrightarrow \Cl\br{K} $. Now prove that $ \CCC_K^1 = \JJJ_K^1 / K^\times $ is compact. A corollary is that $ \Cl\br{K} $ is finite, since compact and discrete. The following is a variant.

\begin{definition*}
Let $ S \subset \V_{K, \f} $ be a finite subset, and let
$$ \I^S\br{K} = \cbr{\text{fractional ideals prime to} \ S} = \cbr{I \st \forall v \in S, \ v\br{I} = 0}. $$
Define
$$ \function[\c^S]{\JJJ_K}{\I^S\br{K}}{\br{x_v}}{\prod_{v \in \V_{K, \f} \setminus S} \ppp_v^{v\br{x_v}}}. $$
\end{definition*}

This will be useful later on.

\pagebreak

\section{Geometry of numbers}

\subsection{Minkowski's theorem}

Classically, embed
$$ \sigma : K \hookrightarrow K_\infty = \prod_{v \mid \infty} K_v \cong \RR^{\r_1} \times \CC^{\r_2} \cong \RR^n, $$
and study the image $ \sigma\br{I} \subset \RR^n $ for $ I $ a fractional ideal.

\begin{definition*}
Let $ U $ be a finite-dimensional real vector space. A \textbf{lattice} $ \Lambda \subset U $ is a discrete subgroup such that $ U / \Lambda $ is compact.
\end{definition*}

\begin{proposition}
A subgroup $ \Lambda \subset U $ is a lattice if and only if $ \Lambda = \bigoplus_{1 \le i \le n} \ZZ e_i $, where $ \br{e_i} $ is an $ \RR $-basis for $ U $ where $ n = \dim_\RR U $.
\end{proposition}

\begin{proof}
Example sheet.
\end{proof}

\begin{theorem}[Minkowski's theorem]
\label{thm:7.2}
Let $ \Lambda \subset \RR^n $ be a lattice, and let $ \mu_\Lambda = \meas\br{\RR^n / \Lambda} $, the \textbf{covolume} of $ \Lambda $. Let $ X \subset \RR^n $ be a compact subset, which is
\begin{itemize}
\item convex, that is if $ t \in \sbr{0, 1} $ and $ x, y \in X $ then $ tx + \br{1 - t}y \in X $, and
\item symmetric about the origin, that is if $ x \in X $ then $ -x \in X $.
\end{itemize}
If $ \meas\br{X} > 2^n\mu_\Lambda $, then $ X \cap \Lambda \ne \cbr{0} $.
\end{theorem}

\begin{remark*}
$ \RR^n $ has a Lebesgue measure, and $ \meas\br{X} $ is the measure of $ X $. The Lebesgue measure defines a measure on $ \RR^n / \Lambda $, and $ \mu_\Lambda $ is the measure of $ \RR^n / \Lambda $. Naively, if $ \Lambda = \bigoplus_i \ZZ e_i $ for $ \br{e_i} $ linearly independent over $ \RR $ and $ \PPP = \cbr{\sum_i x_ie_i \st 0 \le x_i < 1} $, then $ \PPP $ is a set of coset representatives for $ \Lambda \subset \RR^n $, and $ \mu_\Lambda = \meas\br{\PPP} = \abs{\det \br{e_{ij}}} $, which is independent of the basis.
\end{remark*}

\begin{proof}
Let $ \pi : \RR^n \to \RR^n / 2\Lambda $. Then
$$ \meas\br{\pi\br{X}} \le \meas\br{\RR^n / 2\Lambda} = 2^n\meas\br{\RR^n / \Lambda} < \meas\br{X}. $$
So $ X \to \pi\br{X} $ is not one-to-one, so there exists $ x \ne y $ in $ X $ such that $ x - y = 2\lambda \in 2\Lambda $. Then $ 0 \ne \lambda = \br{x - y} / 2 = \tfrac{1}{2}x + \tfrac{1}{2}\br{-y} \in X $ as $ -y \in X $, by symmetry, and $ X $ is convex.
\end{proof}

\begin{theorem}
\label{thm:7.3}
There exists a constant $ \r_K > 0 $ such that, if $ \br{d_v}_{v \in K} $ are positive reals with
\begin{itemize}
\item $ d_v \in \abs{K_v^\times}_v = \cbr{\abs{x}_v \st x \in K_v^\times} \subset \RR_{> 0} $ for all $ v $,
\item $ d_v = 1 $ for all but finitely many $ v $, and
\item $ \prod_{v \in \V_K} d_v > \r_K $,
\end{itemize}
then $ \cbr{x \in K \st \forall v, \ \abs{x}_v \le d_v} \ne \cbr{0} $.
\end{theorem}

\begin{proof}
For $ v \nmid \infty $, write $ d_v = \q_v^{-n_v} $ for $ n_v \in \ZZ $. Let
$$ I = \cbr{x \in K \st \forall v \nmid \infty, \ \abs{x}_v \le d_v} = \prod_v \ppp_v^{n_v} $$
be a fractional ideal of $ K $. Then $ mI \subset \OOO_K $ for $ m > 0 $, so
\begin{equation}
\label{eq:4}
\mu_{\sigma\br{I}} = m^{-n}\mu_{\sigma\br{mI}} = m^{-n}\mu_{\sigma\br{\OOO_K}}\br{\sigma\br{\OOO_K} : \sigma\br{mI}} = m^{-n}\mu_{\sigma\br{\OOO_K}}\N\br{mI} = \mu_{\sigma\br{\OOO_K}}\prod_v \q_v^{n_v},
\end{equation}
and $ \sigma\br{I} $ is a lattice in $ \RR^n $, by the non-vanishing of the discriminant. Let
$$ X = \cbr{x \in \prod_{v \in \infty} K_v \cong \RR^n \st \forall v, \ \abs{x_v}_v \le d_v} = \prod_{v \ \text{real}} \sbr{-d_v, d_v} \times \prod_{v \ \text{complex}} \cbr{\abs{z}^2 \le d_v} \subset K_\infty \cong \RR^{\r_1} \times \CC^{\r_2}. $$

\pagebreak

This is convex, compact, symmetric, and
$$ \meas\br{X} = 2^{\r_1}\pi^{\r_2}\prod_{v \mid \infty} d_v > 2^n\prod_{v \nmid \infty} d_v^{-1}\mu_{\sigma\br{\OOO_K}} = 2^n\mu_{\sigma\br{I}}, $$
by $ \br{\ref{eq:4}} $, provided
$$ \prod_v d_v > \r_K = \br{\dfrac{4}{\pi}}^{\r_2}\mu_{\sigma\br{\OOO_K}} = \br{\dfrac{2}{\pi}}^{\r_2}\abs{\d_K}^{\tfrac{1}{2}}. $$
Then applying \ref{thm:7.2}, $ X \cap \sigma\br{I} \ne \cbr{0} $ and any $ x \in X \cap \sigma\br{I} $ has $ \abs{x}_v \le d_v $ for all $ v $.
\end{proof}

This is the translation of a classical result that if $ 0 \ne I $ is an ideal then there exists $ x \in I \setminus \cbr{0} $ such that $ \abs{\N_{K / \QQ}\br{x}} < \r_K\N\br{I} $.

\lecture{11}{Saturday}{13/02/21}

\begin{remark*}
Used Minkowski's theorem, with convex symmetric set $ X = \sbr{-d_v, d_v}^{\r_1} \times \cbr{\abs{z}^2 \le d_v}^{\r_2} $ and obtained $ \r_K = \br{4 / \pi}^{\r_2}\mu_{\sigma\br{\OOO_K}} $. Using better chosen $ X $, can get a better bound, the Minkowski bound $ \c_K $, which is useful for computation.
\end{remark*}

\subsection{Compactness of idele class group}

Recall $ K^\times \subset \JJJ_K^1 = \ker \br{\abs{\cdot}_\AA : \JJJ_K \to \RR_{> 0}} $ is discrete. Based on \ref{thm:7.3} and the following.

\begin{proposition}
\label{prop:7.5}
Let $ \rho_v > 0 $ for $ v \in \V_K $, with $ \rho_v = 1 $ for all but finitely many $ v $. Then
$$ X = \cbr{x \in \JJJ_K^1 \st \forall v, \ \abs{x_v}_v \le \rho_v} $$
is compact.
\end{proposition}

This is false for $ \JJJ_K $. Note that $ \abs{x_v}_v \le \rho_v $ for all $ v $ defines a compact subset of $ \AA_K $.

\begin{proof}
Let $ R = \prod_v \rho_v $, and let
$$ S = \V_{K, \infty} \cup \cbr{v \st \rho_v \ne 1} \cup \cbr{v \in \V_{K, \f} \st \q_v \le R} $$
be a finite set of places, since the last set is contained in $ \cbr{v \mid p \st p \le R} $, which is finite. If $ v \notin S $, and $ x \in X $, since $ \rho_v = 1 $,
$$ 1 \ge \abs{x_v}_v = \prod_{w \ne v} \abs{x_w}_w^{-1} \ge \prod_{w \ne v} \rho_w^{-1} = R^{-1}. $$
As $ \q_v > R $, this forces $ \abs{x_v}_v = 1 $. So $ X = X' \times \prod_{v \notin S} \OOO_v^\times $, where
$$ X' = \cbr{\br{x_v} \in \prod_{v \in S} K_v^\times \st \prod_{v \in S} \abs{x_v}_v = 1, \ \forall v \in S, \ \abs{x_v}_v \le \rho_v}, $$
which is a closed subset of
$$ X'' = \cbr{\br{x_v} \in \prod_{v \in S} K_v^\times \st \forall v \in S, \ \dfrac{\rho_v}{R} \le \abs{x_v}_v \le \rho_v}, $$
which is compact. So $ X' $ is compact, hence so is $ X $, since $ \prod_{v \notin S} \OOO_v^\times $ is compact.
\end{proof}

\begin{theorem}
\label{thm:7.4}
$ \CCC_K^1 = \JJJ_K^1 / K^\times $ is compact.
\end{theorem}

\begin{proof}
Let $ \r_K $ be as in \ref{thm:7.3}. Pick any $ y \in \JJJ_K $ with $ \abs{y}_\AA > \r_K $, and let
$$ X = \cbr{x \in \JJJ_K^1 \st \forall v \in \V_K, \ \abs{x_v}_v \le \abs{y_v}_v}, $$
which is compact by \ref{prop:7.5}. Show that
$$ \JJJ_K^1 = K^\times X = \cbr{ax \st a \in K^\times, \ x \in X}. $$
Let $ z \in \JJJ_K^1 $. Then $ \prod_v \abs{y_vz_v}_v = \abs{y}_\AA > \r_K $. So by \ref{thm:7.3}, there exists $ b \in K^\times $ such that for all $ v \in \V_K $, $ \abs{b}_v \le \abs{y_vz_v}_v $. Therefore $ bz^{-1} \in X $, that is $ z^{-1} \in b^{-1}X \subset K^\times X $.
\end{proof}

\pagebreak

\subsection{Finiteness of ideal class group and \texorpdfstring{$ S $}{S}-unit theorem}

The following are two corollaries.

\begin{corollary}
The ideal class group $ \Cl\br{K} $ is finite.
\end{corollary}

\begin{proof}
$ \CCC_K^1 \twoheadrightarrow \Cl\br{K} $ by the content map, which is continuous, so $ \Cl\br{K} $ is discrete and compact, therefore finite.
\end{proof}

\begin{corollary}[$ S $-unit theorem]
\label{cor:7.7}
Let $ S \subset \V_{K, \f} $ be finite, possibly empty, and let
$$ \OOO_{K, S} = \cbr{x \in K \st \forall v \in \V_{K, \f} \setminus S, \ \abs{x}_v \le 1} $$
be the \textbf{$ S $-integers} of $ K $, sometimes written $ \OOO_K\sbr{1 / S} $. Then
$$ \OOO_{K, S}^\times = \mu\br{K} \times \ZZ^{\r_1 + \r_2 - 1 + \#S}, $$
where $ \mu\br{K} = \cbr{\text{roots of unity in} \ K} $ is finite.
\end{corollary}

The case $ S = \emptyset $ is Dirichlet's unit theorem,
$$ \OOO_K^\times = \mu\br{K} \times \ZZ^{\r_1 + \r_2 - 1}. $$

\begin{proof}
\hfill
\begin{itemize}
\item First explain the proof for $ S = \emptyset $. Recall
$$ \JJJ_{K, \emptyset} = K_\infty^\times \times \prod_{v \nmid \infty} \OOO_v^\times \supset \JJJ_{K, \emptyset}^1 = K_\infty^{\times, 1} \times \prod_{v \nmid \infty} \OOO_v^\times, \qquad K_\infty^{\times, 1} = \cbr{\br{x_v} \in K_\infty^\times \st \prod_{v \mid \infty} \abs{x_v}_v = 1}. $$
Then $ \JJJ_{K, \emptyset} \cap K^\times = \JJJ_{K, \emptyset}^1 \cap K^\times = \OOO_K^\times $ is discrete in $ \JJJ_{K, \emptyset}^1 $ and by \ref{thm:7.4}, the closed $ \JJJ_{K, \emptyset}^1 / \OOO_K^\times \subset \CCC_K^1 $ is compact. Let
$$ \function[\lambda]{\JJJ_{K, \emptyset}}{\LLL_K = \prod_{v \mid \infty} \RR \cong \RR^{\r_1 + \r_2}}{\br{x_v}_v}{\br{\log \abs{x_v}_v}_v} $$
be the \textbf{logarithm map}, such that
$$ \lambda\br{\JJJ_{K, \emptyset}^1} \subset \LLL_K^0 = \cbr{\br{l_v} \in \LLL_K \st \sum_v l_v = 0}. $$
Then
$$ \ker \lambda = \cbr{\br{x_v} \in \JJJ_K \st \forall v, \ \abs{x_v}_v = 1} = \cbr{\pm 1}^{\r_1} \times \U\br{1}^{\r_2} \times \prod_{v \nmid \infty} \OOO_v^\times, \qquad \U\br{1} = \cbr{z \in \CC \st \abs{z} = 1} $$
is compact. So $ \ker \lambda \cap \OOO_K^\times $ is discrete and compact, hence finite. Obviously $ \mu\br{K} \subset \ker \lambda $, so $ \mu\br{K} $ is finite and equals $ \ker \lambda \cap \OOO_K^\times $. Next, show $ \lambda\br{\OOO_K^\times} \subset \LLL_K^0 \cong \RR^{\r_1 + \r_2 - 1} $ is a lattice. Then we get
$$ 1 \to \mu\br{K} \to \OOO_K^\times \to \lambda\br{\OOO_K^\times} \cong \ZZ^{\r_1 + \r_2 - 1} \to 0, $$
which gives \ref{cor:7.7}. Now
$$
\begin{tikzcd}
\JJJ_{K, \emptyset} \arrow[cong]{r} \arrow{d}[swap]{\lambda} & \displaystyle\prod_{v \mid \infty} \RR_{> 0} \times \ker \lambda \arrow{d}{\pi_1} \\
\LLL_K & \displaystyle\prod_{v \mid \infty} \RR_{> 0} \arrow{l}{\log}[swap]{\sim}
\end{tikzcd},
$$

\pagebreak

where $ \RR_{> 0} \hookrightarrow K_v^\times \subset \CC^\times $ for all $ v \mid \infty $. Hence $ \lambda $ has the property that for all compact $ Y $ in its target, $ \lambda^{-1}\br{Y} $ is compact, so $ \lambda $ is a proper map. A simple fact is if $ f : X \to Y $ is a continuous proper map of topological spaces, with $ Y $ locally compact and Hausdorff, then if $ Z \subset X $ is discrete then $ f\br{Z} $ is discrete. \footnote{Exercise: a hint is to take a compact neighbourhood $ V $ of some $ f\br{z} $ for $ z \in Z $ and use compactness of $ f^{-1}\br{V} $} Hence $ \lambda\br{\OOO_K^\times} \subset \LLL_K^0 $ is discrete. Finally,
$$ \lambda : \JJJ_{K, \emptyset}^1 / \OOO_K^\times \twoheadrightarrow \LLL_K^0 / \lambda\br{\OOO_K^\times}, $$
so $ \LLL_K^0 / \lambda\br{\OOO_K} $ is compact, by \ref{thm:7.4}. Thus $ \lambda\br{\OOO_K} $ is a lattice.

\lecture{12}{Tuesday}{16/02/21}

\item For the general case, the difference is mainly notational. Let $ S_\infty = S \cup \V_{K, \infty} $, so
$$ \JJJ_{K, S} = \prod_{v \in S_\infty} K_v^\times \times \prod_{v \notin S_\infty} \OOO_v^\times, \qquad \LLL_{K, S} = \prod_{v \mid \infty} \RR \times \prod_{v \in S} \log \q_v\ZZ \cong \RR^{\r_1 + \r_2} \times \ZZ^{\#S}. $$
Let
$$ \function[\lambda_S]{\JJJ_{K, S}}{\LLL_{K, S}}{\br{x_v}_v}{\br{\log \abs{x_v}_v}_{v \in S_\infty}} $$
be the \textbf{$ S $-logarithm map}, such that
$$ \lambda_S\br{\JJJ_{K, S}^1} \subset \LLL_{K, S}^0 = \cbr{\br{l_v} \in \LLL_{K, S} \st \sum_v l_v = 0}. $$
Note that $ \LLL_{K, S}^0 \cong \RR^{\r_1 + \r_2 - 1} \times \ZZ^{\#S} $ since
$$
\begin{tikzcd}
\LLL_{K, S}^0 \arrow[twoheadrightarrow]{r}{\pi_2} & \prod_{v \in S} \log \q_v\ZZ \arrow[cong]{d} \\
& \ZZ^{\#S} \arrow[dashed]{ul}
\end{tikzcd}
$$
is surjective with kernel $ \RR^{\r_1 + \r_2 - 1} $, so there exists a splitting as $ \ZZ^{\#S} $ is free. Then
$$ \ker \lambda_S \cong \cbr{\pm 1}^{\r_1} \times \U\br{1}^{\r_2} \times \prod_{v \in \V_{K, \f}} \OOO_v^\times, $$
as before, and
$$ \JJJ_{K, S} = \prod_{v \mid \infty} \RR_{> 0} \times \prod_{v \in S} \abr{\pi_v} \times \ker \lambda_S \cong \prod_{v \mid \infty} \RR_{> 0} \times \ZZ^{\#S} \times \ker \lambda_S, $$
where $ \pi_v \in K_v^\times $ such that $ v\br{\pi_v} = 1 $, so $ \lambda_S $ is proper and surjective, so $ \JJJ_{K, S} \cap K^\times = \JJJ_{K, S}^1 \cap K^\times = \OOO_{K, S}^\times $ is discrete and closed in $ \JJJ_{K, S}^1 $. As before, $ \ker \lambda_S \cap \OOO_{K, S}^\times = \mu\br{K} $, since it is discrete and compact, and $ \lambda_S\br{\OOO_{K, S}^\times} \subset \LLL_{K, S}^0 $ is discrete and cocompact. Then prove that if $ G \cong \RR^m \times \ZZ^{\#S} \supset H $ is a discrete and cocompact subgroup then $ H \cong \ZZ^{m + \#S} $. \footnote{Exercise} Then
$$ 1 \to \mu\br{K} \to \OOO_{K, S}^\times \to \lambda_S\br{\OOO_{K, S}^\times} \cong \ZZ^{\r_1 + \r_2 - 1 + \#S} \to 0, $$
and so done.
\end{itemize}
\end{proof}

Let $ T \subset \V_K $ be finite, not necessarily containing $ \V_{K, \infty} $. What can we say about the group
$$ \cbr{x \in K^\times \st \forall v \notin T, \ \abs{x}_v = 1}? $$
The answer is non-trivial and depends on $ K $. See example sheet.

\pagebreak

\subsection{Strong approximation theorem}

Earlier, weak approximation implies that $ K $ is dense in any finite product of $ K_v $'s. Also, $ K \hookrightarrow \AA_K $ is discrete.

\begin{theorem}[Strong approximation]
\label{thm:7.8}
Let $ T \subset \V_K $ be finite, and set
$$ \AA_K^T = \cbr{x = \br{x_v} \in \prod_{v \notin T} K_v \st \text{for all but finitely many} \ v, \ \abs{x_v}_v \le 1}, $$
so $ \AA_K = \prod_{v \in T} K_v \times \AA_K^T $, with the adelic topology. Then if $ T \ne \emptyset $, then $ K $ is dense in $ \AA_K^T $.
\end{theorem}

There are various ways to rewrite this.
\begin{itemize}
\item If $ T \ne \emptyset $, then $ K + \prod_{v \in T} K_v $ is dense in $ \AA_K $, where $ K \hookrightarrow \AA_K $ is the diagonal inclusion and $ K_v \subset \AA_K $ by
$$ y \mapsto \br{x_w}, \qquad x_w =
\begin{cases}
y & w = v \\
0 & w \ne v
\end{cases}.
$$
\end{itemize}
It is enough to prove \ref{thm:7.8} for $ T = \cbr{v_0} $. Will actually prove the following.
\begin{itemize}
\item Let $ S \subset \V_K $ be finite such that $ v_0 \notin S $, let $ y_v \in K_v $ for all $ v \in S $, and let $ \epsilon > 0 $. Then there exists $ x \in K $ such that
\begin{itemize}
\item for all $ v \in S $, $ \abs{x - y_v}_v \le \epsilon $, and
\item for all $ v \notin S $ such that $ v \ne v_0 $, $ \abs{x}_v \le 1 $.
\end{itemize}
\end{itemize}
Take $ y \in \AA_K $ with component $ y_v $ at $ v \in S $ and zero elsewhere. This is equivalent to strong approximation for $ T = \cbr{v_0} $, by definition of the topology.

\begin{proof}
Free to enlarge $ S $. Then by the proof of compactness of $ \AA_K / K $, there exists $ R > 0 $ such that if
$$ X = \cbr{\br{x_v} \in \AA_K \st \begin{array}{l} \forall v \in S, \ \abs{x_v}_v \le R \\ \forall v \notin S, \ \abs{x_v}_v \le 1 \end{array}}, $$
then $ X + K = \AA_K $. For example, assume $ S \supset \V_{K, \infty} $ and let $ \OOO_K = \bigoplus_i \ZZ e_i $, then $ \AA_K = \bigoplus_i \AA_\QQ e_i $ and $ \AA_\QQ = \sbr{0, 1} \times \widehat{\ZZ} + \QQ $. Claim that there exists $ z \in K \setminus \cbr{0} $ such that
$$ \abs{z}_v \le
\begin{cases}
\dfrac{\epsilon}{R} & v \in S \\
1 & v \notin S, \ v \ne v_0
\end{cases}.
$$
Apply Minkowski \ref{thm:7.3} with
\begin{itemize}
\item $ d_v = 1 $ for all $ v \notin S \cup \cbr{v_0} $,
\item $ d_v \le \epsilon / R $ for all $ v \in S $, and
\item $ d_{v_0} > \r_K\br{\prod_{v \in S} d_v}^{-1} $.
\end{itemize}
This defines a box in $ \AA_K $ whose intersection with $ K $ is not $ \cbr{0} $, since $ \prod_v d_v > \r_K $. Now write $ z^{-1}y = a + t $ for $ a \in X $ and $ t \in K $. Then $ x = zt = y - za $ has
$$ \abs{x - y_v}_v = \abs{zt - y_v}_v = \abs{za_v}_v \le
\begin{cases}
\dfrac{\epsilon}{R} \cdot R = \epsilon & v \in S \\
1 \cdot 1 = 1 & v \notin S, \ v \ne v_0
\end{cases},
$$
so done.
\end{proof}

In the special case $ T = \V_{K, \infty} $, $ \AA_K^T $ are the finite adeles. Then \ref{thm:7.8} says
$$ K \hookrightarrow \AA_K^T = \widehat{K} = \widehat{\OOO_K} \otimes_\ZZ \QQ $$
is dense, which is equivalent to the density of
$$ \OOO_K \hookrightarrow \widehat{\OOO_K} = \prod_{v \nmid \infty} \OOO_{K_v} = \prod_{v \nmid \infty} \varprojlim_r \OOO_K / \ppp_v^r \cong \varprojlim_{I \subset \OOO_K} \OOO_K / I, $$
by CRT. So strong approximation is a generalisation of CRT.

\pagebreak

\section{Idele class group and class field theory}

\lecture{13}{Thursday}{18/02/21}

Recall if $ L = \QQ\br{\zeta_m} $, then there is an isomorphism
$$ \function{\Gal\br{L / \QQ}}{\br{\ZZ / m\ZZ}^\times}{\sigma_p}{p \mod m}, \qquad p \nmid m, $$
given by the action on $ \zeta_m $. In particular, $ \sigma_p $ depends only on the congruence class of $ p \mod m $, which implies quadratic reciprocity. As $ \sigma_p $ determines the decomposition of $ \abr{p} $ in $ L $, since $ \f\br{v \mid p} = \ord \D_v = \ord \sigma_p $, this says that the decomposition of $ \abr{p} $ in $ L $ depends only on $ p \mod m $. A consequence is if $ g \in \Gal\br{L / \QQ} $, then there exist infinitely many $ p $ such that $ g = \sigma_p $, by Dirichlet's theorem on primes in arithmetic progressions. The following is a general problem. Let $ L / K $ be a Galois extension of number fields, and let $ v $ be a finite place of $ K $, unramified in $ L $. Then
$$ \Sigma_v = \cbr{\sigma_w \st w \in \V_{L, \f}, \ w \mid v} $$
is a conjugacy class in $ G = \Gal\br{L / K} $, and $ \Sigma_v $ describes the decomposition of $ v $ in $ L $.
\begin{itemize}
\item How does $ \Sigma_v $ depend on $ v $?
\item Can it be any conjugacy class in $ G $?
\end{itemize}

For the first question, do not know the answer for general $ L / K $. This is non-abelian class field theory or the Langlands programme. The second question is answered by the Chebotarev density theorem in the 1920s. Let $ C \subset G $ be a conjugacy class. Then there exist infinitely many $ v $ for which $ C = \Sigma_v $.

\begin{example*}
Let $ C = \cbr{1} $. There exist infinitely many $ v $ such that $ \Sigma_v = \cbr{1} $, that is such that $ v $ splits completely in $ L / K $.
\end{example*}

Class field theory answers the first question completely for $ L / K $ abelian.

\subsection{Artin reciprocity law}

\begin{theorem*}[Artin reciprocity law]
Let $ L / K $ be an abelian extension of number fields. Then there exists a unique continuous homomorphism
$$ \Art_{L / K} : \CCC_K = \JJJ_K / K^\times \to \Gal\br{L / K}, $$
such that for all unramified $ v \in \V_{K, \f} $,
$$ \function[\Art_{L / K}]{K_v^\times \hookrightarrow \CCC_K}{\Gal\br{L / K}}{x}{\sigma_v^{-v\br{x}}}. $$
Moreover, $ \Art_{L / K} $ is surjective with kernel $ K^\times \N_{L / K}\br{\JJJ_L} $.
\end{theorem*}

How does this generalise the cyclotomic theory? Since $ \CC^\times $ is connected, the only open subgroup is $ \CC^\times $, and the only open subgroups of $ \RR^\times $ are $ \RR^\times $ and $ \RR_{> 0} $. Then $ \ker \Art_{L / K} $ is open, so contains some $ K^\times U $, where
$$ U = \prod_{v \ \text{complex}} K_v^\times \times \prod_{v \ \text{real}} \RR_{> 0} \times \prod_{v \in S} U_v \times \prod_{v \in \V_{K, \f} \setminus S} \OOO_v^\times, \qquad U_v = \cbr{x \in \OOO_v^\times \st v\br{x - 1} \ge m_v}, \qquad m_v > 0, $$
where say $ S $ contains all ramified places. If $ w \notin S $ is unramified,
$$ \Art_{L / K} : K^\times\br{\dots, 1, 1, \pi_w^{-1}, 1, 1, \dots} = K^\times\br{\dots, \pi_w, \pi_w, 1, \pi_w, \pi_w, \dots} \mapsto \sigma_w, $$
where $ \pi_w \in \OOO_K $ such that $ w\br{\pi_w} = 1 $ is a uniformiser at $ w $. So if
\begin{enumerate}
\item $ \sigma\br{\pi_w} > 0 $ for all $ \sigma : K \hookrightarrow \RR $,
\item $ v\br{\pi_w - 1} \ge m_v $ for all $ v \in S $, and
\item $ \pi_w \in \OOO_v^\times $ for all $ v \notin S $ such that $ v \ne w $,
\end{enumerate}
which are congruence conditions on $ w $, then $ \sigma_w = 1 $. In particular, if $ \ppp_w = \abr{\pi_w} $ is principal, then $ 3 $ is automatic. So just a congruence condition on $ \pi_w $ modulo some ideal divisible only by primes in $ S $, and positivity.

\pagebreak

\begin{example*}
Let $ L = \QQ\br{\zeta_m} / K = \QQ $. Then
$$
\begin{tikzcd}
\br{\RR^\times \times \widehat{\QQ}^\times} / \QQ^\times \arrow[cong]{d} & \br{\RR^\times \times \widehat{\ZZ}^\times} / \cbr{\pm 1} \arrow{l}[swap]{\sim} \arrow[cong]{d} & \RR_{> 0} \times \widehat{\ZZ}^\times \arrow{l}[swap]{\sim} \arrow{r} \arrow{d} & \prod_{q \mid m} \ZZ_q^\times \arrow{d} \\
\JJJ_\QQ / \QQ^\times \arrow[dashed]{dr} & \JJJ_{\QQ, \emptyset} / \cbr{\pm 1} \arrow{l}{\sim} & \br{\ZZ / m\ZZ}^\times \arrow{dl}{\sim} & \prod_{q \mid m} \br{\ZZ_q / q\ZZ_q}^\times \arrow{l}{\sim} \\
& \Gal\br{L / \QQ} & &
\end{tikzcd}.
$$
Claim this is $ \Art_{L / \QQ} $. Let $ \QQ^\times\br{\dots, 1, 1, p^{-1}, 1, 1, \dots} = \QQ^\times\br{\dots, p, p, 1, p, p, \dots} \in \JJJ_\QQ / \QQ^\times $ for $ p \nmid m $. Then
$$
\begin{array}{ccccccc}
\JJJ_\QQ / \QQ^\times & \longleftarrow & \RR_{> 0} \times \widehat{\ZZ}^\times & \longrightarrow & \br{\ZZ / m\ZZ}^\times & \longrightarrow & \Gal\br{L / \QQ} \\
\QQ^\times\br{\dots, p, p, 1, p, p, \dots} & \longmapsfrom & \br{\dots, p, p, 1, p, p, \dots} & \longmapsto & p \mod m & \longmapsto & \sigma_p
\end{array}.
$$
So via $ \JJJ_\QQ / \QQ^\times \cong \RR_{> 0} \times \widehat{\ZZ}^\times $, $ \Art_{L / \QQ} $ is just the cyclotomic map.
\end{example*}

\subsection{Finite quotients of idele class group}

\begin{proposition}
\label{prop:8.1}
Let $ G $ be a discrete group.
\begin{enumerate}
\item Any continuous homomorphism $ \alpha : \CCC_K \to G $ has finite image.
\item There is a bijection
$$ \correspondence{\text{continuous homomorphisms} \\ \alpha : \JJJ_K \to G}{\text{families} \ \br{\alpha_v : K_v^\times \to G}_{v \in \V_K} \\ \text{such that} \ \alpha_v\br{\OOO_v^\times} = \cbr{1} \\ \text{for all but finitely many} \ v \in \V_{K, \f}}. $$
\end{enumerate}
\end{proposition}

\begin{notation*}
$ \alpha_v : K_v^\times \to G $ is \textbf{unramified} if $ \alpha_v\br{\OOO_v^\times} = \cbr{1} $. See local class field theory, where $ \OOO_v^\times $ corresponds to the inertia.
\end{notation*}

\begin{proof}
\hfill
\begin{enumerate}
\item $ \JJJ_K \cong \RR_{> 0} \times \JJJ_K^1 $, and $ \alpha\br{\RR_{> 0}} = \cbr{1} $ so $ \alpha\br{\CCC_K} = \alpha\br{\CCC_K^1} $, which is compact and discrete so finite.
\item The subgroup
$$ \bigoplus_v K_v^\times = \cbr{\br{x_v} \st x_v = 1 \ \text{for all but finitely many} \ v} \subset \JJJ_K $$
is dense, since $ \bigoplus_v \OOO_v^\times \subset \prod_v \OOO_v^\times $ is dense for the product topology. So a continuous $ \alpha : \JJJ_K \to G $ is determined by its restrictions $ \alpha_v = \eval{\alpha}_{K_v^\times} : K_v^\times \to G $. As $ \ker \alpha $ is open, $ \alpha_v\br{\OOO_v^\times} = \cbr{1} $ for all but finitely many $ v $. So have $ \cbr{\alpha} \hookrightarrow \cbr{\br{\alpha_v}_v} $. Conversely, if $ \br{\alpha_v : K_v^\times \to G}_v $ is such a family, then $ \alpha\br{\br{x_v}} = \prod_v \alpha_v\br{x_v} $ is a finite product for any $ \br{x_v} \in \JJJ_K $, as $ x_v \in \OOO_v^\times $ and $ \alpha_v\br{\OOO_v^\times} = \cbr{1} $ for all but finitely many $ v $, and defines a continuous homomorphism $ \alpha : \JJJ_K \to G $.
\end{enumerate}
\end{proof}

\lecture{14}{Saturday}{20/02/21}

\begin{proposition}
\label{prop:8.2}
Let $ \alpha, \alpha' : \CCC_K \to G $ be continuous homomorphisms, where $ G $ is finite, unramified at all $ v \in \V_{K, \f} \setminus S $, where $ S $ is finite. Then if $ \alpha_v = \alpha_v' $ for all $ v \notin S $ such that $ v $ is finite, that is $ \alpha_v\br{\pi_v} = \alpha_v'\br{\pi_v} $, have $ \alpha = \alpha' $.
\end{proposition}

\begin{proof}
Look at $ \alpha / \alpha' $, so without loss of generality $ \alpha' = 1 $. Then $ \alpha : \JJJ_K / K^\times \to G $ satisfies for all $ v \in \V_{K, \f} \setminus S $, $ \alpha_v = 1 $. Let $ w \in S_\infty = \V_{K, \infty} \cup S $ and $ y \in K_w^\times $. Then by weak approximation, for any $ \epsilon > 0 $, there exists $ x \in K^\times $ such that $ \abs{x - y}_w < \epsilon $ and $ \abs{x - 1}_v < \epsilon $ for all $ v \in S_\infty \setminus \cbr{w} $. Hence $ \alpha_v\br{x} = 1 $ for all $ v \in S_\infty \setminus \cbr{w} $, so $ \alpha_v\br{x} = 1 $ for all $ v \ne w $. Since $ \alpha\br{K^\times} = 1 $, $ \alpha_w\br{x} = 1 $, so $ \alpha_w\br{y} = 1 $. So $ \alpha_w = 1 $, so $ \alpha = 1 $.
\end{proof}

\pagebreak

\begin{definition*}
A \textbf{modulus} is a finite formal sum
$$ \mmm = \sum_{v \in \V_K} \m_v\br{v}, \qquad \m_v \ge 0. $$
The \textbf{support} and \textbf{finite support} of $ \mmm $ are
$$ \supp \mmm = \cbr{v \in \V_K \st \m_v > 0}, \qquad \supp_\f \mmm = \supp \mmm \cap \V_{K, \f}. $$
We may use also $ \mmm_\f = \sum_{v \in \V_{K, \f}} \m_v\br{v} $, the finite part of $ \mmm $, can think of as an ideal of $ \OOO_K $. Define
$$ \U_{K, \mmm} = \prod_{v \in \V_K} \U_v^{\m_v}, \qquad K_v^\times \supset \U_v^m =
\begin{cases}
\OOO_v^\times & v \in \V_{K, \f}, \ m = 0 \\
1 + \pi_v^m\OOO_v & v \in \V_{K, \f}, \ m > 0 \\
\RR^\times & v \ \text{real}, \ m = 0 \\
\RR_{> 0} & v \ \text{real}, \ m > 0 \\
\CC^\times & v \ \text{complex}
\end{cases}.
$$
\end{definition*}

Note that in the definition of the modulus, we may as well forget about $ v $ complex, and for $ v $ real, take $ \m_v \in \cbr{0, 1} $. Then $ \U_{K, \mmm} \subset \JJJ_K $ is an open subgroup, and every open subgroup of $ \JJJ_K $ contains some $ \U_{K, \mmm} $.

\begin{proposition}
$ \JJJ_K / K^\times\U_{K, \mmm} $ is finite.
\end{proposition}

\begin{proof}
$ \JJJ_K / K^\times \to \JJJ_K / K^\times\U_{K, \mmm} $ with discrete image, since $ \U_{K, \mmm} $ is open. So by \ref{prop:8.1}.$ 1 $, the image is finite.
\end{proof}

So every finite quotient of $ \CCC_K $ is a quotient of some $ \JJJ_K / K^\times\U_{K, \mmm} $.

\begin{definition*}
The \textbf{ray class group} of $ K $ modulo $ \mmm $ is
$$ \Cl_\mmm\br{K} = \JJJ_K / K^\times\U_{K, \mmm}. $$
\end{definition*}

\begin{example*}
If $ \mmm = 0 $, then $ \U_{K, \mmm} = \ker \c $, where $ \c : \JJJ_K \to \I\br{K} $ is the content map, and $ \Cl_\mmm\br{K} = \Cl\br{K} $.
\end{example*}

Now relate to ideals.

\begin{notation*}
Let $ x \in K^\times $. Write $ x \equiv 1 \mods \mmm $ if
\begin{itemize}
\item for all $ v \in \supp_\f \mmm $, $ v\br{x - 1} \ge \m_v $, and
\item for all real $ v \in \supp \mmm $, $ x \in \br{K_v^\times}^+ = \RR_{> 0} $.
\end{itemize}
Let
\begin{align*}
K_\mmm^\times & = \cbr{x \in K^\times \st x \equiv 1 \mods \mmm}, \\
\I_\mmm\br{K} & = \cbr{\text{fractional ideals prime to} \ \supp_\f \mmm} \cong \cbr{\text{free abelian group on} \ \V_{K, \f} \setminus \supp_\f \mmm}, \\
\P_\mmm\br{K} & = \cbr{x\OOO_K \st x \in K_\mmm^\times} \subset \I_\mmm\br{K}.
\end{align*}
\end{notation*}

\begin{theorem}
$$ \Cl_\mmm\br{K} \cong \I_\mmm\br{K} / \P_\mmm\br{K}. $$
\end{theorem}

\begin{example*}
Assume $ K $ has real places, and let $ \mmm = \sum_{v \ \text{real}} \br{v} $. Then $ \I_\mmm\br{K} = \I\br{K} $ and $ \P_\mmm\br{K} $ is the group of principal fractional ideals $ x\OOO_K $ where $ x $ is \textbf{totally positive}, that is for all $ \sigma : K \hookrightarrow \RR $, $ \sigma\br{x} > 0 $. Then $ \Cl_\mmm\br{K} $ is called the \textbf{narrow ideal class group} of $ K $, also written $ \Cl^+\br{K} $. Obviously $ \Cl^+\br{K} \twoheadrightarrow \Cl\br{K} $ with kernel killed by two.
\end{example*}

Precisely is the following.

\begin{theorem}
Let $ S \subset \V_{K, \f} $ be finite, containing $ \supp_\f \mmm $. Then there exists a unique continuous homomorphism
$$ \alpha = \br{\alpha_v} : \JJJ_K / K^\times \to \I_\mmm\br{K} / \P_\mmm\br{K}, $$
such that for all $ v \in \V_{K, \f} \setminus S $, $ \alpha_v\br{\OOO_v^\times} = \cbr{1} $ and $ \alpha_v\br{\pi_v} \in \ppp_v^{-1} $. Moreover, $ \alpha $ induces an isomorphism
$$ \JJJ_K / K^\times\U_{K, \mmm} \xrightarrow{\sim} \I_\mmm\br{K} / \P_\mmm\br{K}. $$
\end{theorem}

\pagebreak

\begin{proof}
By \ref{prop:8.2}, $ \alpha $ is unique. For existence, let
$$ \JJJ_{K, \mmm} = \cbr{\br{x_v} \in \JJJ_K \st \forall v \in \supp \mmm, \ x_v \in \U_v^{\m_v}}, $$
the open subgroup generated by $ \U_{K, \mmm} $ and $ \cbr{K_v^\times \st v \notin \supp \mmm} $. Then by weak approximation, $ K^\times\JJJ_{K, \mmm} = \JJJ_K $, and by definition, $ K_\mmm^\times = K^\times \cap \JJJ_{K, \mmm} $, so
$$ \iota : \JJJ_K / K^\times\U_{K, \mmm} \xleftarrow{\sim} \JJJ_{K, \mmm} / K_\mmm^\times\U_{K, \mmm}. $$
Also, there is an isomorphism
$$ \function[\c^S]{\JJJ_{K, \mmm} / \U_{K, \mmm}}{\I_\mmm\br{K}}{\br{x_v}}{\prod_{v \in \V_{K, \f}, \ v \notin \supp_\f \mmm} \ppp_v^{v\br{x_v}}}. $$
Then
$$ \JJJ_K / K^\times\U_{K, \mmm} \xleftarrow{\iota} \JJJ_{K, \mmm} / K_\mmm^\times\U_{K, \mmm} \xrightarrow{\c^S} \I_\mmm\br{K} / \P_\mmm\br{K}, $$
and this is the map $ x \mapsto \alpha\br{x^{-1}} $.
\end{proof}

\begin{remark*}
The isomorphism $ \JJJ_K / K^\times\U_{K, \mmm} \to \I_\mmm\br{K} / \P_\mmm\br{K} $ is not induced by the $ S $-content map $ \JJJ_K \to \I_\mmm\br{K} $ but only on the subgroup $ \JJJ_{K, \mmm} $. Fr\"ohlich called this the \textbf{fundamental mistake of class field theory}.
\end{remark*}

\begin{example*}
Let $ K = \QQ $, let $ m > 1 $, and let $ \mmm = \br{m}\infty = \sum_{p \mid m} \v_p\br{m}\br{p} + \br{\infty} $. If $ I \in \I_\mmm\br{\QQ} $, then $ I = \br{a / b}\ZZ $ for unique positive coprime $ a, b \in \ZZ $ with $ \br{ab, m} = 1 $. Set
$$ \function[\Theta]{\I_\mmm\br{\QQ}}{\br{\ZZ / m\ZZ}^\times}{I}{\dfrac{a}{b} \mod m}. $$
This clearly defines an isomorphism such that
$$
\begin{tikzcd}
p\ZZ \in \I_\mmm\br{\QQ} / \P_\mmm\br{\QQ} \arrow{r}{\Theta}[swap]{\sim} & \br{\ZZ / m\ZZ}^\times \ni p \mod m \\
\QQ^\times\br{\dots, 1, 1, p^{-1}, 1, 1, \dots} \in \JJJ_\QQ / \QQ^\times \arrow{u}{\alpha} \arrow{r}{\sim} & \RR_{> 0} \times \widehat{\ZZ}^\times \arrow{u} \ni \br{\dots, p, p, 1, p, p, \dots}
\end{tikzcd}
$$
commutes.
\end{example*}

This is the reason for using $ \ppp_v^{-1} $, and $ \sigma_v^{-1} $ in the reciprocity law, since it means that for $ \QQ\br{\zeta_m} / \QQ $, recover the usual map $ \Gal\br{\QQ\br{\zeta_m} / \QQ} \cong \br{\ZZ / m\ZZ}^\times $. Older treatments of class field theory use $ \sigma_v $ and end up with the inverse of the usual map. Another reason is that the inverse $ \Fr_v = \F_v = \sigma_v^{-1} $, the so-called \textbf{geometric Frobenius}, is what occurs naturally in algebraic geometry. The modern normalisation of class field theory maps a uniformiser at an unramified $ v $ to the geometric Frobenius $ \sigma_v^{-1} $.

\subsection{Uniqueness}

\lecture{15}{Tuesday}{23/02/21}

By \ref{prop:8.2}, $ \Art_{L / K} $ is unique. A consequence is if $ L' / K' $ is an abelian extension, and have isomorphisms
$$
\begin{tikzcd}
L \arrow{r}{\widetilde{\tau}}[swap]{\sim} & L' \\
K \arrow[hookrightarrow]{u} \arrow{r}{\sim}[swap]{\tau} & K' \arrow[hookrightarrow]{u}
\end{tikzcd},
$$
then get isomorphisms
$$ \function[\tau]{\Gal\br{L / K}}{\Gal\br{L' / K'}}{g}{\widetilde{\tau} \circ g \circ \widetilde{\tau}^{-1}}. $$

\pagebreak

As extensions are abelian, any other $ \widetilde{\tau}' $ with $ \eval{\widetilde{\tau}'}_K = \tau $ is $ \widetilde{\tau}' = \widetilde{\tau} \circ h $ for $ h \in \Gal\br{L / K} $, so $ \widetilde{\tau}' \circ g \circ \widetilde{\tau}'^{-1} = \widetilde{\tau} \circ h \circ g \circ h^{-1} \circ \widetilde{\tau}^{-1} = \widetilde{\tau} \circ g \circ \widetilde{\tau}^{-1} $. So this isomorphism depends only on $ \tau $. Then
$$
\begin{tikzcd}
\CCC_K \arrow{r}{\Art_{L / K}} \arrow{d}{\sim}[swap]{\tau} & \Gal\br{L / K} \arrow{d}{\tau}[swap]{\sim} \\
\CCC_{K'} \arrow{r}[swap]{\Art_{L' / K'}} & \Gal\br{L' / K'}
\end{tikzcd}
$$
commutes, by uniqueness. This sort of argument is often called \textbf{transport of structure}.

\begin{example*}
Suppose $ L / K / F $ is Galois such that $ L / K $ is abelian and $ K / F $ is Galois. Take $ \tau = g \in \Gal\br{K / F} $. As $ L / K $ is abelian, $ \Gal\br{K / F} $ acts by conjugation on $ \Gal\br{L / K} $. Let $ K = K' $ and $ L = L' $. Then
\begin{equation}
\label{eq:5}
\Art_{L / K}\br{gx} = g \circ \Art_{L / K}\br{x} \circ g^{-1}, \qquad g \in \Gal\br{K / F}, \qquad x \in \CCC_K.
\end{equation}
\end{example*}

\subsection{Norms}

\begin{proposition}
Suppose $ L / K $ and $ L' / K' $ are abelian such that $ L \subset L' $ and $ K \subset K' $. Then
$$
\begin{tikzcd}
\Gal\br{L' / K'} \arrow{r}{g \mapsto \eval{g}_L} & \Gal\br{L / K} \\
\CCC_{K'} \arrow{u}{\Art_{L' / K'}} \arrow{r}[swap]{\N_{K' / K}} & \CCC_K \arrow{u}[swap]{\Art_{L / K}}
\end{tikzcd}
$$
commutes.
\end{proposition}

\begin{proof}
It is enough to check for $ \pi_w \in K_w'^\times \subset \CCC_{K'} $ for $ w $ outside a finite set. Assume $ w $ is unramified in $ L' / K' $ such that $ w \mid v \in \V_{K, \f} $ where $ v $ is unramified in $ L / K $. If $ \sigma_w \in \D_w \subset \Gal\br{L' / K'} $, then
$$ \eval{\sigma_w}_L = \eval{\br{x \mapsto x^{\q_w}}}_L = \br{x \mapsto x^{\q_v}}^{\f\br{w \mid v}} = \sigma_v^{\f\br{w \mid v}}. $$
If $ \pi_w \in K_w'^\times $ is a uniformiser, then
$$ \N_{K_w' / K_v}\br{\pi_w} = u\pi_v^{\f\br{w \mid v}}, \qquad u \in \OOO_{K_v}^\times, $$
since $ \pi_v^{\sbr{K_w' : K_v}} = \N_{K_w' / K_v}\br{\pi_v} $ and $ \pi_v = u\pi_w^{\e\br{w \mid v}} $.
\end{proof}

\begin{example*}
A special case is $ K' = L = L' $. Then $ 1 = \Art_{L / L}\br{x} = \Art_{L / K}\br{\N_{L / K}\br{x}} $ for $ x \in \JJJ_L $, so
$$ \N_{L / K}\br{\JJJ_L} \subset \ker \Art_{L / K}. $$
\end{example*}

\subsection{Existence theorem}

By the reciprocity law, there is a map from abelian extensions of $ K $ to finite quotients of $ \CCC_K $.

\begin{theorem*}[Existence theorem]
Let $ U \subset \JJJ_K $ be an open subgroup. Then there exists an abelian extension $ L / K $ with
$$ \ker \Art_{L / K} = UK^\times. $$
\end{theorem*}

Combining with the reciprocity law,
$$ \varprojlim_{\text{open subgroups} \ U \subset \JJJ_K} \JJJ_K / K^\times U \xrightarrow{\sim} \Gal\br{K^{\ab} / K}. $$
In particular, if $ \mmm $ is a modulus, and $ U = \U_{K, \mmm} $, there is a corresponding abelian extension of $ K $, called the \textbf{ray class field}.

\begin{example*}
Let $ K = \QQ $ with $ \mmm = \br{m}\infty $. Then the ray class field is $ \QQ\br{\zeta_m} $. So should think of ray class fields as analogues of cyclotomic fields. Maybe later will discuss ray class fields for $ \QQ\br{\sqrt{-d}} $, which correspond to elliptic curves.
\end{example*}

\pagebreak

\subsection{Relation with local class field theory}

Let $ L / K $ be abelian, let $ v \in \V_K $, and let $ w \mid v $. Then
$$
\begin{tikzcd}
\JJJ_K / K^\times \arrow{r}{\Art_{L / K}} & \Gal\br{L / K} \\
K_v^\times \arrow[hookrightarrow]{u} \arrow{r}[swap]{\psi_v} & \D_v = \Gal\br{L_w / K_v} \arrow[subset]{u}
\end{tikzcd}.
$$
Indeed, in the proof of the reciprocity law, it is usual to start with local Artin maps $ \psi_v $.

\begin{example*}
Let $ v \mid \infty $. If $ K_v = L_w $, then $ \psi_v = 1 $. If $ K_v = \RR $ and $ L_w \cong \CC $, then $ \psi_v = \sign : \RR^\times \to \cbr{\pm 1} \cong \Gal\br{L_w / K_v} $ with kernel $ \RR_{> 0} = \N_{\CC / \RR}\br{\CC^\times} $.
\end{example*}

The $ \br{\psi_v} $ combine to give
$$
\begin{tikzcd}
\JJJ_K / \N_{L / K}\br{\JJJ_L} \arrow{r}{\Art_{L / K}} & \Gal\br{L / K} \\
\displaystyle\bigoplus_v K_v^\times / \N_{L_w / K_v}\br{L_w^\times} \arrow{u}{\sim} \arrow{r}[swap]{\sim} & \displaystyle\bigoplus_v \D_v \arrow[subset]{u}
\end{tikzcd}.
$$
So the fact that $ \Art_{L / K}\br{K^\times} = \cbr{1} $, the hard part of the reciprocity law, is equivalent to knowing the relations between the various $ \D_v \subset \Gal\br{L / K} $. Why are ideles better than ideals?
\begin{itemize}
\item Ideals only will tell you about relations between $ \D_v $ for $ v $ unramified.
\item Need ideles to understand properly ramification.
\end{itemize}

\subsection{Hilbert class field}

Let $ K $ be arbitrary with modulus $ \mmm = 0 $. Then $ \Cl_\mmm\br{K} = \Cl\br{K} $. By the existence theorem, there is a corresponding abelian extension $ H / K $, the \textbf{Hilbert class field}, with
$$ \Art_{H / K} : \Cl\br{K} \xrightarrow{\sim} \Gal\br{H / K}. $$
Then $ H / K $ satisfies the following.
\begin{itemize}
\item It is abelian.
\item For all $ v \in \V_{K, \f} $, it is unramified at $ v $, since $ \OOO_v^\times \subset \U_{K, \mmm} $ for all $ v $.
\item At an infinite place $ v $, $ \U_{K, \mmm} \supset K_v^\times $, so the local decomposition group at $ v $ is trivial, that is if $ v $ is a real place of $ K $, then if $ w \mid v $ then $ w $ is also real.
\end{itemize}
Thus $ H / K $ is unramified at all places of $ K $, and $ H $ is the maximal extension with these properties.

\begin{example*}
Let $ K = \QQ\br{\sqrt{-23}} $, so $ \OOO_K = \ZZ\sbr{\tfrac{1 + \sqrt{-23}}{2}} $. By standard computation, $ \Cl\br{K} \cong \ZZ / 3\ZZ $ is generated by $ \sbr{\ppp} $ for $ \ppp = \abr{2, \tfrac{1 + \sqrt{-23}}{2}} $. Let $ \tau \in \Gal\br{K / \QQ} $ be complex conjugation. Then $ \tau\br{\ppp} = \abr{2, \tfrac{1 - \sqrt{-23}}{2}} $ and $ \ppp \cdot \tau\br{\ppp} = \abr{2} $, that is $ \tau\br{\sbr{\ppp}} = \sbr{\ppp}^{-1} $, so $ \tau $ acts as $ -1 $ on $ \Cl\br{K} $. Let $ H $ be the Hilbert class field of $ K $, which is the maximal abelian extension of $ K $ which is unramified at all $ v \in \V_{K, \f} $, that is $ \delta_{H / K} = \OOO_K $. Then $ \sbr{H : K} = 3 $ and Galois. By $ \br{\ref{eq:5}} $ above, $ \tau $ acts as $ -1 $ on $ \Gal\br{H / K} $, so $ H / \QQ $ is an $ \SSS_3 $-extension. Show that $ H $ is the splitting field of $ f = T^3 - T + 1 $ with discriminant $ -23 $. \footnote{Exercise}
\end{example*}

\pagebreak

\subsection{Another example}

\lecture{16}{Thursday}{25/02/21}

A research problem is to show there is no $ \SSS_3 $-extension $ L / \QQ $, so Galois with group $ \SSS_3 $, which is unramified outside $ 2, 7, \infty $, with quadratic subfield $ K = \QQ\br{\sqrt{-7}} $ or $ K = \QQ\br{\sqrt{2}} $. Let
$$ \Art_{L / K} : \JJJ_K / K^\times \twoheadrightarrow \Gal\br{L / K} \cong \ZZ / 3\ZZ. $$
The condition that $ L / \QQ $ is Galois with group $ \SSS_3 $ is $ \Art_{L / K}\br{\tau\br{x}} = \Art_{L / K}\br{x^{-1}} $, by $ \br{\ref{eq:5}} $, since $ \Gal\br{K / \QQ} = \abr{\tau} $ acts on $ \Gal\br{L / K} $ by conjugation non-trivially. For both $ \QQ\br{\sqrt{-7}} $ and $ \QQ\br{\sqrt{2}} $, $ \Cl\br{K} = 1 $. So
$$ \JJJ_K / K^\times \xleftarrow{\sim} \JJJ_{K, \emptyset} / \OOO_K^\times = \br{K_\infty^\times \times \widehat{\OOO_K}^\times} / \OOO_K^\times. $$
Then $ \Art_{L / K} : K_\infty^\times = \br{\RR^\times}^{\r_1} \times \br{\CC^\times}^{\r_2} \hookrightarrow \JJJ_{K, \emptyset} \to \ZZ / 3\ZZ $ is trivial on $ \CC^\times $ and $ \RR_{> 0} $, and even on $ \RR^\times $, since there is no non-zero continuous homomorphism $ \RR^\times \to \ZZ / 3\ZZ $. So $ \Art_{L / K} $ factors through $ \widehat{\OOO_K}^\times / \OOO_K $, and since $ L / K $ is unramified at $ v \nmid 14 $, factors further by
$$
\begin{tikzcd}
\JJJ_K / K^\times \cong \JJJ_{K, \emptyset} / \OOO_K^\times \arrow{r} \arrow[twoheadrightarrow]{d}[swap]{\Art_{L / K}} & \widehat{\OOO_K}^\times / \OOO_K^\times \arrow{d} \\
\Gal\br{L / K} \cong \ZZ / 3\ZZ & \br{\displaystyle\prod_{v \mid 14} \OOO_v^\times} / \OOO_K^\times \arrow[twoheadrightarrow]{l}{\psi}
\end{tikzcd},
$$
since $ \Art_{L / K}\br{\OOO_v^\times} = 1 $ for all $ v \nmid 14 $. Thus
\begin{equation}
\label{eq:6}
\psi \circ \tau = -\psi.
\end{equation}
\begin{itemize}
\item Let $ K = \QQ\br{\sqrt{-7}} $, so $ \OOO_K^\times = \cbr{\pm 1} $.
\begin{itemize}
\item Since $ -7 \equiv 1 \mod 8 $, $ 2 $ splits in $ K $, so $ \prod_{v \mid 2} \OOO_v^\times = \ZZ_2^\times \times \ZZ_2^\times $ is a pro-$ 2 $ group, so $ \psi\br{\prod_{v \mid 2} \OOO_v^\times} = 0 $.
\item $ 7 $ ramifies, so if $ v \mid 7 $, then $ \OOO_v^\times = \FF_7^\times \times \br{1 + \pi_v\OOO_v} $, where $ \FF_7^\times $ is the Teichm\"uller and $ 1 + \pi_v\OOO_v $ is a pro-$ 7 $ group.
\end{itemize}
So $ \psi $ factors through $ \FF_7^\times $, and $ \tau \in \Gal\br{K / \QQ} $ acts trivially on $ \FF_7 $. So by $ \br{\ref{eq:6}} $, there is no possible $ \psi $. There does exist a $ \psi $ with $ \psi \circ \tau = \psi $, unique up to inverse, corresponding to an abelian $ L / \QQ $, which has to be $ \QQ\br{\zeta_7} $.
\item Let $ K = \QQ\br{\sqrt{2}} $, so $ \OOO_K^\times = \abr{-1, \epsilon = 1 + \sqrt{2}} $.
\begin{itemize}
\item $ 2 $ ramifies, so if $ v \mid 2 $, then $ \OOO_v^\times = 1 + \pi_v\OOO_v $ is a pro-$ 2 $ group and $ \psi\br{\OOO_v^\times} = 0 $.
\item Since $ 7 = \br{3 + \sqrt{2}}\br{3 - \sqrt{2}} $, $ \prod_{v \mid 7} \OOO_v^\times = \ZZ_7^\times \times \ZZ_7^\times \cong \FF_7^\times \times \FF_7^\times \times \br{1 + 7\ZZ_7}^2 $, where $ 1 + 7\ZZ_7 $ is a pro-$ 7 $ group, so $ \psi\br{1 + 7\ZZ_7} = 0 $.
\end{itemize}
So $ \psi $ factors through $ \psi : \br{\FF_7^\times \times \FF_7^\times} / \OOO_K^\times \twoheadrightarrow \ZZ / 3\ZZ $. Then $ \tau : \br{x, y} \mapsto \br{y, x} $, so
\begin{equation}
\label{eq:7}
\psi\br{x, x} = 0,
\end{equation}
by $ \br{\ref{eq:6}} $. Now
$$ \epsilon = 1 + \sqrt{2} \equiv
\begin{cases}
-2 & \mod 3 + \sqrt{2} \\
4 & \mod 3 - \sqrt{2}
\end{cases},
$$
that is $ \psi\br{-2, 4} = 0 $. By this and $ \br{\ref{eq:7}} $, $ \psi = 0 $.
\end{itemize}

\pagebreak

\subsection{Galois group of maximal abelian extension}

Fix $ K \subset \overline{\QQ} $. Then
$$ \Art_K : \CCC_K \to \Gal\br{K^{\ab} / K} = \varprojlim_{L / K \ \text{finite abelian} \ L \subset \overline{\QQ}} \Gal\br{L / K}, $$
where $ K^{\ab} $ is the \textbf{maximal abelian extension} of $ K $ in $ \overline{\QQ} $, the union of all finite abelian $ L / K $, and $ \Gal\br{K^{\ab} / K} $ is profinite. As $ \CCC_K^1 \twoheadrightarrow \Gal\br{L / K} $ for all $ L $ and $ \CCC_K^1 $ is compact, $ \CCC_K^1 \twoheadrightarrow \Gal\br{K^{\ab} / K} $, since the image is dense and compact. The existence theorem is equivalent to the statement that $ \Gal\br{K^{\ab} / K} $ is the maximal profinite quotient of $ \CCC_K $, or of $ \CCC_K^1 $. There is a diagram
$$
\begin{tikzcd}
1 \arrow{r} & \JJJ_{K, \emptyset} / \OOO_K^\times \arrow{r} \arrow[twoheadrightarrow]{d} & \CCC_K \arrow{r}{\c} \arrow[twoheadrightarrow]{d}[swap]{\Art_K} & \Cl\br{K} \arrow{r} \arrow{d}{\sim} & 1 \\
1 \arrow{r} & \Gal\br{K^{\ab} / H} \arrow{r} & \Gal\br{K^{\ab} / K} \arrow{r} & \Gal\br{H / K} \arrow{r} & 1
\end{tikzcd},
$$
where $ H $ is the Hilbert class field. What is the kernel of the vertical maps?

\begin{example*}
\hfill
\begin{itemize}
\item If $ K = \QQ $, then $ \CCC_K \cong \RR_{> 0} \times \widehat{\ZZ}^\times \twoheadrightarrow \widehat{\ZZ}^\times = \Gal\br{\QQ^{\ab} / \QQ} $.
\item If $ K = \QQ\br{\sqrt{-d}} $, then $ \JJJ_{K, \emptyset} / \OOO_K^\times \cong \br{\CC^\times \times \widehat{\OOO_K}^\times} / \mu\br{K} $, where $ \mu\br{K} $ is finite, so the maximal profinite quotient is $ \widehat{\OOO_K}^\times / \mu\br{K} $.
\item If $ K = \QQ\br{\sqrt{2}} $ is real quadratic, then $ \OOO_K^\times = \abr{-1, \epsilon = 1 + \sqrt{2}} $ and $ \Cl\br{K} = 1 $, where $ \N_{K / \QQ}\br{\epsilon} = -1 $ and $ \epsilon $ has signature $ \br{1, -1} $. Let $ \epsilon_+ = \epsilon^2 $ be the least totally positive unit. Then
$$
\begin{tikzcd}[row sep=tiny]
\CCC_K = \JJJ_{K, \emptyset} / \OOO_K^\times \arrow{r}{\sim} & \br{\br{\RR^\times}^2 \times \widehat{\OOO_K}^\times} / \abr{-1, \epsilon} & \br{\RR_{> 0}^2 \times \widehat{\OOO_K}^\times} / \abr{\epsilon_+} \arrow{l}[swap]{\sim} \\
\CCC_K^1 = \JJJ_{K, \emptyset}^1 / \OOO_K^\times \arrow[subset]{u} & \br{\RR_{> 0} \times \widehat{\OOO_K}^\times} / \abr{\epsilon_+} \arrow{l}{\sim} \arrow[twoheadrightarrow]{r}[swap]{\pi} & \widehat{\OOO_K}^\times / \overline{\abr{\epsilon_+}}
\end{tikzcd},
$$
so $ \ker \pi = \br{\RR_{> 0} \times \overline{\abr{\epsilon_+}}} / \abr{\epsilon_+} $. If $ G = \varprojlim_i G_i $ is a profinite group and $ g \in G $, there exists a unique continuous $ \phi : \widehat{\ZZ} \to G $ such that $ \phi\br{1} = g $. \footnote{Exercise: easy} So have
$$ \function{\widehat{\ZZ}}{\overline{\abr{\epsilon_+}} \subset \widehat{\OOO_K}^\times}{1}{\epsilon_+}. $$
One can show that $ \widehat{\ZZ} \xrightarrow{\sim} \overline{\abr{\epsilon_+}} $, so there is an isomorphism $ \ker \pi \cong \br{\RR \times \widehat{\ZZ}} / \ZZ = \AA_\QQ / \QQ $, that is have
$$ 1 \to \AA_\QQ / \QQ \to \CCC_K^1 \to \Gal\br{K^{\ab} / K} \to 1, $$
where $ \AA_\QQ / \QQ $ is compact and connected.
\end{itemize}
\end{example*}

For general $ K $, what happens is that
$$
\begin{tikzcd}[row sep=tiny]
1 \arrow{r} & \CCC_K^0 \arrow{r} \arrow[cong]{d} & \CCC_K \arrow{r} \arrow{d} & \Gal\br{K^{\ab} / K} \arrow{r} & 1 \\
1 \arrow{r} & \CCC_K^0 \arrow{r} & \JJJ_{K, \emptyset} / \OOO_K^\times \arrow[subset]{u} \arrow{r} & \Gal\br{K^{\ab} / H} \arrow[subset]{u} \arrow{r} \arrow[cong]{d} & 1 \\
& & & \br{\cbr{\pm 1}^{\r_1} \times \widehat{\OOO_K}^\times} / \overline{\OOO_K^\times} &
\end{tikzcd},
$$
where the maximal connected subgroup of $ \CCC_K $, the closure of $ \RR_{> 0}^{\r_1} \times \br{\CC^\times}^{\r_2} $, is $ \CCC_K^0 = \RR_{> 0} \times \U\br{1}^{\r_2} \times \AA_\QQ^{\r_1 + \r_2 - 1} $.

\pagebreak

\section{\texorpdfstring{$ \zeta $}{Zeta}-functions and \texorpdfstring{$ \L $}{L}-functions}

\subsection{Riemann \texorpdfstring{$ \zeta $}{zeta}-function}

\lecture{17}{Saturday}{27/02/21}

The \textbf{Riemann $ \zeta $-function} is
$$ \zeta\br{s} = \sum_{n \ge 1} \dfrac{1}{n^s} = \prod_p \dfrac{1}{1 - p^{-s}}, \qquad s \in \CC, \qquad \Re s > 1, $$
by unique factorisation in $ \ZZ $. Define
$$ \Z\br{s} = \pi^{-\tfrac{s}{2}}\Gamma\br{\dfrac{s}{2}}\zeta\br{s}. $$

\begin{theorem}
\label{thm:9.1}
$ \Z\br{s} = \Z\br{1 - s} $, with analytic continuation to $ \CC $ except for simple poles at $ s = 0, 1 $ with residues $ \pm 1 $.
\end{theorem}

\begin{proof}
There are three steps.
\begin{enumerate}[leftmargin=0.5in, label=Step \arabic*.]
\item The \textbf{Mellin transform} of $ \tfrac{1}{2}\br{\Theta\br{y} - 1} $ is
$$ \Z\br{2s} = \pi^{-s}\sum_{n \ge 1} \dfrac{1}{n^{2s}}\intd{0}{\infty}{e^{-t}t^{s - 1}}{t} = \intd{0}{\infty}{\sum_{n = 1}^\infty e^{-\pi n^2y}y^{s - 1}}{y} = \intd{0}{\infty}{\dfrac{1}{2}\br{\Theta\br{y} - 1}\dfrac{y^s}{y}}{y}, $$
where $ \Theta $ is the \textbf{theta function}
$$ \Theta\br{y} = \sum_{n = -\infty}^\infty e^{-\pi n^2y}. $$
\item If $ f : \RR \to \CC $ is nice, then the \textbf{Poisson summation formula} is
$$ \sum_{n = -\infty}^\infty f\br{n} = \sum_{n = -\infty}^\infty \widehat{f}\br{n}, $$
where $ \widehat{f} $ is the \textbf{Fourier transform}
$$ \widehat{f}\br{u} = \intd{-\infty}{\infty}{e^{-2\pi iux}f\br{x}}{x}. $$
Take $ f\br{x} = e^{-\pi x^2y} $. Then $ \widehat{f}\br{u} = y^{-1 / 2}e^{\pi u^2 / y} $, so $ \Theta\br{y} = y^{-1 / 2}\Theta\br{1 / y} $.
\item In step $ 1 $, split
$$ \intd{0}{\infty}{\dfrac{1}{2}\br{\Theta\br{y} - 1}\dfrac{y^s}{y}}{y} = \intd{1}{\infty}{\dfrac{1}{2}\br{\Theta\br{y} - 1}\dfrac{y^s}{y}}{y} + \intd{0}{1}{\dfrac{1}{2}\br{\Theta\br{y} - 1}\dfrac{y^s}{y}}{y}, $$
and in the second term, use step $ 2 $ to make into
$$ \intd{0}{1}{\dfrac{1}{2}\br{\Theta\br{y} - 1}\dfrac{y^s}{y}}{y} = \intd{1}{\infty}{\dfrac{1}{2}\br{\Theta\br{\dfrac{1}{y}} - 1}\dfrac{y^{-s}}{y}}{y}, $$
by $ y \mapsto 1 / y $. Get that
$$ \Z\br{2s} = \dfrac{1}{2}\intd{1}{\infty}{\br{\Theta\br{y} - 1}\br{y^s + y^{\tfrac{1}{2} - s}}\dfrac{1}{y}}{y} + \dfrac{1}{2s - 1} - \dfrac{1}{2s}, $$
where the first term is an entire function of $ s $ since $ \Theta\br{y} - 1 \to 0 $ rapidly as $ y \to \infty $, so $ \Z\br{2s} = \Z\br{1 - 2s} $.
\end{enumerate}
\end{proof}

\pagebreak

\subsection{Dedekind \texorpdfstring{$ \zeta $}{zeta}-function}

Let $ K $ be a number field. The \textbf{Dedekind $ \zeta $-function of $ K $} is
$$ \zeta_K\br{s} = \sum_{0 \ne \aaa \subset \OOO_K \ \text{ideals}} \dfrac{1}{\N\br{\aaa}^s}. $$

\begin{proposition}[Euler product]
$$ \zeta_K\br{s} = \prod_{v \in \V_{K, \f}} \dfrac{1}{1 - \q_v^{-s}}, $$
absolutely convergent for $ \Re s > 1 $.
\end{proposition}

\begin{proof}
Formally, if $ \aaa \subset \OOO_K $ such that $ \aaa = \prod_v \ppp_v^{n_v} $ then $ \N\br{\aaa}^{-s} = \prod_v \q_v^{-n_vs} $, so
$$ \zeta_K\br{s} = \prod_v \br{1 + \q_v^{-s} + \dots} = \prod_v \dfrac{1}{1 - \q_v^{-s}}. $$
Now $ \#\cbr{v \mid p} \le n = \sbr{K : \QQ} $, and if $ v \mid p $ then $ \q_v \ge p $, so the product converges by comparison with $ \prod_p \br{1 - p^{-s}}^{-n} = \zeta\br{s}^n $.
\end{proof}

The $ 1 / \br{1 - \q_v^{-s}} $ are \textbf{Euler factors at $ v $}. Define
$$ \Gamma_\RR\br{s} = \pi^{-\tfrac{s}{2}}\Gamma\br{\dfrac{s}{2}}, \qquad \Gamma_\CC\br{s} = 2\br{2\pi}^{-s}\Gamma\br{s}, $$
the Euler factors for the infinite places, and
$$ \Z_K\br{s} = \abs{\d_K}^{\tfrac{s}{2}}\Gamma_\RR\br{s}^{\r_1}\Gamma_\CC\br{s}^{\r_2}\zeta_K\br{s}. $$
The following is a generalisation of \ref{thm:9.1}.

\begin{theorem}
\hfill
\begin{enumerate}
\item $ \Z_K\br{s} $ has an analytic continuation to $ \CC $, apart from simple poles at $ s = 0, 1 $, and $ \Z_K\br{1 - s} = \Z_K\br{s} $.
\item $ \zeta_K\br{s} $ has a simple zero of order $ r = \r_1 + \r_2 - 1 $ at $ s = 0 $, and
\begin{equation}
\label{eq:8}
\lim_{s \to 0} \dfrac{1}{s^r}\zeta_K\br{s} = -\dfrac{\h_K\R_K}{\w_K},
\end{equation}
the \textbf{analytic class number formula}.
\end{enumerate}
\end{theorem}

Here, $ \h_K = \#\Cl\br{K} $ is the class number, $ \w_K = \#\mu\br{K} $ is the number of roots of unity in $ K $, and $ \R_K $ is the \textbf{regulator} of $ K $. If $ \epsilon_1, \dots, \epsilon_r $ are generators for $ \OOO_K^\times / \mu\br{K} \cong \ZZ^r $, by the unit theorem, $ \R_K $ is the absolute value of any $ \br{r \times r} $-minor of the matrix
$$ \br{\log \abs{\epsilon_j}_v}_{1 \le j \le r, \ v \in \V_{K, \infty}}. $$
Note that by the product formula, the sum of the columns of this matrix is zero, so minors are equal up to sign. Then $ \R_K \ne 0 $ by the proof of the unit theorem. More usual to write $ \br{\ref{eq:8}} $ at $ s = 1 $ but more complicated.

\begin{example*}
If $ K = \QQ $, then $ \zeta\br{0} = -\tfrac{1}{2} $.
\end{example*}

There are two ways to prove this.
\begin{itemize}
\item Hecke, using theta functions.
\item Tate, using adeles. Generalises much more easily to other $ \L $-functions, such as $ \L $-functions of characters of $ \CCC_K $.
\end{itemize}

\pagebreak

Tate's proof is an adelic version of \ref{thm:9.1}. The idea is to first interpret $ \zeta_K\br{s} $, or $ \Z_K\br{s} $, as an adelic integral. Assuming we know how to integrate on $ \QQ_p $,
$$ \intd{\ZZ_p \setminus \cbr{0}}{}{\abs{x}_p^{s - 1}}{x} = \sum_{n \ge 0} \intd{p^n\ZZ_p \setminus p^{n + 1}\ZZ_p}{}{p^{-n\br{s - 1}}}{x} = \sum_{n \ge 0} p^{-n\br{s - 1}}\meas\br{p^n\ZZ_p \setminus p^{n + 1}\ZZ_p}. $$
Then
$$ \ZZ_p = \bigsqcup_{a = 0}^{p^n - 1} a + p^n\ZZ_p, \qquad \meas\br{a + p^n\ZZ_p} = \dfrac{1}{p^n}\meas\br{\ZZ_p}, $$
so
$$ \intd{\ZZ_p \setminus \cbr{0}}{}{\abs{x}_p^{s - 1}}{x} = \sum_{n \ge 0} p^{-n\br{s - 1}}\br{\dfrac{1}{p^n} - \dfrac{1}{p^{n + 1}}}\meas\br{\ZZ_p} = \br{1 - p^{-1}}\meas\br{\ZZ_p}\dfrac{1}{1 - p^{-s}}, $$
where $ 1 / \br{1 - p^{-s}} $ is the Euler factor at $ p $ in $ \zeta\br{s} $. Suggests that $ \zeta\br{s} $ is a product of $ p $-adic integrals, an adelic integral.
\begin{itemize}
\item The $ \Gamma $-factor will be an integral at an infinite place.
\item Have to normalise measure to get $ 1 / \br{1 - p^{-s}} $ for almost all $ p $.
\item The functional equation will come from a Fourier transform.
\end{itemize}

\subsection{Fourier analysis}

On $ \RR $,
$$ \widehat{f}\br{y} = \intd{-\infty}{\infty}{e^{-2\pi ixy}f\br{x}}{x}, $$
which has three ingredients. Define $ \widehat{f} $ replacing $ \RR $ by any local field $ F $, of characteristic zero.
\begin{itemize}
\item An \textbf{additive character} is a continuous $ 1 \ne \psi : F \to \U\br{1} = \cbr{\abs{z} = 1} \subset \CC^\times $.
\begin{itemize}
\item If $ F = \RR $, then $ \psi\br{x} = e^{-2\pi ix} $.
\item If $ F = \CC $, then $ \psi\br{z} = e^{-2\pi i\br{z + \overline{z}}} $.
\item Let $ F / \QQ_p $ be finite. Since $ \QQ_p = \ZZ\sbr{1 / p} + \ZZ_p $, define
$$ \function[\psi_p]{\QQ_p / \ZZ_p}{\U\br{1}}{x}{e^{2\pi iy}}, \qquad y \in \ZZ\sbr{\dfrac{1}{p}}, \qquad x - y \in \ZZ_p, $$
which is well-defined. Let $ \psi = \psi_p \circ \Tr_{F / \QQ_p} : F \to \U\br{1} $.
\end{itemize}
Why the sign in the case $ F = \RR $ or $ F = \CC $? If $ x \in \QQ $, then $ \psi_\infty\br{x}\prod_p \psi_p\br{x} = 1 $.
\end{itemize}

\end{document}