\def\module{Profinite Groups and Group Cohomology}
\def\lecturer{Dr Gareth Wilkes}
\def\term{Lent 2020}
\def\cover{}
\def\syllabus{}
\def\thm{subsection}

\documentclass{article}

% Packages

\usepackage{amssymb}
\usepackage{amsthm}
\usepackage[UKenglish]{babel}
\usepackage{commath}
\usepackage{enumitem}
\usepackage{etoolbox}
\usepackage{fancyhdr}
\usepackage[margin=1in]{geometry}
\usepackage{graphicx}
\usepackage[hidelinks]{hyperref}
\usepackage[utf8]{inputenc}
\usepackage{listings}
\usepackage{mathdots}
\usepackage{mathtools}
\usepackage{stmaryrd}
\usepackage{tikz-cd}
\usepackage{csquotes}

% Formatting

\addto\captionsUKenglish{\renewcommand{\abstractname}{Syllabus}}
\delimitershortfall5pt
\ifx\thm\undefined\newtheorem{n}{}\else\newtheorem{n}{}[\thm]\fi
\newcommand\newoperator[1]{\ifcsdef{#1}{\cslet{#1}{\relax}}{}\csdef{#1}{\operatorname{#1}}}
\newcommand\vC{\check{\C}}
\newcommand\vH{\check{\H}}
\setlength{\parindent}{0cm}

% Environments

\theoremstyle{plain}
\newtheorem{algorithm}[n]{Algorithm}
\newtheorem*{algorithm*}{Algorithm}
\newtheorem{algorithm**}{Algorithm}
\newtheorem{conjecture}[n]{Conjecture}
\newtheorem*{conjecture*}{Conjecture}
\newtheorem{conjecture**}{Conjecture}
\newtheorem{corollary}[n]{Corollary}
\newtheorem*{corollary*}{Corollary}
\newtheorem{corollary**}{Corollary}
\newtheorem{lemma}[n]{Lemma}
\newtheorem*{lemma*}{Lemma}
\newtheorem{lemma**}{Lemma}
\newtheorem{proposition}[n]{Proposition}
\newtheorem*{proposition*}{Proposition}
\newtheorem{proposition**}{Proposition}
\newtheorem{theorem}[n]{Theorem}
\newtheorem*{theorem*}{Theorem}
\newtheorem{theorem**}{Theorem}

\theoremstyle{definition}
\newtheorem{aim}[n]{Aim}
\newtheorem*{aim*}{Aim}
\newtheorem{aim**}{Aim}
\newtheorem{axiom}[n]{Axiom}
\newtheorem*{axiom*}{Axiom}
\newtheorem{axiom**}{Axiom}
\newtheorem{condition}[n]{Condition}
\newtheorem*{condition*}{Condition}
\newtheorem{condition**}{Condition}
\newtheorem{definition}[n]{Definition}
\newtheorem*{definition*}{Definition}
\newtheorem{definition**}{Definition}
\newtheorem{example}[n]{Example}
\newtheorem*{example*}{Example}
\newtheorem{example**}{Example}
\newtheorem{exercise}[n]{Exercise}
\newtheorem*{exercise*}{Exercise}
\newtheorem{exercise**}{Exercise}
\newtheorem{fact}[n]{Fact}
\newtheorem*{fact*}{Fact}
\newtheorem{fact**}{Fact}
\newtheorem{goal}[n]{Goal}
\newtheorem*{goal*}{Goal}
\newtheorem{goal**}{Goal}
\newtheorem{law}[n]{Law}
\newtheorem*{law*}{Law}
\newtheorem{law**}{Law}
\newtheorem{plan}[n]{Plan}
\newtheorem*{plan*}{Plan}
\newtheorem{plan**}{Plan}
\newtheorem{problem}[n]{Problem}
\newtheorem*{problem*}{Problem}
\newtheorem{problem**}{Problem}
\newtheorem{question}[n]{Question}
\newtheorem*{question*}{Question}
\newtheorem{question**}{Question}
\newtheorem{warning}[n]{Warning}
\newtheorem*{warning*}{Warning}
\newtheorem{warning**}{Warning}
\newtheorem{acknowledgements}[n]{Acknowledgements}
\newtheorem*{acknowledgements*}{Acknowledgements}
\newtheorem{acknowledgements**}{Acknowledgements}
\newtheorem{annotations}[n]{Annotations}
\newtheorem*{annotations*}{Annotations}
\newtheorem{annotations**}{Annotations}
\newtheorem{assumption}[n]{Assumption}
\newtheorem*{assumption*}{Assumption}
\newtheorem{assumption**}{Assumption}
\newtheorem{conclusion}[n]{Conclusion}
\newtheorem*{conclusion*}{Conclusion}
\newtheorem{conclusion**}{Conclusion}
\newtheorem{claim}[n]{Claim}
\newtheorem*{claim*}{Claim}
\newtheorem{claim**}{Claim}
\newtheorem{notation}[n]{Notation}
\newtheorem*{notation*}{Notation}
\newtheorem{notation**}{Notation}
\newtheorem{note}[n]{Note}
\newtheorem*{note*}{Note}
\newtheorem{note**}{Note}
\newtheorem{remark}[n]{Remark}
\newtheorem*{remark*}{Remark}
\newtheorem{remark**}{Remark}

% Lectures

\newcommand{\lecture}[3]{ % Lecture
  \marginpar{
    Lecture #1 \\
    #2 \\
    #3
  }
}

% Blackboard

\renewcommand{\AA}{\mathbb{A}} % Blackboard A
\newcommand{\BB}{\mathbb{B}}   % Blackboard B
\newcommand{\CC}{\mathbb{C}}   % Blackboard C
\newcommand{\DD}{\mathbb{D}}   % Blackboard D
\newcommand{\EE}{\mathbb{E}}   % Blackboard E
\newcommand{\FF}{\mathbb{F}}   % Blackboard F
\newcommand{\GG}{\mathbb{G}}   % Blackboard G
\newcommand{\HH}{\mathbb{H}}   % Blackboard H
\newcommand{\II}{\mathbb{I}}   % Blackboard I
\newcommand{\JJ}{\mathbb{J}}   % Blackboard J
\newcommand{\KK}{\mathbb{K}}   % Blackboard K
\newcommand{\LL}{\mathbb{L}}   % Blackboard L
\newcommand{\MM}{\mathbb{M}}   % Blackboard M
\newcommand{\NN}{\mathbb{N}}   % Blackboard N
\newcommand{\OO}{\mathbb{O}}   % Blackboard O
\newcommand{\PP}{\mathbb{P}}   % Blackboard P
\newcommand{\QQ}{\mathbb{Q}}   % Blackboard Q
\newcommand{\RR}{\mathbb{R}}   % Blackboard R
\renewcommand{\SS}{\mathbb{S}} % Blackboard S
\newcommand{\TT}{\mathbb{T}}   % Blackboard T
\newcommand{\UU}{\mathbb{U}}   % Blackboard U
\newcommand{\VV}{\mathbb{V}}   % Blackboard V
\newcommand{\WW}{\mathbb{W}}   % Blackboard W
\newcommand{\XX}{\mathbb{X}}   % Blackboard X
\newcommand{\YY}{\mathbb{Y}}   % Blackboard Y
\newcommand{\ZZ}{\mathbb{Z}}   % Blackboard Z

% Brackets

\renewcommand{\eval}[1]{\left. #1 \right|}                     % Evaluation
\newcommand{\br}{\del}                                         % Brackets
\newcommand{\abr}[1]{\left\langle #1 \right\rangle}            % Angle brackets
\newcommand{\fbr}[1]{\left\lfloor #1 \right\rfloor}            % Floor brackets
\newcommand{\st}{\ \middle| \ }                                % Such that
\newcommand{\intd}[4]{\int_{#1}^{#2} \, #3 \, \dif #4}         % Single integral
\newcommand{\iintd}[4]{\iint_{#1} \, #2 \, \dif #3 \, \dif #4} % Double integral

% Calligraphic

\newcommand{\AAA}{\mathcal{A}} % Calligraphic A
\newcommand{\BBB}{\mathcal{B}} % Calligraphic B
\newcommand{\CCC}{\mathcal{C}} % Calligraphic C
\newcommand{\DDD}{\mathcal{D}} % Calligraphic D
\newcommand{\EEE}{\mathcal{E}} % Calligraphic E
\newcommand{\FFF}{\mathcal{F}} % Calligraphic F
\newcommand{\GGG}{\mathcal{G}} % Calligraphic G
\newcommand{\HHH}{\mathcal{H}} % Calligraphic H
\newcommand{\III}{\mathcal{I}} % Calligraphic I
\newcommand{\JJJ}{\mathcal{J}} % Calligraphic J
\newcommand{\KKK}{\mathcal{K}} % Calligraphic K
\newcommand{\LLL}{\mathcal{L}} % Calligraphic L
\newcommand{\MMM}{\mathcal{M}} % Calligraphic M
\newcommand{\NNN}{\mathcal{N}} % Calligraphic N
\newcommand{\OOO}{\mathcal{O}} % Calligraphic O
\newcommand{\PPP}{\mathcal{P}} % Calligraphic P
\newcommand{\QQQ}{\mathcal{Q}} % Calligraphic Q
\newcommand{\RRR}{\mathcal{R}} % Calligraphic R
\newcommand{\SSS}{\mathcal{S}} % Calligraphic S
\newcommand{\TTT}{\mathcal{T}} % Calligraphic T
\newcommand{\UUU}{\mathcal{U}} % Calligraphic U
\newcommand{\VVV}{\mathcal{V}} % Calligraphic V
\newcommand{\WWW}{\mathcal{W}} % Calligraphic W
\newcommand{\XXX}{\mathcal{X}} % Calligraphic X
\newcommand{\YYY}{\mathcal{Y}} % Calligraphic Y
\newcommand{\ZZZ}{\mathcal{Z}} % Calligraphic Z

% Fraktur

\newcommand{\aaa}{\mathfrak{a}}   % Fraktur a
\newcommand{\bbb}{\mathfrak{b}}   % Fraktur b
\newcommand{\ccc}{\mathfrak{c}}   % Fraktur c
\newcommand{\ddd}{\mathfrak{d}}   % Fraktur d
\newcommand{\eee}{\mathfrak{e}}   % Fraktur e
\newcommand{\fff}{\mathfrak{f}}   % Fraktur f
\renewcommand{\ggg}{\mathfrak{g}} % Fraktur g
\newcommand{\hhh}{\mathfrak{h}}   % Fraktur h
\newcommand{\iii}{\mathfrak{i}}   % Fraktur i
\newcommand{\jjj}{\mathfrak{j}}   % Fraktur j
\newcommand{\kkk}{\mathfrak{k}}   % Fraktur k
\renewcommand{\lll}{\mathfrak{l}} % Fraktur l
\newcommand{\mmm}{\mathfrak{m}}   % Fraktur m
\newcommand{\nnn}{\mathfrak{n}}   % Fraktur n
\newcommand{\ooo}{\mathfrak{o}}   % Fraktur o
\newcommand{\ppp}{\mathfrak{p}}   % Fraktur p
\newcommand{\qqq}{\mathfrak{q}}   % Fraktur q
\newcommand{\rrr}{\mathfrak{r}}   % Fraktur r
\newcommand{\sss}{\mathfrak{s}}   % Fraktur s
\newcommand{\ttt}{\mathfrak{t}}   % Fraktur t
\newcommand{\uuu}{\mathfrak{u}}   % Fraktur u
\newcommand{\vvv}{\mathfrak{v}}   % Fraktur v
\newcommand{\www}{\mathfrak{w}}   % Fraktur w
\newcommand{\xxx}{\mathfrak{x}}   % Fraktur x
\newcommand{\yyy}{\mathfrak{y}}   % Fraktur y
\newcommand{\zzz}{\mathfrak{z}}   % Fraktur z

% Maps

\newcommand{\bijection}[7][]{    % Bijection
  \ifx &#1&
    \begin{array}{rcl}
      #2 & \longleftrightarrow & #3 \\
      #4 & \longmapsto         & #5 \\
      #6 & \longmapsfrom       & #7
    \end{array}
  \else
    \begin{array}{ccrcl}
      #1 & : & #2 & \longrightarrow & #3 \\
         &   & #4 & \longmapsto     & #5 \\
         &   & #6 & \longmapsfrom   & #7
    \end{array}
  \fi
}
\newcommand{\correspondence}[2]{ % Correspondence
  \cbr{
    \begin{array}{c}
      #1
    \end{array}
  }
  \qquad
  \leftrightsquigarrow
  \qquad
  \cbr{
    \begin{array}{c}
      #2
    \end{array}
  }
}
\newcommand{\function}[5][]{     % Function
  \ifx &#1&
    \begin{array}{rcl}
      #2 & \longrightarrow & #3 \\
      #4 & \longmapsto     & #5
    \end{array}
  \else
    \begin{array}{ccrcl}
      #1 & : & #2 & \longrightarrow & #3 \\
         &   & #4 & \longmapsto     & #5
    \end{array}
  \fi
}
\newcommand{\functions}[7][]{    % Functions
  \ifx &#1&
    \begin{array}{rcl}
      #2 & \longrightarrow & #3 \\
      #4 & \longmapsto     & #5 \\
      #6 & \longmapsto     & #7
    \end{array}
  \else
    \begin{array}{ccrcl}
      #1 & : & #2 & \longrightarrow & #3 \\
         &   & #4 & \longmapsto     & #5 \\
         &   & #6 & \longmapsto     & #7
    \end{array}
  \fi
}

% Matrices

\newcommand{\onebytwo}[2]{      % One by two matrix
  \begin{pmatrix}
    #1 & #2
  \end{pmatrix}
}
\newcommand{\onebythree}[3]{    % One by three matrix
  \begin{pmatrix}
    #1 & #2 & #3
  \end{pmatrix}
}
\newcommand{\twobyone}[2]{      % Two by one matrix
  \begin{pmatrix}
    #1 \\
    #2
  \end{pmatrix}
}
\newcommand{\twobytwo}[4]{      % Two by two matrix
  \begin{pmatrix}
    #1 & #2 \\
    #3 & #4
  \end{pmatrix}
}
\newcommand{\threebyone}[3]{    % Three by one matrix
  \begin{pmatrix}
    #1 \\
    #2 \\
    #3
  \end{pmatrix}
}
\newcommand{\threebythree}[9]{  % Three by three matrix
  \begin{pmatrix}
    #1 & #2 & #3 \\
    #4 & #5 & #6 \\
    #7 & #8 & #9
  \end{pmatrix}
}

% Operators

\newoperator{Aut}   % Automorphism
\newoperator{cell}  % Cell
\newoperator{ch}    % Characteristic
\newoperator{cl}    % Cup length
\newoperator{codim} % Codimension
\newoperator{coker} % Cokernel
\newoperator{Cone}  % Cone
\newoperator{Crit}  % Critical
\newoperator{div}   % Divisor
\newoperator{Div}   % Divisor group
\newoperator{End}   % Endomorphism
\newoperator{Ext}   % Ext
\newoperator{Frac}  % Fraction
\newoperator{Frob}  % Frobenius
\newoperator{Gal}   % Galois
\newoperator{GL}    % General linear
\newoperator{Gr}    % Grassmannian
\newoperator{Ht}    % Height
\newoperator{Hom}   % Homomorphism
\newoperator{id}    % Identity
\newoperator{im}    % Image
\newoperator{Int}   % Integral
\newoperator{ker}   % Kernel
\newoperator{Mat}   % Matrix
\newoperator{Morse} % Morse
\newoperator{MV}    % Mayer-Vietoris
\newoperator{nr}    % Unramified
\newoperator{ns}    % Nonsingular
\newoperator{Ob}    % Object
\newoperator{ord}   % Order
\newoperator{Pic}   % Picard
\newoperator{Proj}  % Projective
\newoperator{res}   % Restriction
\newoperator{rk}    % Rank
\newoperator{Spec}  % Spectrum
\newoperator{supp}  % Support
\newoperator{taut}  % Tautological
\newoperator{tors}  % Torsion
\newoperator{Tr}    % Trace
\newoperator{triv}  % Trivial

% Roman

\newcommand{\A}{\mathrm{A}}   % Roman A
\newcommand{\B}{\mathrm{B}}   % Roman B
\newcommand{\C}{\mathrm{C}}   % Roman C
\newcommand{\D}{\mathrm{D}}   % Roman D
\newcommand{\E}{\mathrm{E}}   % Roman E
\newcommand{\F}{\mathrm{F}}   % Roman F
\newcommand{\G}{\mathrm{G}}   % Roman G
\renewcommand{\H}{\mathrm{H}} % Roman H
\newcommand{\I}{\mathrm{I}}   % Roman I
\newcommand{\J}{\mathrm{J}}   % Roman J
\newcommand{\K}{\mathrm{K}}   % Roman K
\renewcommand{\L}{\mathrm{L}} % Roman L
\newcommand{\M}{\mathrm{M}}   % Roman M
\newcommand{\N}{\mathrm{N}}   % Roman N
\renewcommand{\O}{\mathrm{O}} % Roman O
\renewcommand{\P}{\mathrm{P}} % Roman P
\newcommand{\Q}{\mathrm{Q}}   % Roman Q
\newcommand{\R}{\mathrm{R}}   % Roman R
\renewcommand{\S}{\mathrm{S}} % Roman S
\newcommand{\T}{\mathrm{T}}   % Roman T
\newcommand{\U}{\mathrm{U}}   % Roman U
\newcommand{\V}{\mathrm{V}}   % Roman V
\newcommand{\W}{\mathrm{W}}   % Roman W
\newcommand{\X}{\mathrm{X}}   % Roman X
\newcommand{\Y}{\mathrm{Y}}   % Roman Y
\newcommand{\Z}{\mathrm{Z}}   % Roman Z

\renewcommand{\a}{\mathrm{a}} % Roman a
\renewcommand{\b}{\mathrm{b}} % Roman b
\renewcommand{\c}{\mathrm{c}} % Roman c
\renewcommand{\d}{\mathrm{d}} % Roman d
\newcommand{\e}{\mathrm{e}}   % Roman e
\newcommand{\f}{\mathrm{f}}   % Roman f
\newcommand{\g}{\mathrm{g}}   % Roman g
\newcommand{\h}{\mathrm{h}}   % Roman h
\renewcommand{\i}{\mathrm{i}} % Roman i
\renewcommand{\j}{\mathrm{j}} % Roman j
\renewcommand{\k}{\mathrm{k}} % Roman k
\renewcommand{\l}{\mathrm{l}} % Roman l
\newcommand{\m}{\mathrm{m}}   % Roman m
\renewcommand{\n}{\mathrm{n}} % Roman n
\renewcommand{\o}{\mathrm{o}} % Roman o
\newcommand{\p}{\mathrm{p}}   % Roman p
\newcommand{\q}{\mathrm{q}}   % Roman q
\renewcommand{\r}{\mathrm{r}} % Roman r
\newcommand{\s}{\mathrm{s}}   % Roman s
\renewcommand{\t}{\mathrm{t}} % Roman t
\renewcommand{\u}{\mathrm{u}} % Roman u
\renewcommand{\v}{\mathrm{v}} % Roman v
\newcommand{\w}{\mathrm{w}}   % Roman w
\newcommand{\x}{\mathrm{x}}   % Roman x
\newcommand{\y}{\mathrm{y}}   % Roman y
\newcommand{\z}{\mathrm{z}}   % Roman z

% Tikz

\tikzset{
  arrow symbol/.style={"#1" description, allow upside down, auto=false, draw=none, sloped},
  subset/.style={arrow symbol={\subset}},
  cong/.style={arrow symbol={\cong}}
}

% Fancy header

\pagestyle{fancy}
\lhead{\module}
\rhead{\nouppercase{\leftmark}}

% Make title

\title{\module}
\author{Lectured by \lecturer \\ Typed by David Kurniadi Angdinata}
\date{\term}

% Macros
\newcommand{\Grp}{\textbf{Grp}}
\newcommand{\Set}{\textbf{Set}}
\newcommand{\twobytwosmall}[4]{
  \begin{psmallmatrix}
    #1 & #2 \\
    #3 & #4
  \end{psmallmatrix}
}

\begin{document}

% Title page
\maketitle
\cover
\vfill
\begin{abstract}
\noindent\syllabus
\end{abstract}

\pagebreak

% Contents page
\tableofcontents

\pagebreak

% Document page
\setcounter{section}{-1}

\section{Introduction}

\lecture{1}{Thursday}{21/01/21}

A question is, when are things different?
\begin{itemize}
\item $ \ZZ $ is in bijection with $ \QQ $, by writing down a bijection.
\item $ \QQ $ is not in bijection with $ \RR $, by diagonalisation.
\end{itemize}
A solution is to try to find an invariant, which is
\begin{itemize}
\item easier to compute,
\item computable, and
\item preserved under isomorphism.
\end{itemize}

\begin{example*}
\hfill
\begin{itemize}
\item Cardinality of a set.
\item Dimension and base field of a vector space, which is complete.
\item For an algebraic field extension $ K $ over $ \QQ $, the degree $ \sbr{K : \QQ} $ and the Galois group $ \Gal\br{K / \QQ} $.
\item For a topological space $ X $, compactness, connectedness, simplicial homology groups $ \H_\bullet\br{X} $, and the fundamental group $ \pi_1\br{X} $.
\end{itemize}
\end{example*}

\begin{theorem}
There is no algorithm that decides whether a finite presentation represents the trivial group.
\end{theorem}

Finite groups we can decide.
\begin{itemize}
\item List all the finite quotients of a group.
\item If you have two such lists, you can compare.
\item If two groups have different sets of finite quotients, they are not isomorphic.
\end{itemize}
How often does this work?
\begin{itemize}
\item Combine all the finite quotients into one object to study, the \textbf{profinite completion}, which is a limit of the finite groups.
\item More generally, a limit of finite groups is called a \textbf{profinite group}.
\end{itemize}

\begin{example*}
\hfill
\begin{itemize}
\item In Galois theory, let $ K = \bigcup_{N \in \NN} K_N $ be the extension of $ \QQ $ adjoining all $ p^N $-th roots of unity for $ p $ a fixed prime and $ N \in \NN $, which gives a short exact sequence of Galois groups
$$ \Gal\br{K / K_N} \to \Gal\br{K / \QQ} \twoheadrightarrow \Gal\br{K_N / \QQ}. $$
Then $ \Gal\br{K_N / \QQ} = \br{\ZZ / p^N\ZZ}^\times $ and $ \Gal\br{K / \QQ} = \varprojlim_N \br{\ZZ / p^N\ZZ}^\times = \ZZ_p^\times $.
\item In algebraic geometry, \'etale fundamental groups are profinite groups.
\end{itemize}
\end{example*}

The second part of the course is \textbf{group cohomology}, which is another invariant, with the following applications.
\begin{itemize}
\item Can tell if a group is free for some profinite groups.
\item Given a group $ G $ and an abelian group $ A $, group cohomology tells us how many groups $ E $ exist such that $ A \triangleleft E $ and $ E / A = G $.
\end{itemize}

\pagebreak

\section{Inverse limits}

\subsection{Categories and limits}

Let $ A $ and $ B $ be sets. How to combine into one thing? The disjoint union $ A \sqcup B $ has inclusion maps $ i_A : A \hookrightarrow A \sqcup B $ and $ i_B : B \hookrightarrow A \sqcup B $, and for any other set $ Z $, with functions $ j_A : A \to Z $ and $ j_B : B \to Z $ there is a unique function defined by
$$ \functions[f]{A \sqcup B}{Z}{a}{j_A\br{a}}{b}{j_B\br{b}}, $$
such that $ f \circ i_A = j_A $ and $ f \circ i_B = j_B $, so
$$
\begin{tikzcd}
A \arrow{r}{i_A} \arrow{dr}[swap]{j_A} & A \sqcup B \arrow[dashed]{d}{\exists !f} & B \arrow{l}[swap]{i_B} \arrow{dl}{j_B} \\
& Z &
\end{tikzcd}.
$$
The product $ A \times B $ comes with $ p_A : A \times B \to A $ and $ p_B : A \times B \to B $ such that
$$
\begin{tikzcd}
A & A \times B \arrow{l}[swap]{p_A} \arrow{r}{p_B} & B \\
& Z \arrow{ul}{q_A} \arrow[dashed]{u}{\exists !f} \arrow{ur}[swap]{q_B} &
\end{tikzcd},
$$
where $ f\br{z} = \br{q_A\br{z}, q_B\br{z}} $. Reversed all arrows, so there is a duality, and disjoint union is a coproduct. What about groups, and group homomorphisms? The product still works, but the disjoint union is not a group. The coproduct is the free product $ A * B $ such that
$$
\begin{tikzcd}
A \arrow{r} \arrow{dr} & A * B \arrow[dashed]{d} & B \arrow{l} \arrow{dl} \\
& Z &
\end{tikzcd}.
$$
More generally is the pushout. Given groups $ A $, $ B $, and $ C $, and homomorphisms $ \phi_A : C \to A $ and $ \phi_B : C \to B $, the \textbf{pushout} $ A \sqcup_C B $ is
$$
\begin{tikzcd}
C \arrow{r}{\phi_A} \arrow{d}[swap]{\phi_B} & A \arrow{d}{i_A} \arrow[bend left=30]{ddr}{j_A} & \\
B \arrow{r}[swap]{i_B} \arrow[bend right=30]{drr}[swap]{j_B} & A \sqcup_C B \arrow[dashed]{dr}{\exists !f} & \\
& & Z
\end{tikzcd}.
$$

\begin{definition*}
A \textbf{category} $ \CCC $ consists of
\begin{itemize}
\item a collection of \textbf{objects} $ \Obj \CCC $,
\item a collection of \textbf{morphisms} or \textbf{arrows} $ \Mor \CCC $, such that each $ f \in \Mor \CCC $ has a \textbf{domain} $ X \in \Obj \CCC $ and a \textbf{codomain} $ Y \in \Obj \CCC $ written as $ f : X \to Y $,
\item for all objects $ X \in \Obj \CCC $, you have $ \id_X : X \to X $, and
\item if $ f : X \to Y $ and $ g : Y \to Z $, we have a defined composition $ g \circ f : X \to Z $,
\end{itemize}
such that
\begin{itemize}
\item if $ f : X \to Y $, then $ \id_Y \circ f = f = f \circ \id_X $, and
\item if $ f : W \to X $, $ g : X \to Y $, and $ h : Y \to Z $, then $ h \circ \br{g \circ f} = \br{h \circ g} \circ f $.
\end{itemize}
\end{definition*}

\pagebreak

\begin{example*}
\hfill
\begin{itemize}
\item In $ \Set $, objects are sets and morphisms are functions.
\item In $ \Grp $, objects are groups and morphisms are group homomorphisms.
\item In $ \Grp_{\text{fin}} $, objects are finite groups.
\item In $ \Grp_{\text{inj}} $, morphisms are injective group homomorphisms.
\end{itemize}
\end{example*}

\begin{definition*}
A \textbf{partial ordering} on a set $ J $ is a binary relation $ \le $ such that
\begin{itemize}
\item $ i \le i $,
\item if $ i \le j $ and $ j \le i $, then $ i = j $, and
\item if $ i \le j $ and $ j \le k $, then $ i \le k $.
\end{itemize}
A \textbf{poset} is a pair $ \br{J, \le} $, which is a \textbf{total ordering} if for all $ i, j \in J $ either $ i \le j $ or $ j \le i $. The \textbf{poset category} $ \JJJ $ has objects $ \Obj \JJJ = J $ and morphisms $ \Mor \JJJ = \cbr{i \to j \st i \le j} $.
\end{definition*}

\lecture{2}{Saturday}{23/01/21}

\begin{definition*}
Let $ \CCC $ be a category. A \textbf{product} of $ A, B \in \Obj \CCC $ is an object $ P $, equipped with morphisms $ p_A : P \to A $ and $ p_B : P \to B $, such that for all $ Z \in \Obj \CCC $ and for all $ q_A : Z \to A $ and $ q_B : Z \to B $, there exists a unique $ f : Z \to P $ such that $ p_A \circ f = q_A $ and $ p_B \circ f = q_B $, so
$$
\begin{tikzcd}
& Z \arrow{dl}[swap]{q_A} \arrow[dashed]{d}{\exists !f} \arrow{dr}{q_B} & \\
A & P \arrow{l}{p_A} \arrow{r}[swap]{p_B} & B
\end{tikzcd}.
$$
\end{definition*}

\begin{definition*}
Objects $ A $ and $ B $ in a category $ \CCC $ are \textbf{isomorphic} if there exist $ f : A \to B $ and $ g : B \to A $ such that $ g \circ f = \id_A $ and $ f \circ g = \id_B $.
\end{definition*}

\begin{proposition}
If a product of $ A $ and $ B $ in $ \CCC $ exists, then it is unique up to a unique isomorphism.
\end{proposition}

\begin{proof}
Let $ \br{P, p_A, p_B} $ and $ \br{P', p_A', p_B'} $ be products. Then
$$
\begin{tikzcd}
& P' \arrow{dl}[swap]{p_A'} \arrow[bend right=30, dashed]{dd}[swap]{\exists !f} \arrow{dr}{p_B'} & \\
A & & B \\
& P \arrow{ul}{p_A} \arrow[bend right=30, dashed]{uu}[swap]{\exists !g} \arrow{ur}[swap]{p_B} &
\end{tikzcd}.
$$
Consider $ f \circ g : P \to P $. Then $ p_A \circ f \circ g = p_A' \circ g = p_A $ and $ p_B \circ f \circ g = p_B' \circ g = p_B $. By uniqueness, $ f \circ g = \id_P $. Similarly, $ g \circ f = \id_{P'} $.
\end{proof}

\begin{notation*}
Define $ P = A \times B $.
\end{notation*}

\begin{definition*}
Let $ \CCC $ be a category and $ A, B \in \Obj \CCC $. Then a \textbf{coproduct} is an object $ A \sqcup B $, together with maps $ i_A : A \to A \sqcup B $ and $ i_B : B \to A \sqcup B $, with the universal property
$$
\begin{tikzcd}
A \arrow{r}{i_A} \arrow{dr}[swap]{j_A} & A \sqcup B \arrow[dashed]{d}{\exists !f} & B \arrow{l}[swap]{i_B} \arrow{dl}{j_B} \\
& Z &
\end{tikzcd}.
$$
\end{definition*}

Products are examples of limits and coproducts are examples of colimits.

\pagebreak

\begin{definition*}
Let $ \CCC $ and $ \DDD $ be categories. A \textbf{functor} $ F : \CCC \to \DDD $ associates an object $ F\br{X} \in \Obj \DDD $ to each $ X \in \Obj \CCC $, and a morphism $ F\br{f} : F\br{X} \to F\br{Y} $ for each $ f : X \to Y $ in $ \CCC $, such that
\begin{itemize}
\item $ F\br{\id_X} = \id_{F\br{X}} $, and
\item $ F\br{g \circ f} = F\br{g} \circ F\br{f} $.
\end{itemize}
\end{definition*}

\begin{definition*}
Let $ \JJJ $ and $ \CCC $ be categories. A \textbf{diagram of shape $ \JJJ $ in $ \CCC $} is a functor $ X : \JJJ \to \CCC $. Often write $ X\br{j} = X_j $, for $ j \in \Obj \JJJ $.
\end{definition*}

Very often, $ \JJJ $ is a poset category. In that case, if $ i \le j $, there exists a unique arrow $ f : i \to j $ and then denote $ X\br{f} = \phi_{ij} $.

\begin{definition*}
A \textbf{cone} on a diagram $ X : \JJJ \to \CCC $ is an object $ Z \in \Obj \CCC $, together with maps $ p_j : Z \to X_j = X\br{j} $ for all $ j \in \Obj \JJJ $ such that for all $ f : i \to j $, $ X\br{f} \circ p_i = p_j $, so
$$
\begin{tikzcd}
& Z \arrow{dl}[swap]{p_i} \arrow{dr}{p_j} & \\
X_i \arrow{rr}[swap]{X\br{f}} & & X_j
\end{tikzcd}.
$$
A \textbf{limit} of a diagram $ X : \JJJ \to \CCC $ is a cone $ L $, with morphisms $ p_j $, such that for any cone $ Z $, with morphisms $ q_j $, there is a unique $ g : Z \to L $ such that $ p_j \circ f = q_j $, for all $ j \in \Obj \JJJ $, so
$$
\begin{tikzcd}
& Z \arrow[dashed]{d}{\exists !g} \arrow[bend right=30]{ddl}[swap]{q_i} \arrow[bend left=30]{ddr}{q_j} & \\
& L \arrow{dl}[swap]{p_i} \arrow{dr}{p_j} & \\
X_i \arrow{rr}[swap]{X\br{f}} & & X_j
\end{tikzcd},
$$
for $ f : i \to j $. \textbf{Colimits} are as limits, but arrows are reversed.
\end{definition*}

\begin{example*}
\hfill
\begin{itemize}
\item If $ \JJJ $ is the category
$$
\begin{tikzpicture}
\fill (0, 0) circle (0.05);
\fill (1, 0) circle (0.05);
\end{tikzpicture},
$$
then a diagram of shape $ \JJJ $ is a pair of objects. The limit is the product and the colimit is the coproduct.
\item If $ \JJJ $ is the category
$$
\begin{tikzpicture}
\fill (0, 0) circle (0.05);
\fill (1, 0) circle (0.05);
\fill (0, -1) circle (0.05);
\draw [->] (0, 0) to (1, 0);
\draw [->] (0, 0) to (0, -1);
\end{tikzpicture},
$$
then a diagram of shape $ \JJJ $ in $ \Grp $ would be
$$
\begin{tikzcd}
C \arrow{r}{\phi_{CA}} \arrow{d}[swap]{\phi_{CB}} & A \\
B &
\end{tikzcd}.
$$
The colimit is the pushout.
\end{itemize}
\end{example*}

\begin{proposition}
Limits and colimits are unique up to unique isomorphism.
\end{proposition}

\pagebreak

\subsection{Inverse limits and profinite groups}

Let $ G $ be a group. Let $ \NNN $ be the poset category whose objects are $ \cbr{N \triangleleft_\f G} $, where $ N \triangleleft_\f G $ are finite index, with ordering $ N_1 \le N_2 $ if and only if $ N_1 \subseteq N_2 $. There is a diagram of shape $ \NNN $ in $ \Grp $,
$$ \function[X]{\NNN}{\Grp}{N}{X_N = G / N}. $$
If $ N_1 \le N_2 $, then $ X\br{N_1 \to N_2} $ is the quotient map $ \phi_{N_1N_2} : G / N_1 \to G / N_2 $, the transition maps.

\begin{definition*}
Let $ G $ be a group. The \textbf{profinite completion} of $ G $ is the limit of this diagram, denoted $ \widehat{G} $.
\end{definition*}

Then $ G $ comes with projections $ p_N : \widehat{G} \to G / N $ for all $ N \triangleleft_\f G $ such that
\begin{itemize}
\item if $ N_1 \subseteq N_2 $, then $ \phi_{N_1N_2} \circ p_{N_1} = p_{N_2} $, and
\item if $ Z $ is a group, with $ q_N : Z \to G / N $ such that $ \phi_{N_1N_2} \circ q_{N_1} = q_{N_2} $, there exists a unique $ f : Z \to \widehat{G} $ such that $ p_N \circ f = q_N $ for all $ N $.
\end{itemize}
Thus
$$
\begin{tikzcd}
& Z \arrow[dashed]{d}{\exists !f} \arrow[bend right=30]{ddl} \arrow[bend left=30]{ddr} & \\
& \widehat{G} \arrow{dl} \arrow{dr} & \\
G / N_1 \arrow{rr} & & G / N_2
\end{tikzcd}.
$$
In particular, $ Z = G $ works, so there is a unique morphism $ \iota_G : G \to \widehat{G} $, the \textbf{canonical morphism}, such that the diagrams commute.

\begin{definition*}
A poset $ \br{J, \le} $ is an \textbf{inverse system} if for all $ i, j \in J $ there exists $ k \in J $ such that $ k \le i $ and $ k \le j $. An \textbf{inverse system of groups} consists of an inverse system $ \br{J, \le} $ and a diagram of shape $ \JJJ $ in $ \Grp $, so $ G : \JJJ \to \Grp $. Thus an inverse system is a group $ G_j $ for all $ j \in J $ and transition maps $ \phi_{ij} : G_i \to G_j $ if $ i \le j $ such that $ \phi_{ii} = \id $ and $ \phi_{jk} \circ \phi_{ij} = \phi_{ik} $ for all $ i \le j \le k $. The \textbf{inverse limit} of this inverse system of groups $ G_j $ is the limit of this diagram, denoted $ \varprojlim_j G_j $.
\end{definition*}

\begin{definition*}
A \textbf{profinite group} is the inverse limit of an inverse system of groups, all of which are finite.
\end{definition*}

\begin{proposition}
Let $ \br{G_j}_{j \in J} $ be an inverse system of groups. Then the inverse limit exists, and is given by the explicit description
$$ \varprojlim_j G_j = \cbr{\br{g_j}_{j \in J} \in \prod_{j \in J} G_j \st \forall i \le j, \ \phi_{ij}\br{g_i} = g_j}. $$
\end{proposition}

\begin{proof}
This is a group. We have $ p_j : \varprojlim_j G_j \to G_j $, restricted from $ \prod_{j \in J} G_j \to G_j $, so $ g_j = p_j\br{\br{g_j}_{j \in J}} $. Take a cone $ Z $ on the system. Define
$$ \function[f]{Z}{\varprojlim_j G_j}{z}{\br{q_j\br{z}}_{j \in J}}. $$
Then $ \phi_{ij}\br{q_i\br{z}} = q_j\br{z} $, so
$$
\begin{tikzcd}
& Z \arrow[dashed]{d}{\exists !f} \arrow[bend right=30]{ddl}[swap]{q_i} \arrow[bend left=30]{ddr}{q_j} & \\
& \displaystyle\varprojlim_j G_j \arrow{dl}[swap]{p_i} \arrow{dr}{p_j} & \\
G_i \arrow{rr} & & G_j
\end{tikzcd}.
$$
\end{proof}

\pagebreak

\lecture{3}{Tuesday}{27/01/21}

\begin{definition*}
Let $ \br{G_j}_{j \in J} $ be an inverse system of finite groups. Give each $ G_j $ the discrete topology. Give $ \prod_j G_j $ the product topology. Then $ \varprojlim_j G_j \le \prod_j G_j $ gets the subspace topology.
\end{definition*}

\begin{proposition}
$ \varprojlim_j G_j $ is compact Hausdorff.
\end{proposition}

\begin{proof}
$ \prod_j G_j $ is Hausdorff and compact, by Tychonoff's theorem. Each condition $ \phi_{ij}\br{g_i} = g_j $ is a closed condition, since $ \prod_{j \in J} G_j \to G_i \times G_j $, so $ \varprojlim_j G_j $ is closed in $ \prod_j G_j $.
\end{proof}

\begin{proposition}
Let $ \br{X_j}_{j \in J} $ be an inverse system of non-empty finite sets. Then $ \varprojlim_j X_j $ is non-empty.
\end{proposition}

\begin{proof}
Use the finite intersection property. Let $ I_1 \subseteq J $ be a finite subset. Define
$$ Y_{I_1} = \cbr{\br{x_j} \in \prod_j X_j \st \forall i, j \in I_1, \ \forall i \le j, \ \phi_{ij}\br{x_i} = x_j} \subseteq \prod_j X_j, $$
a closed subset of the product. Since $ J $ is an inverse system and $ I_1 $ is finite, there exists $ k \in J $ such that $ k \le i $ for all $ i \in I_1 $. Choose $ x_k \in X_k \ne \emptyset $. Define $ x_j = \phi_{kj}\br{x_k} $ for all $ j \ge k $. Choose $ x_j $ arbitrarily elsewhere. This gives $ x = \br{x_j} \in \prod_{j \in J} X_j $, which lies in $ Y_{I_1} $, since if $ i, j \in I_1 $ such that $ i \le j $ then
$$ x_j = \phi_{kj}\br{x_k} = \phi_{ij}\br{\phi_{ki}\br{x_k}} = \phi_{ij}\br{x_i}. $$
So $ Y_{I_1} $ is non-empty. Then $ Y_{I_1} \cap \dots \cap Y_{I_n} \supseteq Y_{I_1 \cup \dots \cup I_n} \ne \emptyset $. By the finite intersection property, since $ \prod_j X_j $ is compact, $ \bigcap_{I_1} Y_{I_1} = \varprojlim_j X_j $ is non-empty.
\end{proof}

\begin{proposition}
Let $ J $ be a countable set and let $ \br{X_j}_{j \in J} $ be a family of finite sets. Then $ X = \prod_{j \in J} X_j $ is \textbf{metrisable}, so the metric topology equals to the other topology.
\end{proposition}

\begin{proof}
Without loss of generality $ J = \NN $. Give each $ X_n $ the discrete metric $ d_n $, where
$$ d_n\br{x_n, y_n} =
\begin{cases}
0 & x_n = y_n \\
1 & x_n \ne y_n
\end{cases},
\qquad x_n, y_n \in X_n. $$
Define
$$ d\br{\br{x_n}, \br{y_n}} = \sum_{n = 1}^\infty \dfrac{1}{3^n}d_n\br{x_n, y_n}, \qquad \br{x_n}, \br{y_n} \in \prod_n X_n. $$
We need to show this gives the product topology. Let $ f : \br{X, \tau_{\text{product}}} \to \br{X, d} $ be the identity function. A basis for the metric topology are open balls $ \B\br{x, 1 / 3^n} $ for $ x \in X $ and $ n \in \NN $. Then $ d\br{\br{x_n}, \br{y_n}} < 1 / 3^m $ if and only if $ x_n = y_n $ for all $ n \le m $, and
$$ f^{-1}\br{\B\br{\br{x_n}, \dfrac{1}{3^m}}} = \cbr{\br{y_n} \st \forall n \le m, \ y_n = x_n} = \bigcap_{n = 1}^m p_n^{-1}\br{\cbr{x_n}}, \qquad p_n : \prod_n X_n \to X_n $$
is open in the product topology. So $ f $ is continuous, so a homeomorphism.
\end{proof}

\begin{proposition}
A continuous bijection from a compact space to a Hausdorff space is a homeomorphism.
\end{proposition}

\begin{lemma}
\label{lem:1.2.18}
Let $ G $ be a finitely generated group. For each $ n \in \NN $, there are only finitely many subgroups of index $ n $.
\end{lemma}

\begin{proof}
For a subgroup $ H \le G $ of index $ n $, we get a homomorphism $ G \to \SSS_n $, since by labelling cosets $ H, \dots, g_nH $ by symbols $ 1, \dots, n $, $ G $ permutes these right cosets by $ g \cdot g_iH = \br{gg_i}H $ and $ H $ is recovered from this as the stabiliser of $ 1 $. So there are at most as many subgroups $ H $ as homomorphisms to $ \SSS_n $, and there are only finitely many.
\end{proof}

\begin{corollary}
If $ G $ is finitely generated, the inverse system $ \NNN = \cbr{N \triangleleft_\f G} $ is countable.
\end{corollary}

\pagebreak

\begin{proposition}
Let $ G $ be a profinite group. Then $ G $ is a \textbf{topological group}, so
$$ \function[\m]{G \times G}{G}{\br{g, h}}{gh}, \qquad \function[\i]{G}{G}{g}{g^{-1}} $$
are continuous.
\end{proposition}

\begin{definition*}
Let $ G $ and $ H $ be topological groups. We say $ G $ and $ H $ are \textbf{isomorphic as topological groups} if and only if there exists $ f : G \to H $ which is both an isomorphism of groups and a homeomorphism.
\end{definition*}

Recall that if $ G $ and $ H $ are profinite, this is the same as there exists $ f $ a continuous isomorphism.

\begin{proposition}
\label{prop:1.2.25}
Let $ H $ be a topological group and $ G = \varprojlim_j G_j $ be an inverse limit of finite groups. Let $ p_j : G \to G_j $ be the projection maps. A homomorphism $ f : H \to G $ is continuous if and only if each map $ f_j = p_j \circ f $ is continuous.
\end{proposition}

\begin{proof}
$ f : H \to G \le \prod_j G_j $. This is continuous if and only if all $ f_j $ are continuous, by definition of the product topology.
\end{proof}

\begin{proposition}
Let $ f : H \to G_j $ be a homomorphism from a topological group to a finite group, with the discrete topology. Then $ f $ is continuous if and only if $ \ker f $ is open in $ H $.
\end{proposition}

\begin{proof}
If $ f $ is continuous then $ \ker f = f^{-1}\br{\cbr{1}} $ is open. Assume $ f^{-1}\br{\cbr{1}} $ is open. Then $ f^{-1}\br{\cbr{g}} $ is open for all $ g \in G $, since multiplication is continuous and $ f^{-1}\br{\cbr{g}} = hf^{-1}\br{\cbr{1}} $ for some $ h \in H $. Taking unions, the preimage of any set in $ G_j $ is open in $ H $, so $ f $ is continuous.
\end{proof}

\begin{proposition}
Let $ G $ be a compact topological group. A subgroup of $ G $ is open if and only if it is closed and of finite index.
\end{proposition}

\begin{proposition}
\label{prop:1.2.28}
Let $ \br{G_j}_{j \in J} $ be an inverse system of finite groups. If $ G = \varprojlim_j G_j $, then the open subgroups $ U_j = \ker \br{p_j : G \to G_j} $ form a \textbf{basis of open neighbourhoods} of the identity $ 1 \in G $, so if $ V \subseteq G $ is any open set with $ 1 \in V $, then there exists $ j $ such that $ U_j \subseteq V $.
\end{proposition}

\begin{proof}
Let $ V \ni 1 $ be open. By definition of the product topology,
$$ V \supseteq p_{j_1}^{-1}\br{X_{j_1}} \cap \dots \cap p_{j_n}^{-1}\br{X_{j_n}} \supseteq p_{j_1}^{-1}\br{\cbr{1}} \cap \dots \cap p_{j_n}^{-1}\br{\cbr{1}} = U_{j_1} \cap \dots \cap U_{j_n}. $$
for $ X_{j_i} \subseteq G_{j_i} $. There exists $ k $ such that $ k \le j_i $. Since $ p_{j_i} = \phi_{kj_i} \circ p_k $, $ \ker p_k = U_k \subseteq U_{p_{j_i}} = \ker p_{j_i} $ for all $ i $. Thus $ V \supseteq U_k $.
\end{proof}

\lecture{4}{Thursday}{28/01/21}

\begin{corollary}
If $ g = \br{g_j}_{j \in J} \in G $, then the open cosets $ gU_j = p_j^{-1}\br{\cbr{g_j}} $ form a neighbourhood base at $ g $, so for all open set $ V \ni g $, there exists $ j \in J $ such that $ gU_j \subseteq V $.
\end{corollary}

\begin{proof}
Continuity of multiplication.
\end{proof}

\begin{corollary}
\label{cor:1.2.30}
A subset $ X \subseteq G $ is dense if and only if $ p_j\br{X} = p_j\br{G} $ for all $ j \in J $.
\end{corollary}

\begin{proof}
Suppose $ X $ is not dense. There exists a non-empty open set $ V $ such that $ V \cap X = \emptyset $. Pick $ g \in V $. There exists $ j \in J $ such that $ p_j^{-1}\br{\cbr{g_j}} = gU_j \subseteq V $, where $ g_j = p_j\br{g} $. Then $ g_j \in p_j\br{G} $. But for any $ x \in X $, $ p_j\br{x} \ne g_j $, otherwise $ x \in p_j^{-1}\br{\cbr{g_j}} = gU_j \subseteq V $, so $ p_j\br{X} \ne p_j\br{G} $. Assume $ X $ is dense. Then $ p_j\br{X} \subseteq p_j\br{G} $ is obvious. If $ g_j \in p_j\br{G} $, then $ p_j^{-1}\br{\cbr{g_j}} $ is a non-empty open set, so there exists $ x \in X \cap p_j^{-1}\br{\cbr{g_j}} $, then $ p_j\br{x} = g_j $. So $ g_j \in p_j\br{X} $, so $ p_j\br{X} = p_j\br{G} $.
\end{proof}

\begin{corollary}
\label{cor:1.2.31}
Let $ Y $ be a compact topological space and let $ f : Y \to G $ be a continuous function. Then $ f $ is surjective if and only if $ p_j\br{f\br{Y}} = p_j\br{G} $ for all $ j \in J $.
\end{corollary}

\begin{proof}
$ p_j\br{f\br{Y}} = p_j\br{G} $ if and only if $ f\br{Y} $ is dense, if and only if $ f\br{Y} = G $, since $ f\br{Y} $ is closed.
\end{proof}

\pagebreak

\begin{proposition}
Let $ G $ be a profinite group and $ X \subseteq G $ be a subset. Then the closure of $ X $ is
$$ \overline{X} = \bigcap_{N \le_\o G} XN, $$
where $ N \le_\o G $ are open subgroups.
\end{proposition}

\begin{proof}
$ XN $ is a union of cosets, hence it is open and closed in $ G $. So $ \overline{X} \subseteq XN $ for all $ N \le_\o G $, so $ \overline{X} \subseteq \bigcap_{N \le_\o G} XN $. Take $ g \notin \overline{X} $. There exists an open $ V \subseteq G $ such that $ g \in V $ but $ X \cap V = \emptyset $. Then there exists $ j \in J $ such that $ V \supseteq gU_j $ for $ N = U_j = \ker p_j $. Then $ g \notin XN $, since if $ g = xn $ for $ x \in X $ and $ n \in N = U_j $ then $ x = gn^{-1} \in gN = gU_j \subseteq V $, a contradiction. Thus $ g \notin \bigcap_N XN $, so $ \bigcap_N XN \subseteq \overline{X} $.
\end{proof}

\begin{proposition}
\label{prop:1.2.33}
Let $ G $ be a profinite group and let $ \UUU $ be a collection of open normal subgroups which form a neighbourhood base at the identity. Then
$$ G \cong \varprojlim_{U \in \UUU} G / U, $$
as topological groups, where $ G / U $ are finite groups.
\end{proposition}

\begin{proof}
The quotient maps $ G \twoheadrightarrow G / U $ are a cone on the inverse system, so we get a well-defined homomorphism $ f : G \to \varprojlim_U G / U $. Then
\begin{itemize}
\item $ f $ is continuous, since compositions with projection maps are continuous,
\item $ f $ is surjective, since $ G \twoheadrightarrow G / U $ are surjective, and
\item $ f $ is injective, since if $ g \in G \setminus \cbr{1} $, there exists an open subset $ V $ such that $ 1 \in V $ and $ g \notin V $ and there exists $ U \in \UUU $ such that $ 1 \in U \subseteq V $, then $ g \notin \ker \br{G \to G / U} $, so $ g \notin \ker f $.
\end{itemize}
\end{proof}

\subsection{Change of inverse system}

\begin{definition*}
Let $ \br{J, \le} $ be an inverse system. A \textbf{cofinal subsystem} of $ J $ is a subset $ I \subseteq J $ such that for all $ j \in J $ there exists $ i \in I $ such that $ i \le j $.
\end{definition*}

Then $ I $ is an inverse system.

\begin{example*}
If $ k \in J $, then the set
$$ J_{\le k} = \cbr{j \in J \st j \le k}, $$
the \textbf{principal cofinal subsystem}, is cofinal in $ J $.
\end{example*}

\begin{proposition}
Let $ \br{G_j}_{j \in J} $ be an inverse system of finite groups, and let $ I \subseteq J $ be cofinal. Then $ H = \varprojlim_{i \in I} G_i $ is topologically isomorphic to $ G = \varprojlim_{j \in J} G_j $.
\end{proposition}

\begin{proof}
The projection map $ \prod_{j \in J} G_j \to \prod_{i \in I} G_i $ is a continuous homomorphism, and it restricts to $ f : G \to H $. Check that $ f $ is bijective.
\begin{itemize}
\item Injective. Take $ g = \br{g_j}_{j \in J} \in G $. Assume $ f\br{g} = 1 $, so $ g_i = p_i\br{f\br{g}} = 1 $ for all $ i \in I $. For any $ j \in J $, there exists $ i \in I $ such that $ i \le j $. Then $ g_j = \phi_{ij}\br{g_i} = \phi_{ij}\br{1} = 1 $. So $ g = 1 $.
\item Surjective. Let $ h = \br{h_i}_{i \in I} \in H $ for $ h_i \in G_i $. Define $ g = \br{g_j} \in \prod_{j \in J} G_j $ by setting $ g_j = \phi_{ij}\br{h_i} $ for some $ i \in I $ such that $ i \le j $. If $ i_1 \le j $ and $ i_2 \le j $, there exists $ i_0 \in I $ such that $ i_0 \le i_1 $ and $ i_0 \le i_2 $, then
$$ \phi_{i_1j}\br{h_{i_1}} = \phi_{i_1j}\br{\phi_{i_0i_1}\br{h_{i_0}}} = \phi_{i_0j}\br{h_{i_0}} = \phi_{i_2j}\br{\phi_{i_0i_2}\br{h_{i_0}}} = \phi_{i_2j}\br{h_{i_2}}. $$
It also follows that $ g \in G $, since if $ j_1 \le j_2 $, choose $ i \in I $ such that $ i \le j_1 $, then
$$ g_{j_2} = \phi_{ij_2}\br{h_i} = \phi_{j_1j_2}\br{\phi_{ij_1}\br{h_i}} = \phi_{j_1j_2}\br{g_{j_1}}. $$
Finally, $ f\br{g} = h $, since $ g_i = \phi_{ii}\br{h_i} = h_i $ for all $ i \in I $.
\end{itemize}
\end{proof}

\pagebreak

\begin{definition*}
An inverse system of groups is \textbf{surjective} if all transition maps are surjective.
\end{definition*}

\begin{proposition}
Let $ \br{X_j}_{j \in J} $ be an inverse system of finite sets where all transition maps are surjective. Then the projection maps $ p_j : \varprojlim_j X_j \to X_j $ are surjective.
\end{proposition}

\begin{proposition}
\label{prop:1.3.7}
Let $ \br{G_j}_{j \in J} $ be an inverse system of finite groups. Then there exists an inverse system $ \br{G_j'}_{j \in J} $ such that $ G_j' \le G_j $, with surjective transition maps, such that $ \varprojlim_j G_j = \varprojlim_j G_j' $.
\end{proposition}

\begin{proof}
Let $ p_j : G = \varprojlim_j G_j \to G_j $ be the projection. Define $ G_j' = p_j\br{G} $. Since $ \phi_{ij} \circ p_i = p_j $, $ \br{G_j'} $ is an inverse system with $ \eval{\phi_{ij}}_{G_i'} : G_i' \to G_j' $, and $ \eval{\phi_{ij}}_{G_i'} $ is surjective. If $ g = \br{g_j} \in G $ then $ g_j = p_j\br{g} \in G_j' $, so $ g \in \varprojlim_j G_j' \le G \le \prod_j G_j $. Thus $ \varprojlim_j G_j' = G $.
\end{proof}

\begin{definition*}
An inverse system $ \br{J, \le} $ is \textbf{linearly ordered} if there exists a bijection $ f : J \to \NN $ such that $ i \le j $ if and only if $ f\br{i} \ge f\br{j} $, the \textbf{wrong-way ordering} on $ \NN $.
\end{definition*}

Thus cofinal if and only if increasing subsequence.

\begin{proposition}
\label{prop:1.3.11}
If $ J $ is a countable inverse system, with no \textbf{global minimum}, so there does not exist $ m \in J $ such that $ m \le j $ for all $ j $, then $ J $ has a linearly ordered cofinal subsystem.
\end{proposition}

\pagebreak

\section{Profinite groups}

\subsection{The \texorpdfstring{$ p $}{p}-adic integers}

\lecture{5}{Saturday}{30/01/21}

Let $ p $ be a prime. Consider
$$ \dots \to \ZZ / p^2\ZZ \to \ZZ / p\ZZ \to 1. $$
The \textbf{ring of $ p $-adic integers} is
$$ \ZZ_p = \varprojlim_{n \in \NN} \ZZ / p^n\ZZ. $$
Thus $ \alpha \in \ZZ_p $ is a sequence $ \br{a_n}_{n \in \NN} $ of integers modulo $ p^n $ for $ a_n \in \ZZ / p^n\ZZ $ such that $ a_n \equiv a_m \mod p^m $ whenever $ n \ge m $, since $ \ZZ / p^n\ZZ \to \ZZ / p^m\ZZ $, and
$$ \function[p_n]{\ZZ_p}{\ZZ / p^n\ZZ}{\alpha}{a_n = \alpha \mod p^n}. $$
Given $ a \in \ZZ $, setting $ a_n = a \mod p^n $ gives an element $ \iota\br{a} \in \ZZ_p $ for $ \iota : \ZZ \to \ZZ_p $. Then $ \iota $ is injective, since if $ a \in \ZZ $, and $ p^n > \abs{a} $ then $ a \not\equiv 0 \mod p^n $, so $ \iota\br{a} \ne 0 $ in $ \ZZ_p $. Often $ \ZZ \le \ZZ_p $.

\begin{definition*}
Let $ \alpha = \br{a_n}, \beta = \br{b_n} \in \ZZ_p $. If $ \alpha = \beta $ then $ \d\br{\alpha, \beta} = 0 $. If $ \alpha \ne \beta $, take the smallest $ n $ such that $ a_n \ne b_n $, and set $ \d\br{\alpha, \beta} = p^{-n} $, the \textbf{$ p $-adic metric on $ \ZZ_p $}. The restriction of $ \d $ to $ \iota\br{\ZZ} $ is the \textbf{$ p $-adic metric on $ \ZZ $}.
\end{definition*}

Thus $ \alpha $ and $ \beta $ are close if $ \br{a_n} $ and $ \br{b_n} $ agree modulo $ p^n $ for all but large $ n $. Since
$$ \B\br{0, r} = \cbr{\alpha = \br{a_n} \st \forall n \le -\log_p r, \ a_n = 0} = \ker \br{\ZZ_p \to \ZZ / p^{\fbr{-\log_p r}}\ZZ}, $$
open balls are the subgroups $ p^n\ZZ_p \le \ZZ_p $.

\begin{itemize}
\item $ \iota\br{\ZZ} $ is dense in this metric. Let $ \alpha = \br{a_n} \in \ZZ_p $ and $ \epsilon > 0 $. Take $ n > -\log_p \epsilon $, and choose $ a \in \ZZ $ such that $ a \equiv a_n \mod p^n $. Then $ \d\br{\alpha, \iota\br{a}} \le p^{-n} < \epsilon $.
\item The $ p $-adic metric on $ \ZZ $ is not complete, since $ a_n = 1 + \dots + p^n $ does not converge in $ \ZZ $, but does converge in $ \ZZ_p $.
\item The $ p $-adic metric on $ \ZZ_p $ is complete. Let $ \alpha^{\br{k}} = \br{a_n^{\br{k}}}_{n \in \NN} $ be a Cauchy sequence in $ \ZZ_p $. For all $ n $ there exists $ K_n $ such that for all $ k, l \ge K_n $, we have $ \d\br{\alpha^{\br{k}}, \alpha^{\br{l}}} \le p^{-n} $, so $ a_n^{\br{k}} = a_n^{\br{l}} $ for all $ k, l \ge K_n $ so for fixed $ n $, $ a_n^{\br{k}} $ is eventually a constant $ b_n $. Then $ \beta = \br{b_n} \in \ZZ_p $, and $ \alpha^{\br{k}} \to \beta $ in $ \ZZ_p $.
\end{itemize}

Thus $ \ZZ_p $ is a completion of $ \ZZ $, but is not the profinite completion of $ \ZZ $.

\begin{definition*}
Let $ p $ be a prime. A \textbf{$ p $-group} is a finite group of order $ p^n $ for $ n \ge 0 $. A \textbf{pro $ p $-group} is an inverse limit of $ p $-groups.
\end{definition*}

\begin{definition*}
Let $ G $ be a group and $ p $ prime. The set of normal subgroups $ N \triangleleft G $ such that $ \sbr{G : N} = p^n $ for some $ n $ form an inverse system $ \NNN_p $. Since $ G / N_1 \times G / N_2 $ are $ p $-groups, $ N_1 \cap N_2 = \ker \br{G \to G / N_1 \times G / N_2} $ is a $ p $-group. The \textbf{pro-$ p $ completion} is
$$ \widehat{G_{\br{p}}} = \varprojlim_{N \in \NNN_p} G / N, $$
where $ G / N_1 \to G / N_2 $ if $ N_1 \le N_2 $.
\end{definition*}

\begin{proposition}
The additive group $ \ZZ_p $ is abelian and torsionfree.
\end{proposition}

\begin{proof}
$ \ZZ_p \le \prod_{n \in \NN} \ZZ / p^n\ZZ $ is abelian. Let $ \alpha = \br{a_n} \in \ZZ_p \setminus \cbr{0} $. Suppose $ m\alpha = 0 $ for $ m \in \ZZ $. We want $ m = 0 $. Assume $ m = p^rs $ for $ s $ coprime to $ p $. Then $ \alpha \ne 0 $, so there exists $ n $ such that $ a_n \ne 0 $. Consider $ a_{n + r} $. Then $ 0 \equiv ma_{n + r} \equiv p^ra_{n + r}s \mod p^{n + r} $, so $ p^n \mid a_{n + r}s $. Thus $ p^n \mid a_{n + r} $, so $ a_n \equiv a_{n + r} \equiv 0 \mod p^n $, a contradiction.
\end{proof}

\pagebreak

\begin{proposition}
The ring $ \ZZ_p $ has no zero-divisors.
\end{proposition}

\begin{proof}
Exercise. \footnote{Exercise}
\end{proof}

\subsection{The profinite completion of the integers}

The \textbf{profinite completion of the integers} is
$$ \widehat{\ZZ} = \varprojlim_n \ZZ / n\ZZ, $$
where $ \ZZ / n\ZZ \to \ZZ / m\ZZ $ whenever $ n\ZZ \le m\ZZ $, which is if and only if $ m \mid n $, so $ n = mr $.

\begin{theorem}[Chinese remainder theorem]
There is an isomorphism of topological rings
$$ \widehat{\ZZ} \cong \prod_{p \ \text{prime}} \ZZ_p. $$
\end{theorem}

\begin{proof}
Each natural number $ n $ is written as a product of prime powers $ n = \prod_{p \ \text{prime}} p^{\e_p\br{n}} $. The classical CRT gives natural isomorphisms
$$ \function[f_n]{\ZZ / n\ZZ}{\prod_{p \ \text{prime}} \ZZ / p^{\e_p\br{n}}\ZZ}{1}{\br{1, \dots, 1}}, $$
and commutative diagrams
$$
\begin{tikzcd}
\ZZ / mn\ZZ \arrow{r}{f_{mn}}[swap]{\sim} \arrow[twoheadrightarrow]{d} & \displaystyle\prod_p \ZZ / p^{\e_p\br{mn}}\ZZ \arrow[twoheadrightarrow]{d} \\
\ZZ / n\ZZ \arrow{r}{\sim}[swap]{f_n} & \displaystyle\prod_p \ZZ / p^{\e_p\br{n}}\ZZ
\end{tikzcd}.
$$
Passing to inverse limits,
$$
\begin{tikzcd}[row sep=tiny]
\widehat{\ZZ} = \displaystyle\varprojlim_n \ZZ / n\ZZ \arrow{r}{\sim} \arrow[subset]{d} & \displaystyle\varprojlim_n \prod_p \ZZ / p^{\e_p\br{n}}\ZZ \arrow[subset]{d} \\
\displaystyle\prod_n \ZZ / n\ZZ \arrow{r}[swap]{\sim} & \displaystyle\prod_n \prod_p \ZZ / p^{\e_p\br{n}}\ZZ
\end{tikzcd}.
$$
The natural continuous surjections
$$ \prod_p \ZZ_p \twoheadrightarrow \prod_p \ZZ / p^{\e_p\br{n}}\ZZ $$
form a cone on the inverse system $ \cbr{\prod_p \ZZ / p^{\e_p\br{n}}\ZZ} $, so there exists
$$ f : \prod_p \ZZ_p \twoheadrightarrow \varprojlim_n \prod_p \ZZ / p^{\e_p\br{n}}\ZZ, $$
which is continuous by Proposition \ref{prop:1.2.25}, surjective by Corollary \ref{cor:1.2.31}, and injective since every non-trivial element of $ \prod_p \ZZ_p $ is non-trivial in some quotient $ \ZZ / p^e\ZZ $. So $ f $ is a topological isomorphism as required.
\end{proof}

\begin{corollary}
The abelian group $ \widehat{\ZZ} $ is torsionfree abelian.
\end{corollary}

\begin{corollary}
The ring $ \widehat{\ZZ} $ is not an integral domain.
\end{corollary}

\begin{proof}
Any product of non-trivial rings $ R_1 \times R_2 $ has zero-divisors, since $ \br{r_1, 0} \cdot \br{0, r_2} = \br{0, 0} $. An element of $ \widehat{\ZZ} $ is a zero-divisor if and only if it is zero in some $ \ZZ_p $-factor.
\end{proof}

Elements of $ \iota\br{\ZZ} $ are not zero divisors in $ \widehat{\ZZ} $.

\pagebreak

\subsection{Profinite matrix groups}

\lecture{6}{Tuesday}{02/02/21}

For a commutative ring $ R $, we have
$$ \Mat_{N \times M} R = \cbr{N \times M \ \text{matrices with elements in} \ R}. $$
If $ N = M $, we have a ring structure, where addition and multiplication are given by the usual formula. There exists a determinant function $ \det : \Mat_{N \times N} R \to R $. Then
$$ \ZZ_p^{NM} \cong \Mat_{N \times M} \ZZ_p = \varprojlim_{n \in \NN} \Mat_{N \times M} \ZZ / p^n\ZZ. $$
By continuity of ring operations on $ \ZZ_p $, addition and multiplication on matrices are continuous, and $ \det : \Mat_{N \times N} \ZZ_p \to \ZZ_p $ is continuous. Since $ \ZZ_p $ is an integral domain, it has a field of fractions $ \QQ_p $, so you can do linear algebra over $ \QQ_p $. A matrix over $ \QQ_p $ has an inverse over $ \QQ_p $ if and only if its determinant is non-zero, and a matrix over $ \ZZ_p $ has an inverse over $ \ZZ_p $ if and only if its determinant and its inverse are in $ \ZZ_p^\times $. Define
$$ \GL_N \ZZ_p = \cbr{A \in \Mat_{N \times N} \ZZ_p \st \det A \in \ZZ_p^\times}, \qquad \SL_N \ZZ_p = \cbr{A \in \Mat_{N \times N} \ZZ_p \st \det A = 1}. $$
Both are profinite groups.

\begin{lemma}
For all $ N \ge 1 $ and $ p $ prime,
$$ \GL_N \ZZ_p = \varprojlim_n \GL_N \ZZ / p^n\ZZ, \qquad \SL_N \ZZ_p = \varprojlim_n \SL_N \ZZ / p^n\ZZ. $$
\end{lemma}

\begin{proof}
The diagrams
$$
\begin{tikzcd}
\Mat_{N \times N} \ZZ_p \arrow{r} \arrow{d}[swap]{\det} & \Mat_{N \times N} \ZZ / p^n\ZZ \arrow{d}{\det} \\
\ZZ_p \arrow{r} & \ZZ / p^n\ZZ
\end{tikzcd}
$$
commute.
\begin{itemize}
\item $ A \in \GL_N \ZZ_p $ if and only if $ \det A \in \ZZ_p^\times $, if and only if $ \det A_n \in \br{\ZZ / p^n\ZZ}^\times $ for all $ n $, if and only if $ A_n \in \GL_N \ZZ / p^n\ZZ $ for all $ n $.
\item $ A \in \SL_N \ZZ_p $ if and only if $ \det A = 1 $, if and only if $ \det A_n = 1 $ for all $ n $, if and only if $ A_n \in \SL_N \ZZ / p^n\ZZ $ for all $ n $.
\end{itemize}
\end{proof}

Also have matrices over $ \widehat{\ZZ} $. A warning is that $ \widehat{\ZZ} $ is not an integral domain. Analogously,
$$ \GL_N \widehat{\ZZ} = \cbr{A \in \Mat_{N \times N} \widehat{\ZZ} \st \det A \in \widehat{\ZZ}^\times} = \varprojlim_n \GL_N \ZZ / n\ZZ = \prod_p \GL_N \ZZ_p, $$
$$ \SL_N \widehat{\ZZ} = \cbr{A \in \Mat_{N \times N} \widehat{\ZZ} \st \det A = 1} = \varprojlim_n \SL_N \ZZ / n\ZZ = \prod_p \SL_N \ZZ_p, $$
since $ \Mat_{N \times N} \widehat{\ZZ} = \prod_p \Mat_{N \times N} \ZZ_p $, and
$$ \SL_N \ZZ \le \SL_N \ZZ_p, \qquad \SL_N \ZZ \le \SL_N \widehat{\ZZ} = \varprojlim_n \SL_N \ZZ / n\ZZ $$
are dense. See problem sheet $ 2 $.

\begin{example*}
$ \twobytwosmall{7}{9}{4}{9} \in \SL_2 \ZZ / 13\ZZ $ is in the image of $ \SL_2 \ZZ $.
\end{example*}

\pagebreak

\subsection{Subgroups, quotients, and homomorphisms}

\begin{proposition}
A closed subgroup of a profinite group is a profinite group.
\end{proposition}

\begin{proof}
Let $ G = \varprojlim_{j \in J} G_j $ be a profinite group for $ G_j $ finite. Take a closed subgroup $ H \le_\c G $ of $ G $. Define $ H_j = p_j\br{H} \le G_j $. Then $ H_j $, with transition maps $ \eval{\phi_{ij}}_{H_i} : H_i \to H_j $, are an inverse system of finite groups. Define
$$ H' = \varprojlim_j H_j = \cbr{\br{g_j} \in \prod_{j \in J} G_j \st \forall i \le j, \ \phi_{ij}\br{g_i} = g_j, \ g_j \in H_j}. $$
Show that $ H = H' $. If $ h = \br{h_j} \in H $, by definition $ h_j = p_j\br{h} \in H_j $, so $ H \le H' $. Suppose $ g = \br{g_j} \notin H $. Since $ H $ is closed, $ G \setminus H $ is open, so there exists a basic open set containing $ g $, which does not intersect $ H $. There exists $ j \in J $ such that $ gU_j = p_j^{-1}\br{\cbr{g_j}} \le G \setminus H $. Therefore for all $ h \in H $, $ p_j\br{h} \ne g_j $, since then $ h \in H \cap p_j^{-1}\br{\cbr{g_j}} $, so $ g_j \notin H_j $, so $ g \notin H' $. So $ H = H' $.
\end{proof}

\begin{remark*}
\hfill
\begin{itemize}
\item The two topologies on $ H $ agree by $ \id : \br{H, \tau_{\text{profinite}}} \to \br{H, \tau_{\text{subspace}}} $, which is continuous by Proposition \ref{prop:1.2.25}.
\item A better name for $ H' $ is $ \overline{H} $, the closure. Actually proved that $ H' = \overline{H} = H $.
\end{itemize}
\end{remark*}

\begin{proposition}
Let $ G = \varprojlim_j G_j $ and $ H \le G $. Set $ H_j = p_j\br{H} \le G_j $. Then the closure of $ H $ is $ \overline{H} = \varprojlim_j H_j $.
\end{proposition}

\begin{lemma}
Let $ f : G_1 \to G_2 $ be a surjective homomorphism and $ H \le G_1 $. Then $ \sbr{G_1 : H} \ge \sbr{G_2 : f\br{H}} $.
\end{lemma}

\begin{proposition}
Let $ G = \varprojlim_j G_j $ for $ \br{G_j} $ a surjective inverse system, so $ G \twoheadrightarrow G_j $. Let $ H \le_\c G $ and set $ H_j = p_j\br{H} \le G_j $. Then $ H $ is finite index if and only if $ \sbr{G_j : H_j} $ is constant on a cofinal subsystem, if and only if $ \sbr{G_j : H_j} $ is bounded for all $ j $. If this is true, then $ \sbr{G : H} = \sbr{G_i : H_i} $ for $ i \in I $.
\end{proposition}

\begin{proof}
$ p_j : G \to G_j $ are surjective, so $ \sbr{G : H} \ge \sbr{G_j : H_j} $. Suppose $ \sbr{G : H} \ge N $. There exist distinct cosets $ g_1H, \dots, g_NH $ of $ H $ in $ G $, if and only if $ g_n^{-1}g_m \notin H $ if $ n \ne m $, so there exists $ j_{n, m} \in J $ such that $ p_{j_{n, m}}\br{g_n^{-1}g_m} \notin H_{j_{n, m}} $. Take $ k \le j_{n, m} $ for all $ n $ and $ m $. Then $ p_k\br{g_n^{-1}g_m} \notin H_k $ for all $ n \ne m $, so $ p_k\br{g_n}H_k $ are distinct cosets of $ H_k $ in $ G_k $, so $ \sbr{G_k : H_k} \ge N $. For any $ i $ in the cofinal subsystem $ J_{\le k} $, it follows $ \sbr{G_i : H_i} \ge N $ for all $ i \le k $. If $ \sbr{G : H} = N $ is finite, take $ k $ as above and $ I = J_{\le k} $. Then $ \sbr{G : H} \ge \sbr{G_i : H_i} \ge N = \sbr{G : H} $ for all $ i \in I $. If $ \sbr{G : H} $ is infinite, assume $ I $ is cofinal and $ \sbr{G_i : H_i} = N $ for all $ i \in I $. Then there exists $ k $ such that $ \sbr{G_k : H_k} \ge N + 1 $. But there exists $ i \in I $ such that $ i \le k $, then $ \sbr{G_i : H_i} \ge \sbr{G_k : H_k} \ge N + 1 > N = \sbr{G_i : H_i} $, a contradiction.
\end{proof}

\begin{proposition}
Let $ G $ be a profinite group and $ N $ a closed normal subgroup. Then $ G / N $, with the quotient topology, is a profinite group.
\end{proposition}

\begin{proof}
Take $ G = \varprojlim_j G_j $ for $ \br{G_j} $ a surjective inverse system. Let $ N_j = p_j\br{N} \triangleleft G_j = p_j\br{G} $. Recall $ N = \varprojlim_j N_j $. Define $ Q_j = G_j / N_j $. Since $ \phi_{ij}\br{N_i} \le N_j $, we get quotient homomorphisms $ \psi_{ij} : Q_i \to Q_j $, which are transition maps for the $ Q_j $. Set $ Q = \varprojlim_j Q_j $. The map $ \prod_h G_j \to \prod_j Q_j $ is continuous, so there is a continuous surjective group homomorphism $ f : G \to Q $. The kernel of this map is $ N $, since $ f\br{g} = 1 $ if and only if $ q_j\br{f\br{g}} = 1 $ for all $ j $, if and only if $ g_j \in N_j $ for all $ j $, if and only if $ g \in \varprojlim_j N_j = N $. By the first isomorphism theorem for groups,
$$
\begin{tikzcd}
G \arrow{d} \arrow{dr} & \\
G / N \arrow{r}{\sim}[swap]{\overline{f}} & Q
\end{tikzcd}.
$$
Since $ G \to Q $ is continuous and $ G \to G / N $ is the quotient map, $ \overline{f} $ is continuous. Since $ G / N $ is compact and $ Q $ is Hausdorff, $ \overline{f} $ is a homeomorphism.
\end{proof}

This is the first isomorphism theorem for profinite groups.

\pagebreak

\lecture{7}{Thursday}{04/02/21}

\begin{definition*}
Let $ \br{G_j}_{j \in J} $ and $ \br{H_j}_{j \in J} $ be inverse systems of finite groups, over the same poset $ J $. A \textbf{morphism of inverse systems} $ \br{f_j} $ is a family of homomorphisms $ f_j : G_j \to H_j $, such that for all $ i \le j $,
$$
\begin{tikzcd}
G_i \arrow{r}{f_i} \arrow{d}[swap]{\phi_{ij}^G} & H_i \arrow{d}{\phi_{ij}^H} \\
G_j \arrow{r}[swap]{f_j} & H_j
\end{tikzcd}
$$
commutes, so $ \phi_{ij}^H \circ f_i = f_j \circ \phi_{ij}^G $.
\end{definition*}

\begin{proposition}
Let $ \br{f_j} : \br{G_j} \to \br{H_j} $ be a morphism of inverse systems. Then there is a unique continuous homomorphism $ f : G = \varprojlim_j G_j \to H = \varprojlim_j H_j $ such that
$$
\begin{tikzcd}
G \arrow{r}{f} \arrow{d}[swap]{p_j^G} & H \arrow{d}{p_j^H} \\
G_j \arrow{r}[swap]{f_j} & H_j
\end{tikzcd},
$$
so $ p_j^H \circ f = f_j \circ p_j^G $ for all $ j \in J $.
\end{proposition}

\begin{proof}
The maps $ f_j \circ p_j^G : G \to H_j $ form a cone on the inverse system $ \br{H_j} $,
$$
\begin{tikzcd}
& G \arrow{dl}[swap]{f_i \circ p_i^G} \arrow{dr}{f_j \circ p_j^G} & \\
H_i \arrow{rr}[swap]{\phi_{ij}^H} & & H_j
\end{tikzcd},
$$
since $ \phi_{ij}^H \circ f_i \circ p_i^G = f_j \circ \phi_{ij}^G \circ p_i^G = f_j \circ p_j^G $. So by definition of limits, there exists a unique $ f : G \to H = \varprojlim_j H_j $ such that $ p_j^H \circ f = f_j \circ p_j^G $.
\end{proof}

Thus $ f $ is \textbf{induced} by the $ f_j $ by passing to an inverse limit.

\begin{proposition}
\label{prop:2.4.11}
Let $ G = \varprojlim_{j \in J} G_j $ and $ H = \varprojlim_{i \in I} H_i $ be inverse limits of finite groups, where $ I $ and $ J $ are countable inverse systems with no minimum element. Let $ f : G \to H $ be a continuous homomorphism. Then there exist cofinal subsystems $ J' \subseteq J $ and $ I' \subseteq I $, an order-preserving bijection $ J' \cong I' $, and a morphism of inverse systems $ \br{f_j} : \br{G_j}_{j \in J'} \to \br{H_i}_{i \in I'} $ inducing $ f $.
\end{proposition}

\begin{proof}
Without loss of generality, use Proposition \ref{prop:1.3.11} to assume $ J $ and $ I $ are linearly ordered. Without loss of generality both are $ \NN $, with the wrong-way ordering. Construct an increasing sequence $ \br{k_n} $ of natural numbers as follows. Each map $ p_n^H \circ f : G \to H \to H_n $ is a continuous homomorphism, so its kernel is open in $ G $. By Proposition \ref{prop:1.2.28} there exists $ k_n $ such that $ \ker \br{G \to G_{k_n}} \le \ker \br{G \to H_n} $, which means there is a quotient homomorphism
$$
\begin{tikzcd}
G \arrow{r}{f} \arrow{d}[swap]{p_{k_n}^G} \arrow{dr} & H \arrow{d}{p_n^H} \\
G_{k_n} \arrow{r}[swap]{f_n} & H_n
\end{tikzcd}.
$$
Then $ \ker \br{G \to G_{n + 1}} \le \ker \br{G \to G_n} $, so without loss of generality $ k_n > k_{n - 1} $. Now $ J' = \cbr{k_n}_{n \in \NN} $ give a cofinal subsystem of $ J = \NN $, and the $ f_n $ are the required morphisms of inverse systems.
\end{proof}

\subsection{Generators of profinite groups}

\begin{definition*}
Let $ G $ be a topological group, and let $ S $ be a subset of $ G $. Then $ S $ is a \textbf{topological generating set (TGS)} for $ G $ if the subgroup $ \abr{S} $ is dense in $ G $, and $ G $ is \textbf{topologically finitely generated (TFG)} if it has some finite TGS $ S $.
\end{definition*}

\pagebreak

\begin{definition*}
Let $ G $ be a topological group and $ S \subseteq G $. The \textbf{closed subgroup of $ G $ topologically generated by $ S $} is the smallest closed subgroup of $ G $ which contains $ S $. Denoted $ \overline{\abr{S}} $.
\end{definition*}

\begin{proposition}
Let $ G $ be a topological group and $ H $ a subgroup of $ G $. Then $ \overline{H} $ is a subgroup of $ G $. Hence for $ S \subseteq G $, the closed subgroup of $ G $ generated by $ S $ is equal to the closure of $ \abr{S} $.
\end{proposition}

\begin{proof}
Exercise. \footnote{Exercise}
\end{proof}

\begin{lemma}
A finite index subgroup of a finitely generated group is finitely generated.
\end{lemma}

\begin{proposition}
If a profinite group $ G $ is TFG and $ U $ is an open subgroup of $ G $ then $ U $ is TFG.
\end{proposition}

\begin{proof}
Let $ S $ be a finite set such that $ \abr{S} $ is dense in $ G $. Then $ \Gamma = U \cap \abr{S} $ is finite index in $ \abr{S} $, hence $ \Gamma $ is finitely generated, so $ \Gamma = \abr{S'} $ for $ S' $ finite. Since $ U $ is open, and $ \abr{S} $ is dense, $ \abr{S'} = U \cap \abr{S} $ is dense in $ U $. So $ U $ is TFG.
\end{proof}

\begin{proposition}
Let $ \br{G_j} $ be a surjective inverse system of finite groups with $ G = \varprojlim_j G_j $. Let $ S \subseteq G $. Then $ S $ is a TGS for $ G $ if and only if $ p_j\br{S} $ generates $ G_j $ for all $ j $.
\end{proposition}

\begin{proof}
By Corollary \ref{cor:1.2.30}, $ \abr{S} $ is dense in $ G $ if and only if $ G_j = p_j\br{\abr{S}} = \abr{p_j\br{S}} $ for all $ j $.
\end{proof}

\begin{lemma}
Let $ G $ be a TFG profinite group. Then $ G $ may be written as the inverse limit of a countable inverse system of finite groups.
\end{lemma}

\begin{proof}
A continuous homomorphism from $ G $ to a finite group is determined by the image of a TGS $ S $, since a function on $ S $ determines all of a homomorphism from $ \abr{S} $ and continuity gives the behaviour on all of $ G $. So there are only countably many continuous homomorphisms from $ G $ to $ \SSS_n $ for $ n \in \NN $. Every open normal subgroup of $ G $ is the kernel of such a continuous homomorphism. So there are only countably many open normal subgroups of $ G $. Then $ \UUU = \cbr{U \triangleleft_\o G} $ is a neighbourhood base of the identity, so by Proposition \ref{prop:1.2.33}, $ G = \varprojlim_{U \in \UUU} G / U $.
\end{proof}

\begin{example*}
Let $ G $ be a TFG profinite group. Then there are only finitely many open subgroups of $ G $ of index at most $ n $. See Lemma \ref{lem:1.2.18}. Define
$$ G_n = \bigcap \cbr{U \st U \le_\o G, \ \sbr{G : U} \le n}. $$
Then $ G_n \triangleleft G $, and $ G_n $ is open in $ G $. And $ \cbr{G_n} $ is a neighbourhood base of the identity. So $ G = \varprojlim_{n \in \NN} G / G_n $.
\end{example*}

\begin{proposition}
\label{prop:2.5.10}
Let $ \ZZ_p^\times $ be the set of elements $ \alpha $ of $ \ZZ_p $ which topologically generate $ \ZZ_p $. Then $ \alpha \in \ZZ_p^\times $ if and only if $ \alpha \not\equiv 0 \mod p $. Hence $ \ZZ_p^\times $ is a closed uncountable subset of $ \ZZ_p $. For every $ n $, and every generator $ a_n \in \br{\ZZ / p^n\ZZ}^\times $ there is some $ \alpha \in \ZZ_p^\times $ such that $ \alpha \equiv a_n \mod p^n $.
\end{proposition}

\begin{proof}
For the last part, $ a_n $ is the image of $ \alpha $, since it is a surjective inverse system, and if $ a_n $ generates $ \ZZ / p^n\ZZ $, it is coprime to $ p $. If $ \alpha = \br{a_n} $ such that $ a_1 \ne 0 $, then $ p \nmid a_n $ for any $ n $. Hence $ a_n $ is coprime to $ p $, and so generates $ \ZZ / p^n\ZZ $ for all $ n $. So $ \abr{\alpha} $ is dense in $ \ZZ_p $ by an earlier result.
\end{proof}

\lecture{8}{Saturday}{06/02/21}

\begin{remark*}
$ \ZZ_p^\times $ is the set of units in the ring $ \ZZ_p $.
\begin{itemize}[leftmargin=0.5in]
\item[$ \impliedby $] If $ \alpha $ is a unit, then $ \alpha \mod p^n $ is a unit in $ \ZZ / p^n\ZZ $, so generates $ \ZZ / p^n\ZZ $. Then $ \alpha $ topologically generates $ \ZZ_p $.
\item[$ \implies $] Consider the group homomorphism
$$ \function[f]{\ZZ_p}{\ZZ_p}{x}{\alpha x}, $$
which is continuous as multiplication in a ring is continuous. So $ \im f $ is a closed subgroup of $ \ZZ_p $, containing $ \alpha $. Then $ \alpha $ generates $ \ZZ_p $, so the only closed subgroup containing $ \alpha $ is $ \ZZ_p $ itself. So $ 1 \in \im f $, so there exists $ \beta $ such that $ \alpha\beta = 1 $.
\end{itemize}
Thus $ \alpha $ is a unit if and only if $ \cbr{\alpha} $ is a TGS for $ \ZZ_p $.
\end{remark*}

\pagebreak

\begin{example*}
If $ p \ne 2 $, then $ 2 $ is invertible in $ \ZZ_p $, so $ 2^{-1} $ exists. If $ p = 3 $,
$$ 2^{-1} = \br{\dots, 5, 2} \in \ZZ_3 \le \prod_{n \in \NN} \ZZ / 3^n\ZZ. $$
\end{example*}

\begin{proposition}
$ \alpha \in \widehat{\ZZ}^\times $ if and only if $ \alpha \mod n \in \br{\ZZ / n\ZZ}^\times $ for all $ n $. For any $ n $, and every $ k \in \br{\ZZ / n\ZZ}^\times $ there exists a generator $ \alpha \in \widehat{\ZZ}^\times $ such that $ \alpha \equiv k \mod n $.
\end{proposition}

\begin{proof}
Follows from Proposition \ref{prop:2.5.10} via the CRT, since $ \widehat{\ZZ} = \prod_p \ZZ_p $.
\end{proof}

\begin{theorem}[Gaschutz's lemma for finite groups]
Let $ f : G \twoheadrightarrow H $ be a surjective homomorphism of finite groups. Suppose $ G $ has some generating set of size $ d $. For any generating set $ \cbr{z_1, \dots, z_d} \subseteq H $, there exists a generating set $ \cbr{x_1, \dots, x_d} \subseteq G $ such that $ f\br{x_i} = z_i $ for all $ i $.
\end{theorem}

Really, talking about generating vectors $ \underline{x} = \br{x_1, \dots, x_d} \in G^d $. Extend $ f $ to $ f : G^d \to H^d $.

\begin{proof}
We will prove, by induction on $ \abs{G} $, for $ H $ fixed, the following statement. The number
$$ N_G\br{\underline{y}} = \abs{\cbr{\text{generating vectors} \ \underline{x} \ \text{of} \ G \st f\br{\underline{x}} = \underline{y}}}, $$
where $ \underline{y} \in H^d $ is a generating vector of $ H $, is independent of $ \underline{y} $. Want to show $ N_G\br{\underline{z}} > 0 $, and $ G $ has some generating vector $ \underline{x'} \in G^d $ so $ N_G\br{\underline{z}} = N_G\br{f\br{\underline{x'}}} > 0 $. Let $ \underline{y} \in H^d $ be a generating vector. Let
$$ \CCC = \cbr{\text{$ d $-generator proper subgroups of} \ G}. $$
Every $ \underline{x} \in G^d $ such that $ f\br{\underline{x}} = \underline{y} $ either generates $ G $ or generates some $ C \in \CCC $. Therefore
$$ N_G\br{\underline{y}} + \sum_{C \in \CCC} N_C\br{\underline{y}} = \abs{\cbr{\underline{x} : f\br{\underline{x}} = \underline{y}}} = \abs{\ker f}^d. $$
Thus $ N_G\br{\underline{y}} = \abs{\ker f}^d - \sum_{C \in \CCC} N_C\br{\underline{y}} $, which is independent of $ \underline{y} $ by induction.
\end{proof}

\begin{theorem}[Gaschutz's lemma for profinite groups]
Let $ f : G \to H $ be a continuous surjective homomorphism of profinite groups. Suppose $ G $ has a TGS of size $ d $. Then for any TGS $ \cbr{z_1, \dots, z_d} $ of $ H $, there is a TGS $ \cbr{x_1, \dots, x_d} $ of $ G $ such that $ f\br{x_i} = z_i $ for all $ i $.
\end{theorem}

\begin{proof}
By Proposition \ref{prop:1.3.7} and Proposition \ref{prop:2.4.11} we may assume and write $ G = \varprojlim_{j \in J} G_j $ and $ H = \varprojlim_{j \in H} H_j $, surjective inverse systems of finite groups, with a morphism of inverse systems $ \br{f_j} : \br{G_j} \to \br{H_j} $ such that $ f = \varprojlim_j f_j $. It is forced that $ f_j $ is surjective, since
$$
\begin{tikzcd}
G \arrow[twoheadrightarrow]{r}{f} \arrow[twoheadrightarrow]{d}[swap]{p_j^G} & H \arrow[twoheadrightarrow]{d}{p_j^H} \\
G_j \arrow[twoheadrightarrow]{r}[swap]{f_j} & H_j
\end{tikzcd}.
$$
Let $ \underline{z} $ be the given TGS of $ H $. Set $ \underline{z}_j $ for $ j \in J $ to be the image of $ \underline{z} $ in $ H_j $. Then $ \underline{z}_j = p_j^H\br{\underline{z}} $ is a generating vector of $ H_j $. Consider the finite sets
$$ X_j = \cbr{\text{generating vectors} \ \underline{x}_j \in G_j^d \st f_j\br{\underline{x}_j} = \underline{z}_j} \ne \emptyset, $$
by Gaschutz. The $ X_j $ form an inverse system, so $ \phi_{ij}\br{X_i} \subseteq X_j $, since
$$
\begin{tikzcd}
G_i \arrow{r} \arrow{d} & H_i \arrow{d} \\
G_j \arrow{r} & H_j
\end{tikzcd}.
$$
Therefore $ \varprojlim_j X_j $ is non-empty. If $ \underline{x} \in \varprojlim_j X_j \subseteq G^d $ such that $ p_j^G\br{\underline{x}} \in X_j $, then $ \underline{x} $ is a TGS of $ G $ and $ p_j^H\br{f\br{\underline{x}}} = \underline{z}_j $ for all $ j $, so $ f\br{\underline{x}} = \underline{z} $.
\end{proof}

\pagebreak

\section{Profinite completions}

\subsection{Residual finiteness}

\begin{notation*}
Discrete abstract groups will be Greek letters and profinite groups will be Roman letters.
\end{notation*}

Given an abstract group $ \Gamma $ and an inverse system $ \NNN = \cbr{N \triangleleft_\f \Gamma} $, there is an inverse system of finite groups $ \Gamma / N $. Then $ \widehat{\Gamma} = \varprojlim_{N \in \NNN} \Gamma / N $, where $ \Gamma / N_1 \to \Gamma / N_2 $ if $ N_1 \le N_2 $. Also had a canonical morphism $ \iota_\Gamma = \iota : \Gamma \to \widehat{\Gamma} $. The image of $ \iota $ is dense by Corollary \ref{cor:1.2.30}. Also implies for any finite generating set $ S \subseteq \Gamma $, $ \iota\br{S} $ is a TGS of $ \widehat{\Gamma} $, so if $ \Gamma $ is finitely generated, then $ \widehat{\Gamma} $ is TFG.

\begin{proposition}
\label{prop:3.1.2}
Let $ f : \Delta \to \Gamma $ be a group homomorphism. Then there exists a unique continuous group homomorphism $ \widehat{f} : \widehat{\Delta} \to \widehat{\Gamma} $ such that $ \widehat{f} \circ \iota_\Delta = \iota_\Gamma \circ f $, so
$$
\begin{tikzcd}
\Delta \arrow{r}{f} \arrow{d}[swap]{\iota_\Delta} & \Gamma \arrow{d}{\iota_\Gamma} \\
\widehat{\Delta} \arrow[dashed]{r}[swap]{\widehat{f}} & \widehat{\Gamma}
\end{tikzcd}.
$$
\end{proposition}

\begin{proof}
Uniqueness will follow from the density of $ \iota_\Delta\br{\Delta} $ in $ \widehat{\Delta} $. Take two $ \widehat{f_1} $ and $ \widehat{f_2} $ satisfying Proposition \ref{prop:3.1.2}. Consider
$$ S = \cbr{\delta \in \widehat{\Delta} \st \widehat{f_1}\br{\delta} = \widehat{f_2}\br{\delta}}. $$
Then $ S $ is closed, since it is the preimage of the diagonal in $ \widehat{\Gamma} \times \widehat{\Gamma} $ under $ \br{\widehat{f_1}, \widehat{f_2}} : \widehat{\Delta} \to \widehat{\Gamma} \times \widehat{\Gamma} $, and $ S $ contains $ \iota_\Delta\br{\Delta} $, which is dense. So $ S = \widehat{\Delta} $.
\begin{enumerate}[leftmargin=0.5in, label=Case \arabic*.]
\item $ \Gamma $ is finite, so $ \Gamma = \widehat{\Gamma} $. Then $ \ker f $ is a finite index normal subgroup $ M $ of $ \Delta $, so there exists a projection map $ p_M : \widehat{\Delta} \to \Delta / M $. So we get a composition
$$
\begin{tikzcd}
\Delta \arrow{rr}{\iota_\Delta} \arrow{dr} \arrow{ddr}[swap]{f} & & \widehat{\Delta} \arrow{dl}[swap]{p_M} \arrow[dashed]{ddl}{\widehat{f}} \\
& \Delta / M \arrow[hookrightarrow]{d} & \\
& \Gamma &
\end{tikzcd}.
$$
\item General case. Take some $ N \triangleleft_\f \Gamma $. There exists a unique $ q_N : \widehat{\Delta} \to \Gamma / N $ such that $ q_N \circ \iota_\Delta = p_N \circ \iota_\Gamma \circ f $. Then $ \br{q_N} $ form a cone on the inverse system, since
$$ \phi_{N_1N_2}^\Gamma \circ q_{N_1} \circ \iota_\Delta = \phi_{N_1N_2}^\Gamma \circ p_{N_1} \circ \iota_\Gamma \circ f = p_{N_2} \circ \iota_\Gamma \circ f = q_{N_2} \circ \iota_\Delta. $$
Thus there exists a unique $ \widehat{f} : \widehat{\Delta} \to \widehat{\Gamma} $ such that $ p_N \circ \widehat{f} = q_N $ for all $ N $, so
$$
\begin{tikzcd}
\Delta \arrow{rr}{\iota_\Delta} \arrow{d}[swap]{f} & & \widehat{\Delta} \arrow[dashed]{dl}[swap]{\widehat{f}} \arrow{ddl}{q_N} \\
\Gamma \arrow{r}{\iota_\Gamma} \arrow{dr} & \widehat{\Gamma} \arrow{d}[swap]{p_N} & \\
& \Gamma / N &
\end{tikzcd},
$$
and
$$ p_N \circ \widehat{f} \circ \iota_\Delta = q_N \circ \iota_\Delta = p_N \circ \iota_\Gamma \circ f. $$
\end{enumerate}
\end{proof}

\begin{corollary}
$ \widehat{\cdot} $ is a functor.
\end{corollary}

\end{document}